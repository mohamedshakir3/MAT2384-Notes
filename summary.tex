
\documentclass{article}
\usepackage[landscape]{geometry}
\usepackage{url}
\usepackage{multicol}
\usepackage{amsmath}
\usepackage{esint}
\usepackage{amsfonts}
\usepackage{tikz}
\usetikzlibrary{decorations.pathmorphing}
\usepackage{amsmath,amssymb}
\usepackage{listings}
\usepackage{colortbl}
\usepackage{xcolor}
\usepackage{mathtools}
\usepackage{amsmath,amssymb}
\usepackage{enumitem}
\usepackage{environ}
\makeatletter

\newcommand*\bigcdot{\mathpalette\bigcdot@{.5}}
\newcommand*\bigcdot@[2]{\mathbin{\vcenter{\hbox{\scalebox{#2}{$\m@th#1\bullet$}}}}}
\makeatother

\newcommand{\laplace}[1]{\mathcal{L}\left\{#1\right\}}
\newcommand{\ilaplace}[1]{\mathcal{L}^{-1}\left\{#1\right\}}

\title{SI 2132 Midterm Cheat SheetC}
\usepackage[brazilian]{babel}
\usepackage[utf8]{inputenc}

\advance\topmargin-.8in
\advance\textheight3in
\advance\textwidth3in
\advance\oddsidemargin-1.5in
\advance\evensidemargin-1.5in
\parindent0pt
\parskip2pt
\newcommand{\hr}{\centerline{\rule{3.5in}{1pt}}}
%\colorbox[HTML]{e4e4e4}{\makebox[\textwidth-2\fboxsep][l]{texto}


\definecolor{blue}{HTML}{A7BED3}
\definecolor{brown}{HTML}{DAB894}
\definecolor{pink}{HTML}{FFCAAF}


\newtheorem{theorem}{Theorem}[section]
\newtheorem{definition}{Definition}[section]
\newtheorem{fact}{Fact}[section]
\newtheorem{prop}{Proposition}[section]
\newtheorem{corollary}{Corollary}[section]





\tikzset{header/.style={path picture={
\fill[green, even odd rule, rounded corners]
(path picture bounding box.south west) rectangle (path picture bounding box.north east) 
([shift={( 2pt, 4pt)}] path picture bounding box.south west) -- 
([shift={( 2pt,-2pt)}] path picture bounding box.north west) -- 
([shift={(-2pt,-4pt)}] path picture bounding box.north east) -- 
([shift={(-6pt, 6pt)}] path picture bounding box.south east) -- cycle;
},
label={[anchor=west, fill=green]north west:\textbf{#1:}},
}} 

\tikzstyle{mybox} = [draw=black, fill=white, very thick,
    rectangle, rounded corners, inner sep=10pt, inner ysep=10pt]
\tikzstyle{fancytitle} =[fill=black, text=white, rounded corners, font=\bfseries]


\tikzstyle{bluebox} = [draw=blue, fill=white, very thick,
    rectangle, rounded corners, inner sep=10pt, inner ysep=10pt]
\tikzstyle{bluetitle} =[fill=blue, inner sep=4pt, text=white, font=\small]


\tikzstyle{brownbox} = [draw=brown, fill=white, very thick,
    rectangle, rounded corners, inner sep=10pt, inner ysep=10pt]
\tikzstyle{browntitle} =[fill=brown, inner sep=4pt, text=white, font=\small]

\tikzstyle{pinkbox} = [draw=pink, fill=white, very thick,
    rectangle, rounded corners, inner sep=10pt, inner ysep=10pt]
\tikzstyle{pinktitle} =[fill=pink, inner sep=4pt, text=white, font=\small]

\tikzstyle{redbox} = [draw=red!35, fill=white, very thick,
    rectangle, rounded corners, inner sep=10pt, inner ysep=10pt]
\tikzstyle{redtitle} =[fill=red!35, inner sep=4pt, text=white, font=\small]


\NewEnviron{brownbox}[1]{
    \begin{tikzpicture}
    \node[brownbox](box){%
    \begin{minipage}{0.9\textwidth}
    \BODY
    \end{minipage}};
    \node[browntitle, right=10pt] at (box.north west) {#1};
    \end{tikzpicture}
}

 \NewEnviron{redbox}[1]{
    \begin{tikzpicture}
    \node[redbox](box){%
    \begin{minipage}{0.9\textwidth}
    \BODY
    \end{minipage}};
    \node[redtitle, right=10pt] at (box.north west) {#1};
    \end{tikzpicture}
}
   
    

\NewEnviron{bluebox}[1]{%
\begin{tikzpicture}
    \node[bluebox](box){%
        \begin{minipage}{0.9\textwidth}
            \BODY
        \end{minipage}
    };
    
\node[bluetitle, right=10pt] at (box.north west) {#1};
\end{tikzpicture}
}

\NewEnviron{pinkbox}[1]{%
\begin{tikzpicture}
    \node[pinkbox](box){%
        \begin{minipage}{0.9\textwidth}
            \BODY
        \end{minipage}
    };
    
\node[pinktitle, right=10pt] at (box.north west) {#1};
\end{tikzpicture}
}



\NewEnviron{blackbox}[1]{%
\begin{tikzpicture}
    \node[mybox](box){%
        \begin{minipage}{0.3\textwidth}
        \raggedright
        \small{
            \BODY
        }
        \end{minipage}
    };
    
\node[fancytitle, right=10pt] at (box.north west) {#1};
\end{tikzpicture}
}


\begin{document}





\begin{multicols*}{3}
    \begin{blackbox}{Order 1 ODE's}
        {\footnotesize
        \textbf{Standard Form:} $y' = f(x,y)$
    
        \textbf{Differential Form:} $M(x,y)dx + N(x,y)dy = 0$
    
        \begin{bluebox}{Seperable First Order ODE's}
            An ODE is called seperate if it can be written as \\[-2ex]
            \[F(x)dx = G(y)dy\]
            \textbf{Steps to Solving.}
                \begin{enumerate}[leftmargin=7pt]
                    \item Write $y' = \frac{dy}{dx}$, and seperate the ODE to write it in the form $F(x)dx = G(y)dy$
                    \item Integrate both sides
                    \item If an initial condition is given, solve for the integration constant C. 
                \end{enumerate}
        \end{bluebox}
        \begin{brownbox}{First Order ODE's with Homogeneous Coefficients}
            $F(x,y)$ is called homogeneous of degree $k$ if it can be written\\[-4ex]
            \[F(\lambda x, \lambda y) = \lambda^k \cdot F(x,y)\]
            An ODE is differential form is homogeneous if $M(x,y)$ and $N(x,y)$ are homogeneous of the same degree.

            \textbf{Steps to Solving.}

                A homogeneous ODE can be made seperable by substituting $u = \frac{y}{x}$, or $u = \frac{x}{y}$.\\[-2ex]
                \[u = \frac{y}{x} \implies y = xu, \ dy = udx + xdu\]
                \vspace{-3ex}
                \[u = \frac{x}{y} \implies y = \frac{x}{u}, \frac{dy}{dx} = \frac{x - u\frac{dx}{du}}{u^2}\]
        \end{brownbox}
        
        \begin{bluebox}{Exact First Order ODEs}
            An ODE is called exact if $\frac{\partial M}{\partial y} = \frac{\partial N}{\partial x}$

            \textbf{Steps to Solving.}
                \begin{enumerate}[leftmargin=7pt]
                    \item Check exactness: $\frac{\partial M}{\partial y} = \frac{\partial N}{\partial x}$
                    \item Look for a function $F(x,y)$ such that $\frac{\partial F}{\partial x} = M$, $\frac{\partial F}{\partial y} = N$
                    Integrate $M$ with respect to $x$ or $N$ with respect to $y$ then differentiate the equation with respect to the other variable respectively.
                    \item The general solution is $F(x,y) = C$
                \end{enumerate}
        \end{bluebox}\\[-2ex]
        \begin{brownbox}{Integrating Factor}
            $\mu(x,y)$ is an integrating factor of a if the following equation is exact \\[-2ex]
            \[\mu(x,y)M(x,y)dx + \mu(x,y)N(x,y)dy = 0\]
            \textbf{Theorem.}
                If \\[-5ex]
                \[\frac{\frac{\partial M}{\partial y} - \frac{\partial N}{\partial x}}{M} = g(y)\]
                for some function $g(y)$, then\\[-2ex]
                \[\mu(y) = \exp\left(-\int g(y) dy\right)\]
                If \\[-5ex]
                \[\frac{\frac{\partial M}{\partial y} - \frac{\partial N}{\partial x}}{M} = f(x)\]
                for some function $f(x)$, then\\[-2ex]
                \[\mu(x) = \exp\left(\int f(x) dy\right)\]
        \end{brownbox}\\[-2ex]
        }
\end{blackbox}
        

\begin{blackbox}{Linear First-Order ODEs}
    {\footnotesize

    \textbf{Definition.} A first order ODE is called linear if it can be written in the form\\[-2ex]
    \[y' + f(x)y = r(x)\]
    \begin{bluebox}{Steps to Finding General Solution}
        Given a linear first-order ODE in the form $y' + f(x)y = r(x)$, find $y$ using 
        \[y = \left(\int \exp\left(\int f(x)dx\right)r(x)dx + C\right)\exp\left(-\int f(x)\right)\]
    \end{bluebox}\\[-2ex]
    }
\end{blackbox} 
\begin{blackbox}{Bernoulli ODE's}
    A first order ODE is of Bernoulli type if it can be written as \\[-4ex]
    \[y' + f(x)y= r(x)y^a\]
    \begin{redbox}{Steps to Solving}
        \begin{enumerate}[leftmargin=7pt]
            \item Let $u = y^{1-a}$, then compute $u' = (1-a)y^{-a}y'$.
            \item Isolate $y'$ from the original ODE and substitute into $u'$. 
            \item The resulting ODE is linear and solve for $u$.
        \end{enumerate}
    \end{redbox}\\[-2ex]
\end{blackbox}
\begin{blackbox}{Homogeneous ODEs}
    {\footnotesize
    \begin{bluebox}{Constant Coefficients}
        \textbf{General Form}\\[-2ex]
        \[a_ny^{(n)} + a_{n-1}y^{(n-1)} + \cdots + a_1y' + a_0y = 0\]
        \textbf{Characteristic Equation}\\[-2ex]
        \[\lambda^n + a_{n-1}\lambda^{(n-1)} + \cdots + a_1\lambda + a_0 = 0\]
        If $\lambda$ is a root with multiplicity $k$, 
        \[y_1 = e^{\lambda x}, \ y_2 = xe^{\lambda x}, \ y_3 = x^2e^{\lambda x}, \ldots, y_k = x^{k-1}e^{\lambda x}\]
        If $\alpha + i\beta$ is a pair of complex conjugate roots, then
        \[y_1 = e^{\alpha x}\cos(\beta x), \ y_2 = e^{\alpha x}\sin(\beta x)\]
    \end{bluebox}
    \begin{brownbox}{Euler-Cauchy}
        \textbf{General Form}\\[-2ex]
        \[a_n(x)y^{(n)} + a_{n-1}(x)y^{(n-1)} + \cdots + a_1(x)y' + a_0(x)y = 0\]
        \textbf{Characteristic Equation.} Differentiate $y = x^m$ and plug into ODE. If $m$ is a root of the characteristic equation of multiplicity $k$, then it contributes the following equations to our basis of solutions
        \[y_1 = x^m, y_2 = x^m\ln x, y_3 = x^m (\ln x)^2, y_k = x^m(\ln x)^{k-1}\]
        If $\alpha \pm i\beta$ is a pair of complex conjugate roots of the characteristic equation, then the pair contributes the following 2 equations to our basis of solutions
        \[y_1 = x^{\alpha} \cos(\beta \ln x), y_2 = x^\alpha\sin(\beta \ln x)\]
    \end{brownbox}\\[-2ex]
    }
\end{blackbox}

\begin{blackbox}{Undetermined Coefficients}
    \textbf{General Form}\\[-2ex]
    \[a_ny^{(n)} + \cdots + a_1(x)y' + a_0y = r(x)\]
    All coefficients on the left ($a_n, \ldots, a_0$) are constants and $r(x)$ is a polynomial, exponential, and/or sinusoidal. 
    % Solve by splitting the ODE $y = y_H + y_p$, $y_H$ is the coresponding homogeneous ODE, $y_p$ is decided my the rules below.
    \begin{pinkbox}{Rules}
        \renewcommand{\arraystretch}{1.25}
        \begin{center}
            \begin{tabular}{|c|c|}
                \hline
                $r(x)$ & $y_p$\\
                \hline
                $Ke^{\lambda x}$ & $Ae^{\lambda x}$\\
                $p_n(x)$ & $q_n(x)$\\
                $K \sin(wx)$ & $A \cos(wx) + B \sin(wx)$\\
                $Ke^{\alpha x} \sin(wx)$ & $ A e^{\alpha x}\cos(wx) + Be^{\alpha x} \sin(wx)$\\
                $p_n(x)e^{\alpha x}$ & $q_n(x)e^{\alpha x}$\\
                \hline
            \end{tabular}
        \end{center}
        Where $p_n(x)$ and $q_n(x)$ are polynomials of degree $n$. 
    \end{pinkbox}\\[-2ex]
\end{blackbox}
\begin{blackbox}{Variation of Parameters}
    \textbf{General Form}\\[-2ex]
        \[a_n(x)y^{(n)} + \cdots + a_1(x)y' + a_0y(x) = r(x)\]
        Same solution with $y = y_H + y_p$ where $y_H$ is the solution to the coresponing homogeneous ODE\\[-1.5ex]
        \[a_n(x)y^{(n)} + \cdots + a_1(x)y' + a_0y(x) = 0\]
        \[y_p = u_1y_1 + u_2y_2 + \cdots u_ny_n\]
        Where $\{y_1,y_2,\ldots,y_n\}$ is a basis of solutions for the corresponding homogeneous ODE. $u_1, u_2, \ldots, u_n$ are functions that satisfy the following system of equations 
        \[\begin{cases}
            0 = u_1'y_1 + u2'y_2 + \cdots + u_n'y_n \\
            0 = u_1'y_1' + u_2'y_2' + \cdots + u_n'y_n' \\
            \vdots\\
            \frac{r(x)}{a_n(x)} = u_1'y_1^{(n-1)} + u_2'y_2^{(n-1)} + \cdots + u_n'y_n ^{(n-1)} 
        \end{cases}\]
\end{blackbox}
\begin{blackbox}{Solving IVP's With Laplace Transforms}
    \textbf{Steps to Solving.}
    {\footnotesize
    
    \begin{enumerate}[leftmargin=7pt]
        \item Set $Y = \mathcal{L}\{y(t)\}$.
        \item Apply Laplace transform to both sides of the ODE. 
        \item Isolate $Y$ after preforming the Laplace transform. 
        \item Then solve $y(t) = \mathcal{L}^{-1}\{Y\}$.
    \end{enumerate}
    }
    \vspace{-1.25ex}
    \begin{redbox}{Laplace of a derivative}
        \[\mathcal{L}\{y'(t)\} = s\mathcal{L}\{y(t)\} - y(0)\]
        \[\mathcal{L}\{y''(t)\} = s^2\mathcal{L}\{y(t)\} - sy(0) - y'(0)\]
        \[\mathcal{L}\{y^{(n)}\} = s^n\mathcal{L}\{y(t)\} - s^{n-1}y(0) - \cdots - y^{(n-1)}(0)\]
    \end{redbox}\\[-2ex]
\end{blackbox}
\begin{blackbox}{Systems of ODE's}
    {\footnotesize
    \textbf{General Form}\\[-2ex]
    \[\begin{rcases}
        y_1' = a_{11}y_1 + a_{12}y_2 + r_1(x)\\
        y_1' = a_{21}y_1 + a_{22}y_2 + r_2(x)  
    \end{rcases} \implies \vec{y'} = A\vec{y} + \vec{r}(x)\]
    \[A = \begin{bmatrix}
        a_{11} & a_{12}\\
        a_{21} & a_{22}
    \end{bmatrix}, \vec{y'} = \begin{bmatrix}
        y_1'\\
        y_2'
    \end{bmatrix}, \vec{r}(x) = \begin{bmatrix}
        r_1(x)\\
        r_2(x)
    \end{bmatrix}\]
    \begin{redbox}{Homogenous Systems with Constant Coefficients}
        A system is homogeneous if $\vec{r}(x) = 0$, $\vec{y'} = A\vec{y}$.
        \begin{brownbox}{Steps to Solving}
            \begin{enumerate}[leftmargin=5pt]
                \item Find the eigen values of $A$ $|\lambda I - A| = 0$
                \item If 2 distinct real eigenvalues, find eigenvectors $V_1,V_2$.\\[-5ex]
                \[\vec{y} = c_1\vec{V_1}e^{\lambda_1x} + c_2\vec{V_2}e^{\lambda_2x}\]
                \item If $\lambda$ with multiplicity 2, find generalized eigenvector $\rho$ \\[-4ex]
                \[Y = c_1Ve^{\lambda x} + c_2(xV + \rho)e^{\lambda x}\]                
                \item If $\lambda = \alpha \pm i\beta$, then find eigenvector for $\lambda_1 = \alpha + i\beta$, compute gen. solution $y = c_1V_1 + c_2V_2$ with
                \[\vec{V}e^{(\alpha + i\beta)x} = \vec{V}e^{\alpha x}(\cos(\beta x) + i\sin (\beta x)) = \vec{V_1} + i\vec{V_2}\]
        \end{enumerate}
        \end{brownbox}
    \end{redbox}\\[-2ex]
    \begin{bluebox}{Non-Homogeneous Systems}
        Similar to non-homogeneous ODEs, use undetermined to solve $\vec{y} = \vec{y_H} + \vec{y_p}$. $y_H$ is ODE with $\vec{r}(x) = 0$. Decompose $\vec{r}(x)$ as 
        {\scriptsize
        \[\vec{r(x)} = \begin{bmatrix}
            2 x^3 + x^2 + x \\
            3e^x + x^2 + 2x + 1
        \end{bmatrix} = \begin{bmatrix}
            2\\
            0
        \end{bmatrix}x^3 + \begin{bmatrix}
            0\\
            3
        \end{bmatrix}e^x + \begin{bmatrix}
            1\\
            1
        \end{bmatrix}x^2 + \begin{bmatrix}
            1\\
            2
        \end{bmatrix}x\]
        }
        Solve each $r_1(x)$ using same rules as undetermined coefficients replacing constants with constant vectors
        \[r_1(x) = \begin{bmatrix}
            2\\0
        \end{bmatrix}x^3 \implies \vec{y}_p = \vec{U}x^3 + \vec{V}x^2 + \vec{W}x + \vec{Z} \]
    \end{bluebox}\\[-2ex]
    }
\end{blackbox}
\begin{blackbox}{Linear Algebra and Trig Identities}
    {\footnotesize
    \begin{pinkbox}{Linear Algebra}      
        \textbf{Eigen Values}\\[-3ex]
        \[|A - \lambda I| = |\lambda I - A| = 0\]
        \[A = \begin{bmatrix}
            a & b\\
            c & d
        \end{bmatrix} \implies A^{-1} = \frac{1}{\det A}\begin{bmatrix}
            d & -b\\
            -c & a
        \end{bmatrix}\]
        \textbf{Eigen Vectors.} An eigenvector $\vec{V}$ is a vector that satisfies\\[-5ex]
        \[[A - \lambda I | \vec{0}]\]
        \textbf{Generalized Eigen Vector $\rho$} The solution to, then pick specific value for the parameter $t$ to get a specific vector, (i.e take $t=0$, $t=1$, etc).\\[-2ex]
        \[[A - \lambda I | V]\]
    \end{pinkbox}
    \begin{redbox}{Trig Identities}
        \vspace{-2ex}
        \[\cos^2\left(\frac{t}{2}\right) = \frac{1+\cos(t)}{2}, \  2\sin(t)\cos(t) = \sin(2t)\]
        \vspace{-3ex}
    \end{redbox}\\[-3ex]
    }
\end{blackbox}
\begin{blackbox}{Example of Non-Homogeneous System}
    {\footnotesize
        \textbf{Example.}\\[-2ex] 
        \[\vec{y'} = \begin{bmatrix}
            9 & 18\\
            -2 & -3
        \end{bmatrix}\vec{y} + \begin{bmatrix}
            9x - 51\\
            7 + e^{2x}
        \end{bmatrix}; \ \vec{y}(0) = \begin{bmatrix}
            1\\
            0
        \end{bmatrix}\]
        \textbf{Solution.} Solve the corresponding homogeneous ODE
        \[\vec{y}' = \begin{bmatrix}
            9 & 18\\
            -2 & -3
        \end{bmatrix}\vec{y}\]
        Find eigenvalues of $A$
        \[\det\begin{bmatrix}
            9 - \lambda & 18\\
            -2 & -3 - \lambda\\
        \end{bmatrix} = (9 - \lambda)(-3-\lambda) + 36 = \lambda^2 -6\lambda + 9\]
        This gives us $\lambda = 3$ with multiplicity 2, find eigenvector $V = t\begin{bmatrix}-3\\1\end{bmatrix}$
        Find generalized eigenvector, set parameter $t = 0$,
        \[[A - 3I| V] = \begin{bmatrix}
            6 & 18 & -3\\
            -2 & -6 & 1
        \end{bmatrix} \sim \begin{bmatrix}
            1 & 3 & -1/2\\
            0 & 0 & 0
        \end{bmatrix}\]
        \[\vec{\rho} = \begin{bmatrix}
            -3t - 1/2\\
            t
        \end{bmatrix} \implies \begin{bmatrix}
            -1/2\\
            0
        \end{bmatrix}\]
        The general solution to the homogeneous ODE is 
        \begin{align*}
            \vec{y}_H &= c_1\begin{bmatrix}-3\\1\end{bmatrix}xe^{3x} + c_2\left(x\begin{bmatrix}
                -3\\
                1
            \end{bmatrix} + \begin{bmatrix}
                -1/2\\
                0
            \end{bmatrix}\right)e^{3x}  \\
            &= \begin{bmatrix}
                -3c_1e^{3x} - 3c_2xe^{3x} - \frac{1}{2}c_2e^{3x}\\
                c_1e^{3x} + c_2xe^{3x}
            \end{bmatrix}
        \end{align*}
        For $y_p$, decompose $r(x)$ to get
        \[r(x) = \begin{bmatrix}
            9x - 51\\
            7 + e^{2x}
        \end{bmatrix} = \begin{bmatrix}
            9\\
            0
        \end{bmatrix}x + \begin{bmatrix}
            -51\\
            7
        \end{bmatrix} + \begin{bmatrix}
            0\\
            1
        \end{bmatrix}e^{2x}\]
        Find $y_p$ for each part of $r(x)$\\[-2ex]
        \[y_p = Ux + V + We^{2x}\]
        Rewrite the non-homogeneous system as 
        \[\vec{y'} = \begin{bmatrix}
            9 & 18\\
            -2 & -3
        \end{bmatrix}y + \begin{bmatrix}
            9\\0
        \end{bmatrix}x + \begin{bmatrix}
            -51\\7
        \end{bmatrix} + \begin{bmatrix}
            0\\1
        \end{bmatrix}e^{2x}\]
       Compute $\vec{y'}_p$ and plug into the system to solve for constant vectors, $\vec{y'}_p = U + 2We^{2x}$, $\vec{y'}_p = A\vec{y}_p + \vec{r}(x)$\\[-2ex]
       {\scriptsize
       \[
        U + 2We^{2x} = A(Ux + V + We^{2x}) + \begin{bmatrix}
            9\\0
        \end{bmatrix}x + \begin{bmatrix}
            -51\\7
        \end{bmatrix} + \begin{bmatrix}
            0\\1
        \end{bmatrix}e^{2x}
        \]
    }
        This gives us the three equations 
        \[AU + \begin{bmatrix}
            9\\0
        \end{bmatrix} = 0, AV + \begin{bmatrix}
            -51\\7
        \end{bmatrix} = U, AW + \begin{bmatrix}
            0\\1
        \end{bmatrix} = 2W\]
        Solve equations to find\\[-2ex]
       \[U = \begin{bmatrix}
        3\\-2
       \end{bmatrix}, V = \begin{bmatrix}
        0\\3
       \end{bmatrix}, W = \begin{bmatrix}
        18\\-7
       \end{bmatrix}\]
        This gives us our particular solution
        \[
            y_p = \begin{bmatrix}
                3\\-2
            \end{bmatrix}x + \begin{bmatrix}
                0\\3
            \end{bmatrix} + \begin{bmatrix}
                18\\-7
            \end{bmatrix}e^{2x} = \begin{bmatrix}
                3x + 18e^{2x}\\
                -2x+3 - 7e^{2x}
            \end{bmatrix}  
        \]
        The general solution to the non-homogeneous ODE is 
        \[
            y = \begin{bmatrix}
                -3c_1e^{3x} - 3c_2xe^{3x} - \frac{1}{2}e^{3x}\\
                c_1e^{3x} + c_2xe^{3x}
            \end{bmatrix}
            + \begin{bmatrix}
                3x + 18e^{2x}\\
                -2x+3 - 7e^{2x}
            \end{bmatrix}
        \]
    }
\end{blackbox}
    \begin{blackbox}{Fixed Point Iteration}
        For a given starting value $x_0$, The iteration sequence is given as \\[-5ex]
        \[x_{n+1} = g(x_n)\]
        \textbf{Theorem.} Assume that the function $g(x)$ has a fixed-point $s$ on an interal $I$, if 
        \begin{itemize}
            \item $g(x)$ is continuous on $I$,
            \item $g'(x)$ is continuous on $I$, and
            \item $|g'(x)| < 1$ for all $x \in I$.
        \end{itemize}
        Then then the iteration sequence converges.
        \begin{redbox}{Steps to Solve Using Fixed-Point Iteration}
           
            Then the steps for solving are as follows,
            \begin{enumerate}[leftmargin=8pt]
                \item Start with $f(x) = 0$
                \item Rewrite $f(x) = 0$ under the form $x = g(x)$
                \item Verify the iteration sequence $x_0, x_1 = g(x_0), \ldots, x_n = g(x_{n-1})$ converges using the above theorem 
                \item Compute the terms of the sequence and stop when 2 terms have the same required digits.
            \end{enumerate}
        \end{redbox}\\[-2ex]
    \end{blackbox}
    \begin{blackbox}{Newton's Method}
        Given an equation equation $f(x) = 0$ and a starting point $x_0$, the Newton's method is given as\\[-2ex]
        \[x_{n+1} = x_n - \frac{f(x_n)}{f'(x_n)}\]
        \vspace{-1ex}
        Calculate values for $x_n$ until you reach the accuracy.\\
    \end{blackbox}
    \begin{blackbox}{Secant Method}
        Given 2 estimates for the roots $x_0$, $x_1$, compute the terms in the sequence until you reach the required accuracy.\\[-2ex]
        \[x_{n+1} = x_n - f(x_n)\cdot \frac{x_n - x_{n-1}}{f(x_n) - f(x_{n-1})}\]
        \vspace{-2.5ex}
    \end{blackbox}
    \begin{blackbox}{Lagrange Interpolation}
        Given points $(x_0, f_0), (x_1, f_1), \ldots, (x_n, f_n)$, the Lagrange interpolation polynomial is \\[-2ex]
        \[p_n(x) = L_0(x)f_0 + L_(1)(x)f_1 + \cdots + L_n(x)f_n\]
        where \\[-5ex]
        {\footnotesize
        \begin{align*}
            L_0 &= \frac{(x-x_1)(x-x_2)\cdots(x-x_n)}{(x_0-x_1)(x_0-x_2)\cdots(x_0-x_n)}  \\
            L_1 &= \frac{(x-x_0)(x-x_2)\cdots(x-x_n)}{(x_1-x_0)(x_1-x_2)\cdots(x_1-x_n)}\\
            L_2 &= \frac{(x-x_0)(x-x_1)(x-x_3)\cdots(x-x_n)}{(x_2-x_0)(x_2-x_1)(x_2-x_3)\cdots(x_2-x_n)}
        \end{align*}
        
        The error formula for interpolation is \\[-4ex]
        \[|f(x) - p_n(x)| = \left|(x-x_0)(x-x_1)\cdots(x-x_n)\frac{f^{(n+1)(t)}}{(n+1)!}\right| \]
        }
    \end{blackbox}
    \begin{blackbox}{Newton's Divided-Difference Interpolation}
        Given a node $x_i$, 
        \begin{enumerate}[leftmargin=8pt]
            \item The \emph{first divided difference} at $x_i$ is defined as 
            \[f(x_i,x_{i+1}) = \frac{f_{i+1}- f_i}{x_{i+1} - x_i}\]
            \item The \emph{second divided difference} at $x_i$ is 
            \[f(x_i, x_{i+1}, x_{i+2}) = \frac{f(x_{i+1}, x_{i+2}) - f(x_i,x_{i+1})}{x_{i+2}-x_i}\]
            \item In general, the $k$th divided difference at $x_i$ is 
            {\footnotesize
            \[f(x_i, x_{i+1}, \ldots x_{i+k}) = \frac{f(x_{i+1},\ldots, x_{i+k}) - f(x_i, \ldots x_{i+k-1})}{x_{i+k}-x_i}\]
            }
        \end{enumerate}
        \textbf{Newton's Interpolation Polynomial} 
        \begin{align*}
            p_n(x) &= f_0 + f(x_0,x_1)(x-x_0)\\
            &+ f(x_0,x_1,x_2)(x-x_0)(x-x_1) + \cdots\\
            &+ f(x_0,\ldots,x_n)(x-x_0)(x-x_1)\cdots(x - x_n)
        \end{align*}
    \end{blackbox}
    \begin{blackbox}{Numerical Integration}
        {\footnotesize
            \begin{bluebox}{Midpoint Rule}
                \[\int_a^b f(x)dx \approx h[f(x_1^*) + f(x_2^*) + \cdots + f(x_n^*)]\]
                \[h = \frac{b-a}{n}, x_i^* = \frac{x_i + x_{i+1}}{2}\]
                Error formula with $M$ being $|f''(x)| \leq M$ for $x \in [a,b]$
                \[E_M \leq \frac{M(b-a)^3}{24n^2}\]
            \end{bluebox}
            \begin{brownbox}{Trapezoidal}
                \[\int_a^b f(x)dx\approx \frac{h}{2}\left[f(a) + 2f(x_1) + 2f(x_2) + \cdots + f(x_n)\right]\]
                \textbf{Error Formula}\\[-2ex]
                \[|E_T| \leq \frac{M(b-a)^3}{12n^2}\]
            \end{brownbox}
            \begin{pinkbox}{Simpsons Rule}
                Divide $[a,b]$ into an \emph{EVEN} number of subintervals\\[-2ex]
                \[\int_a^b f(x) \approx \frac{h}{3}\left[f(a) + 4(x_1) + 2f(x_2) + 4f(x_3) + \cdots + f(b)\right]\]
                \[|E_S| \leq \frac{M(b-a)^5}{180n^4}\]
                \textbf{Error Formula} Where  $M$ is the upperbound for the fourth derivative of $f(x)$\\[-2ex]
                \[|E_S| \leq \frac{M(b-a)^5}{180n^4}\]
            \end{pinkbox}\\[-2ex]
        }
    \end{blackbox}
    \begin{blackbox}{Gaussian Quadrature}
        To convert $\int_a^bf(x)dx$ into the form $\int_{-1}^1g(t)dt$, use the substitution\\[-4ex]
        \[x = \frac{b-a}{2}t + \frac{b+a}{2}\]
        Then the Gaussian Quadrature formula is\\[-2ex]
        \[\int_{-1}^{1}f(t)dt \approx w_1f(t_1) + \cdots w_nf(t_n)\]
        \begin{redbox}{Table of Nodes and Coefficients}
            \begin{center}
                \begin{tabular}{c|cc}
                    Order $n$ & Nodes $t_i$ & Coefficients $w_i$\\
                    \hline
                    1 & 0 & 2\\
                    \hline
                    2 & $-0.5773502692$ & 1\\
                    & $0.5773502692$ & 1\\
                    \hline
                    & $-0.7745966692$ & $0.555555556$\\
                    3 & 0 & 0.888888889\\
                    & $0.7745966692$ & $0.555555556$\\
                    \hline
                    & -0.8611363116 &0.3478548451\\
                    4&-0.3399810436& 0.6521451549\\
                    &0.3399810436 &0.6521451549\\
                    &0.8611363116 &0.3478548451\\
                    \hline
                    &-0.9061798459 &0.2369268850\\
                    &-0.5384693101& 0.4786286705\\
                    5 &0.0 &0.5688888889\\
                    &0.5384693101 &0.4786286705\\
                    &0.9061798459& 0.2369268850\\
                \end{tabular}
            \end{center}
        \end{redbox}\\[-2ex]
    \end{blackbox}
    \begin{blackbox}{Euler Method}
        Given a step size $h$ between $x$ values, we estimate values of $y$ with \\[-5ex]
        \[x_{n+1} = x_n + h\]
        \[y_{n+1} = y_n + hf(x_n, y_n)\]
        \vspace{-4ex}
    \end{blackbox}
    \begin{blackbox}{Improved Euler Method}
        We use Euler's method to predict then correct each step with \\[-5ex]
        \[x_{n+1} = x_n + h\]
        \[y^c_{n+1} = y_n^c + \frac{h}{2}(f(x_n, y_n^c) + f(x_{n+1},y^p_{n+1}))\]
        \[y^p_{n+1} = y_n^c + hf(x_n, y_n)\]
        \vspace{-4ex}
    \end{blackbox}
    \begin{blackbox}{Runge-Kutta Method of Order 4}
        Given $y(x_0) = y_0$, use the following formula to compute \\[-2ex]
        {\footnotesize
        \[x_{n+1} = x_n + h\]
        \[k_1 = hf(x_n,y_n)\]
        \[k_2 = hf\left(x_n + \frac{1}{2}h, y_n + \frac{1}{2}k_1\right)\]
        \[k_3 = hf\left(x_n + \frac{1}{2}h, y_n + \frac{1}{2}k_2\right)\]
        \[k_4 = hf\left(x_n + h, y_n + k_3\right)\]
        \[y_{n+1} = y_n + \frac{1}{6}(k_1 + 2k_2 + 2k3 + k_4)\]
        }
        \vspace{-4ex}
    \end{blackbox}
    \begin{blackbox}{Examples for Numerical Integration}
        \begin{bluebox}{Midpoint Rule}
            
            {\small
            \textbf{Example.} Estimate the following integral with a maximal absolute error of $0.001$.\\[-2ex]
            \[I = \int_0^{0.5}x\cos x dx\]
            \textbf{Solution.} We have to first find $n$ \\[-2ex]
            \[f(x) = x\cos x, \ f'(x) = \cos x - x\sin x\]
            \[f''(x) = -\sin x - \sin x - x\cos x = -2\sin x - x\cos x \]
            We can see that \\[-2ex]
            \[|-2\sin x - x\cos x| \leq |-2\sin x| + |-x\cos x| \leq 2.5\]
            Thus $M = 2.5$, then by the error formula we have\\[-2ex]
            {\footnotesize
            \[|E_m| \leq \frac{2.5(0.5-0)}{24n^2} \leq 0.001  \implies n \geq \sqrt{\frac{2.5(0.5)^3}{24(0.001)}} = 2.79\]
            }
            We take $n = 3$. Then we can calculate $h$,\\[-2ex]
            \[h = \frac{0.5 -0}{3} = \frac{1}{6}, \ x_1^* = \frac{0 + 1/6}{2} = \frac{1}{12}\]
            \[x_2^* = x_1^* + \frac{1}{6} = \frac{1}{4}, \ x_3^* = x_2^* + \frac{1}{6} = \frac{5}{12}\]
            By the midpoint rule, we have\\[-1.5ex]
            \[\int_{0}^{0.5} x\cos x dx \approx h[f(x_1^*) + f(x_2^*) + f(x_3^*)]\]
            \[= \frac{1}{6}\left[\frac{1}{12}\cos \left(\frac{1}{12}\right) + \frac{1}{4}\cos \left(\frac{1}{4}\right) + \frac{5}{12}\cos \left(\frac{5}{12}\right)\right]\]
            }
            \vspace{-2.25ex}
        \end{bluebox}
        \begin{brownbox}{Trapezoidal Rule}
            {\small
            \textbf{Example.} Estimate the value of the integral with a maximal absolute error of $0.01$ to 
            \[\int_0^1 e^{-x^2}dx\]
            \vspace{-0.5ex}
            \textbf{Solution.} First we compute $n$, \\[-2ex]
            {\scriptsize
            
            \[f(x) = e^{-x^2}, \ f'(x) = -2xe^{-x^2}, \ f''(x) = -2e^{-x^2} + 4x^2e^{-x^2}\]
            \[f''(x) = -2e^{-x^2} + 4x^2e^{-x^2}, \ f'''(x) = e^{-x^2}4x(3-2x^2)\]
            }
            $f''(x)$ is increasing, therefore \\[-1ex]
            {\scriptsize
            \[f''(0) \leq f''(x) \leq f''(1) \implies -2 \leq f''(x) \leq -2e^{-1} + 4e^{-1}\]
            }
            Thus we can take $M =2$. Then by the error formula we have\\[-2ex]
            \[\frac{2}{12n^2} \leq 0.01 \implies n \geq \sqrt{\frac{1}{6(0.01)}} \approx 4.08\]
            Take $n = 5$. Length of each subinterval is $h = \frac{1 -0}{5} = 0.2$. \\[-1.5ex]
            {\footnotesize
            \[\frac{0.2}{2}\left[f(0) + 2f(0.2) + 2f(0.4) +2f(0.6) + 2f(0.8) + f(1)\right]\]
            }
            }
            \vspace{-2.25ex}
        \end{brownbox}\\[-1ex]
    \end{blackbox}
    \begin{blackbox}{Simpson's Rule}
            \textbf{Example.} Estimate the value of the integral with a maximal error of $0.001$
            \[\int_{0.5}^{1.5}x^2\ln x dx\]
            \textbf{Solution.} We start by computing $n$,\\[-2ex]
            \[f(x) = x^2\ln x, \ f'(x) = 2x\ln x + x, \ f''(x) = 2\ln x + 3\]
            \[f'''(x) = \frac{2}{x}, \ f^{(4)}(x) = -\frac{2}{x^2}, \ f^{(5)}(x) = \frac{4}{x^3}\]
            $f^{(4)}$ is increasing, therefore\\[-2ex]
            \[f^{(4)}(0.5) \leq f^{(4)}(x) \leq f^{(4)}(1.5) \implies |f^{(4)}(x)| \leq 8\]
            
            So we can take $M = 8$. Then by the error formula we have\\[-2ex]
            {\footnotesize
            \[\frac{8(1.5-0.5)}{180n^4} \implies n \geq \sqrt[4]{\frac{8}{180(0.001)}} \approx 2.58\]
            }
            We need an even $n$ so we take $n=4$. Then we can calculate $h$,
            {\footnotesize
            \[h = \frac{1}{4}, \ x_0 = 0.5, \ x_1 = 0.5 + \frac{1}{4} = 0.75, \ x_2 = 0.75 + \frac{1}{4} = 1\]
            \[x_3 = 1 + \frac{1}{4} = 1.25, \ x_4 = 1.25 + \frac{1}{4} = 1.5\]
            }
            Then we can approximate the integral,
            \[\int_{0.5}^{1.5} x^2\ln x \approx 0.123915\]
    \end{blackbox}
    \begin{blackbox}{Guassian Quadrature}
        Use Gaussian Quadrature of order 4 to estimate the value of \\[-2ex]
        \[\int_{0}^1\sin(x^2)dx\]
        \textbf{Solution.} First we substitute $x$ with \\[-1ex]
        \[x = \frac{b-a}{2}t + \frac{b+a}{2} = \frac{1}{2}t +\frac{1}{2}\]
        \vspace{-2ex}
        \[\frac{dx}{dt} = \frac{1}{2} \implies dx = \frac{dt}{2}\]
        \[\int_0^1 \sin(x^2) = \int_{-1}^1 \sin \left(\frac{(t+1)^2}{4}\right) \frac{1}{2}dt\]
        From the table we have \\[-2ex]
        \[w_1 = w_4 = 0.3479, \ w_2 = w_3 = 0.6521\]
        \[f(t_1) = -f(t_4) = - 0.8611, \  f(t_2) = - f(t_3) = -0.3399\]
        Then using the formula
        {\footnotesize
        \[\int_{-1}^1 \sin \left(\frac{(t+1)^2}{4}\right) \frac{1}{2}dt \approx w_1f(t_1) + w_2f(t_2) + w_3(t_3) + w_4f(t_4)\]
        }
    \end{blackbox}
    \begin{blackbox}{Estimating IVP's}
        \begin{bluebox}{Euler's Method}
            \textbf{Example.} Use Euler's method with $h=0.2$ to estimate the IVP on $[0,0.6]$.\\[-2ex]
            \[y' = 2x + y, \ y(0) = -1\]

            \textbf{Solution.} We have $f(x,y) = 2x + y$, $x_0 = 0$, $y_0 = -1$, and $h = 0.2$. Now we can caluclate each step with $x_1 = 0.2$, $x_2 = 0.04$, $x_3 = 0.6$. 
            {\footnotesize
            \[y_1 = y_0 + h f(x_0,y_0) = -1 + 0.2(-1) = -1.2\]
            \[y_2 = y_1 + hf(x_1,y_1) = -1.2 + 0.2(2(0.2) - 1.2) = -1.36\]
            \[y_3 = y_2 + hf(x_2,y_2) = -1.36 + 0.2(2(0.4) - 1.36) = -1.472\]
            }
        \end{bluebox}
        \begin{brownbox}{Improved Euler's Method}
            \textbf{Example.} Using the same IVP as previous problem,
            We're given $f(x,y) = 2x + y$, $h = 0.2$, $y_0 = -1$, and $x_0 = 0$, $x_1 = 0.2$, $x_2 = 0.04$, $x_3 = 0.6$. 
            {\footnotesize
            \begin{align*}
                y_1^p &= y_0 + h f(x_0, y_0) = -1 + 0.2(-1) = -1.2\\
                y_1^c &= y_0 + \frac{h}{2}\left[f(x_0,y_0) + f(x_1,y_1^p)\right]\\
                      &= -1 + 0.1 \left[-1 + 2(0.2) - 1.2\right] = -1.18\\
                y_2^p &= y_1^c + h f(x_1, y_1^c) = -1.18 + 0.2(2(0.2) - 1.18)\\
                      &= -1.336\\
                y_2^c &= y_1^c + \frac{h}{2}\left[f(x_1,y_1^c) + f(x_2,y_2^p)\right]\\
                      &= -1.18 + 0.1\left[2(0.2) - 1.18 + 2(0.4)\right] = -1.3116\\
                y_3^p &= y_2^c + hf(x_2, y_2^c) = -1.3116 + 0.2(2(0.4) - 1.3116)\\
                      &= -1.41392\\
                y_3^c &= -1.3116 + 0.1\left[2(0.4) - 1.18 + 2(0.6) - 1.413192\right]\\
                      &= -1.384152
            \end{align*}
            }
        \end{brownbox}
        \begin{redbox}{Runge-Kutta Method of Order 4}

            \textbf{Example.} \\[-5ex]
            \[y' = y - x^2 + 1; \ y(0) = \frac{1}{2}\]            
            We can compute with a step size $h = 0.5$ on the interval $[0,0.5]$, $x_1 = 0.5$.
            \vspace{-2ex}
            {\footnotesize
            \begin{align*}
                k_1 &= hf(x_0, y_0) = 0.5\left(\frac{1}{2} - 0 + 1\right) = 0.75\\
                k_2 &= hf\left(x_0 + \frac{1}{2}h, y_0 + \frac{1}{2}k_1\right)\\
                &= 0.5f(0.25,0.875) = 0.90625\\
                k_3 &= hf\left(x_0 + \frac{h}{2}, y_0 + \frac{1}{2}k_2\right)\\
                &= 0.5f(0.25, 0.953125) = 0.9453125\\
                k_3 &= hf\left(x_0 + h, y_0 + k_3\right)\\
                &= 0.5f(0.5, 1.4453125) = 1.09765625\\
            \end{align*}
            \vspace{-8ex}
            }

            For $x_2 = 1$, the process repeats and we get 2 new points. 
            \end{redbox}\\[-2ex]
        \end{blackbox}
        \begin{blackbox}{Laplace Transforms}
        %     \begin{bluebox}{Computing Laplace Transforms}
        %         {\footnotesize
        %         \textbf{Example.} Find \\[-4ex]
        %         \[\ilaplace{\frac{2s+3}{s^3 + 5s^2 + 8s + 4}}\]
        %         \textbf{Solution.} Start by decomposing the fraction to get \\[-2ex]
        %         \begin{align*}
        %             \frac{2s+3}{s^3+5s^2+8s + 4} &= \frac{2s+3}{(s+1)(s+2)^2}\\
        %             &=\frac{1}{s+1} - \frac{1}{s+2}  + \frac{1}{(s+2)^2}
        %         \end{align*}
        %         \vspace{-4ex}

        %         Then, \\[-3ex]
        %         % \[\]
        %         % \[= \ilaplace{\frac{1}{s+1}} - \ilaplace{\frac{1}{s+2}} + \ilaplace{\frac{1}{(s+2)^2}}\]
        %         % \[=e^{-t} - e^{-2t} + te^{-2t}\]
        %         \[
        %             \ilaplace{\frac{2s+3}{s^3+5s^2 + 8s+ 4}} =  e^{-t} - e^{-2t} + te^{-2t}
        %         \]
        %         \[\ilaplace{\frac{1}{(s+2)^2}} = te^{-2t}\tag{$1^{st}$ shift. theorem.}\]
        %         }
        %     \end{bluebox}
            \begin{redbox}{Convolution}
                \textbf{Example.} Find \\[-4ex]
                {\footnotesize
                \[\ilaplace{\frac{1}{s^2 + 4s - 5}}\]
                \textbf{Solution.} Rewrite the fraction and use the convolution, \\[-3ex]
                \begin{align*}
                    &\ilaplace{\frac{1}{s^2 + 4s - 5}} = \ilaplace{\frac{1}{s-1} \frac{1}{s+5}} \\
                    &= \ilaplace{\frac{1}{s-1}} * \ilaplace{\frac{1}{s+5}}\\
                    &= e^{t} * e^{-5t} = \int_0^t e^{t-x}e^{-5x}dx\\
                    &= \int_0^t e^{t}e^{-x}e^{-5x}dx = e^t \int_0^t e^{-6x}dx\\
                    &= e^t\left(-\frac{1}{6}e^{-6t} + 1\right) = \frac{1}{6}e^t - \frac{1}{6}e^{-5t}
                \end{align*}
                }
            \end{redbox}
            \begin{brownbox}{Solving an IVP}
                {\footnotesize
                \textbf{Example.} Solve the following IVP \\[-2ex]
                \[y'' + 6y' + 9y = \begin{cases}
                    0 & 0 \leq t < 2\\
                    e^{-3t} & t \geq 0
                \end{cases}, \ y(0) = 1, \ y'(0) = 0\]
                
                \noindent
                \textbf{Solution.} We can rewrite the ODE as \\[-2ex]
                \[y'' + 6y' + 9y = e^{-3t}u(t-2)\]
                Applying the laplace transform to both sides,\\[-2ex] 
                \[\laplace{y''} + 6\laplace{y'} + 9\laplace{y} = \laplace{e^{-3t}u(t-2)}\]
                \vspace{-3.5ex}
                \[s^2Y - sy(0) - y'(0) + 6(sY - y(0)) + 9Y= \frac{e^{-2(s+3)}}{s+3}\]
                This gives us \\[-4ex]
                \[Y = \frac{s+6}{(s+3)^2} + \frac{e^{-2(s+3)}}{(s+3)^3}\]
                Now we can compute the inverse laplace transform of $Y$\\[-2ex]
                \[y(t) = \ilaplace{Y} = \ilaplace{\frac{s+6}{(s+3)^2}} + \ilaplace{\frac{e^{-2(s+3)}}{(s+3)^3}}\]
                The first part simply requires partial fractions, \\[-2ex]
                \[\ilaplace{\frac{s+6}{(s+3)^2}} = \ilaplace{\frac{1}{s+3}} + 3\ilaplace{\frac{1}{(s+3)^2}}\]
                \[=e^{-3t} + 3te^{-3t}\] 
                The second part requires second shifting theorem with $e^{-2s}$ and $f(t) = \ilaplace{\frac{1}{(s+3)^3}}$. \\[-2ex]
               \begin{align*}
                \ilaplace{\frac{e^{-2(s+3)}}{(s+3)^3}} &= \ilaplace{\frac{e^{-2s}e^{-6}}{(s+3)^3}}\\
                &= e^{-6}\ilaplace{e^{-2s}\frac{1}{(s+3)^3}} \\
                f(t) &= \ilaplace{\frac{1}{(s+3)^3}}\\
                &= \frac{1}{2}\ilaplace{\frac{2!}{(s+3)^3}} =  \frac{1}{2}t^2e^{-3t}\\
            \end{align*}
            \vspace{-9.25ex} 

            Thus,\\[-3ex] 
            \[y(t) = e^{-3t} + 3te^{-3t} + e^{-6}u(t-2)\frac{1}{2}(t-2)^2e^{-3(t-2)}\]
                }
                \end{brownbox}\\[-2ex]
            \end{blackbox}
    \end{multicols*}
\end{document}



% \begin{blackbox}{Exact First Order ODE's}
%     \begin{bluebox}{Steps To Solving}
%         \begin{enumerate}[align=left]
%             \item Check exactness: $\frac{\partial M}{\partial y} = \frac{\partial N}{\partial x}$
%             \item Look for a function $F(x,y)$ such that\\[-0.0001mm]
%             \[\frac{\partial F}{\partial x} = M, \ \frac{\partial F}{\partial y} = N\]
%             Do this by integrating $M$ with respect to $x$ or $N$ with respect to $y$ then differentiate the equation with respect to the other variable respectively.
%             \item The general solution is $F(x,y) = C$
%             \item If an initial condition is given, solve for the integration constant $C$.
%         \end{enumerate}
%     \end{bluebox}
%     {\footnotesize
%     \textbf{Example.} Solve the following IVP \\[-2ex]
%     \[(6x - 2y^2 + 2xy^3)dx + (3x^2y^2-4xy)dy = 0\]
%     \textbf{Solution.} Check for exactness \\[-2ex]
%     \[\frac{\partial M}{\partial y} = -4y + 6xy^2 = \frac{\partial N}{\partial x}\]
%     \[F(x,y) = \int 3x^2y^2 - 4xydy = x^2y^3 - 2xy^2 + h(x)\]
%     \[\frac{\partial F}{\partial x} = 2xy^3 - 2y^2 + h'(x) = M \implies h'(x) = 6x\]
%     \[h(x) = \int 6xdx = 3x^2 + k\]
%     Our general solution is  \\[-2ex] 
%     \[F(x,y) = x^2y^3 - 2xy^2 + 3x^2 = C\]
%     }
% \end{blackbox} 
