\documentclass[openany]{report}
\usepackage[utf8]{inputenc}

\usepackage{stylesheets}
\usepackage{lecture_notes_styles}
\usepackage{pgfplots}
\pgfplotsset{compat=1.18}

\newcommand{\powerset}[0]{\mathcal{P}}

\title{MAT 2384: Ordinary Differentials Lecture Notes}
\author{Last Updated:}

\begin{document}

\maketitle

\tableofcontents
\setcounter{chapter}{-1}
\chapter{Introduction and Basic Terminology}
\begin{definition}[Differential Equations]
    A differential equation is an equation involving an unknown function $y$ (of one or many variables), derivatives of $y$, and other known functions of indepedent variables.
\end{definition}
\begin{definition}[Order of Differential Equations]
    The \emph{order} of a differential equation is the highest order of a derivative appearing in the equation.
\end{definition}
If the unknown function $y$ is a function of only one variable, $y = f(x)$, we saw that the differential equation is \emph{ordinary}. If $y$ is a function of two or more variables, we say the differential equation is a \emph{partial} differential equation. \\[2ex]
\textbf{Example.}
\[x^3y'' - 3e^x \sin x y' + 3y = \tan x\]
This is an ODE of order 2. \\[2ex]
\textbf{Example.}
\[x_1x_2 \frac{\partial^2y}{\partial x_1\partial x_2} - 3e^{x_1} \frac{\partial y}{\partial x_1} = 0\]
This is a PDE of order 2. \\[2ex]
\begin{center}
    \textbf{Note:} In this course, we will only consider ODEs.
\end{center}
\begin{definition}
    We say that the function $y$ is a \emph{solution} to a differential equation on an interval $I$ if $y$ is well-defined on $I$ and $y$ satisfies the differential equation.
\end{definition}
\textbf{Example.} Consider the differential equation 
\[y'' - 5y' + 4y = 0\]
Show that the function 
\[y = Ae^x + Be^{4x}\]
is a solution for the differential equation on $\real$ for any constants $A$ and $B$. \\[2ex]
\textbf{Solution:} We have $y = Ae^x + Be^{4x}$ is well defined on $\real$. 
\[y' = Ae^x - 4Be^{4x}\]
\[y'' = Ae^x + 16Be^{4x}\]
So, 
    \[y'' - 5y' + 4y = Ae^x + 16Be^{4x} - 5Ae^x - 20Be^{4x} + 4Ae^x + 4Be^{4x} = 0\]
Therefore, $y = Ae^x + Be^{4x}$ is a solution to the differential equation for any $A,B \in \real$. This is called the \emph{general solution} to the differential equation. \\[2ex]
\textbf{Remark:} The above example shows that a differential equation has infinitely many solutions.
\begin{definition}[Initial Value Problem]
    An \emph{intial value problem} (IVP) of order n consists of an ordinary differential equation of order $n$, and $n$ initial coniditions of the form 
    \[y(x_0) = y_0, \ \ y'(x_0) = y_1 \ldots\]
    \[y^{(n-1)}(x_0) = y_{n-1}\]
\end{definition}
\begin{center}
    \textbf{Note:} $y^{(i)}$ denotes the $i$th derivative of $y$.
\end{center}
\textbf{Example.} Consider the IVP of order 3 
\[y''' - 3e^xy'' + 6xy' + 2y = x^2\]
\[y(0) = -1 \ \ y'(0) = 2 \ \ y''(0) = 1\]
\textbf{Example.} Solve the following IVP
\[y'' - 5y + 4y = 0\]
\[y(0) = 1 \ \ y'(0) = 2\]
\textbf{Solution:} We saw in the previous example that the general solution to this differential equation is 
\[y = Ae^x + Be^{4x}\]
We can use the initial conditions to find the constants $A$ and $B$.
\[y(0) = 1 \implies 1 = Ae^0 + Be^0 = A + B\]
\[y'(0) = 2 \implies 2 = Ae^0 - 4Be^0 = A + 4B\]
\[A + 4B - A - B = 2 - 1 \implies 3B = 1 \implies B = \frac{1}{3} \ \ A = \frac{2}{3}\]
\begin{theorem}[Existence and Uniqueness Theorem for the First Order ODEs]
    Consider the IVP: 
    \[y' = F(x,y), \ \ y(x_0) = y_0\]
    \begin{itemize}
        \item \textbf{Existence:} If $F(x,y)$ is continuous in an open rectangular region 
        \[R = \left\{(x,y) \in \real^2: a < x < b, c < y < d\right\}\]
        of the $xy$-plane that contains the initial point $(x_0, y_0)$, then there exists a solution $y(x)$ to the intial value problem that is defined in some open interval $I = (\alpha, \beta)$ containg $x_0$.
        \item \textbf{Uniqueness:} If the partial derivative $\frac{\partial F}{\partial y}$ of the function $F(x,y)$ is continuous in the recnagular region $R$, then the solution $y(x)$ is unique.
    \end{itemize}
\end{theorem}
\begin{center}
    \textbf{Note:} We will always suppose this condition is satisfied in this course.
\end{center}

\chapter{Ordinary Differential Equations of First Order}
The goal of this chapter is to solve ODE's of order 1. 
\begin{definition}
    The \emph{standard form} of an ODE of order 1 is an expression of the form 
    \[y' = f(x,y)\]
    We can rewrite $y'$ as $\frac{dy}{dx}$ and we have the \emph{differential form} 
    \[M(x,y)dx + N(x,y)dy = 0\]
\end{definition}
\noindent
\textbf{Example.} Consider the differential equation 
\[2xy' + 3y = 2y' + \sin x\]
The standard form is
\[2xy ' - 2y' = \sin x - 3y \implies y' = \frac{\sin x - 3y}{2x-2}\]
The differential form is 
\begin{align*}
    &2x\frac{dy}{dx} + 3y = 2\frac{dy}{dx} + \sin x   \\
    \implies&2xdy + 3ydx = 2dy + \sin x dx\\
    \implies& (3y - \sin x)dx (2x-2)dy = 0
\end{align*}
\section{Seperable First Order Ordinary Differential Equations}

\begin{definition}
    A first order ODE is called \emph{seperable} if it can be written in the form
    \[F(x)dx = G(y)dy\]
\end{definition}
\subsection{Solving Seperable ODE's} 
To solve a seperable ODE,
\begin{enumerate}
    \item Write $y' = \frac{dy}{dx}$
    \item Seperate the ODE to write it in the form 
    \[F(x)dx = G(y)dy\]
    \item Take integrals of both sides
    \item If an initial condition is given, solve for the constant of integration $C$. 
\end{enumerate}
\textbf{Example.} Solve the IVP 
\[(y^2 + 1)y' = \frac{x}{y} \ \ y(1) = 1\]
\textbf{Solution:} We can write $y' = \frac{dy}{dx}$ and we get 
    \[(y^2 + 1)\frac{dy}{dx} = \frac{x}{y} \implies (y^2 + 1)ydy = xdx\]
    Taking integrals on both sides, we have 
    \[\int y^3 + ydy = \int xdx \implies \frac{y^4}{4} + \frac{y^2}{2} = \frac{x^2}{2} + C\]
    Using our initial condition, we have $y = 1$ when $x =1$, then 
    \[\frac{1}{4} + \frac{1}{2} = \frac{1}{2} + C\]
    Therefore $C = \frac{1}{2}$ and the solution to the IVP is
    \[\frac{y^4}{4} + \frac{y^2}{2} = \frac{x}{2} + \frac{1}{4}\]
    This is called the \emph{implicit solution} since we could not explcitly solve for $y$ in terms of $x$. \\[2ex]

\textbf{Example.} Solve the IVP
\[e^xy' = (x+1)y^2 \ \ y(0) = -\frac{1}{2}\]
\textbf{Solution:} 
\begin{align*}
    &e^x\frac{dy}{dx} = (x+1)y^2 \\
    \implies& \frac{1}{y^2}dy = \frac{x+1}{e^x}dx \\
    \implies& \int \frac{1}{y^2}dy = \int (x+1)e^{-x}dx \\
\end{align*}
We can use integration by parts to solve the right hand side integral. Let $u = x+1$ and $dv = e^{-x}dx$, $u' = 1$, and $v = -e^{-x}$. Then 
\begin{align*}
    \int (x+1)e^{-x}dx &= uv - \int u'vdx\\
    &= -(x+1)e^{-x} - \int -e^{-x}dx\\
    &= -(x+1)e^{-x} - e^{-x} + C\\
\end{align*}
Therefore we have 
\begin{align*}
    \frac{y^{-2 +1}}{-2 + 1} &= -(x+1)e^{-x} - e^{-x} + C\\
    -\frac{1}{y} &= -(x+1)e^{-x} - e^{-x} + C\\
\end{align*}
Setting $y = -\frac{1}{2}$ and $x = 0$, we have 
\[2 = -2 + C \implies C = 4\]
Therefore the implicit solution is 
\[-\frac{1}{y} = -(x+1)e^{-x} - e^{-x} - 4\]
We can rewrite this as an explicit solution as 
\[y = \frac{1}{(x+2)e^{-x}-4}\]
\section{First Order ODE's With Homogeneous Coefficients}
\begin{definition}
    A function $F(x,y)$ of two variables is called \emph{homogeneous} of degree $k$ if 
    \[F(\lambda x, \lambda y) = \lambda^k \cdot F(x,y)\]
\end{definition}
This type of ODEs can be made seperable after a suitable change of variables of the unknown function.\\[2ex]
\noindent
\textbf{Example.}
\[F(x,y) = 3x^2y - 2xy^2 + y^3\]
We can check if its homogeneous by the definition, 
\begin{align*}
    F(\lambda x, \lambda y) &= 3(\lambda x)^2 (\lambda y) - 2(\lambda x) (\lambda y)^2 + (\lambda y^3)\\
    &= 3\lambda^3x^2y - 2\lambda^3xy^2 + \lambda^3y^3\\
    &= \lambda^3(3x^2y - 2xy^2 + y^3)\\
    &= \lambda^3F(x,y)
\end{align*}
Therefore, $F(x,y)$ is homogeneous of degree 3. We can tell quickly if a polynomial is homogeneous is by looking at the exponents of each term. If the sum of the exponents of each term is the same, then the polynomial is homogeneous, with order being the sum of the exponents in each term (i.e $x^2y$ has exponents 2,1, $xy^2$ has exponents 1,2, and $y^3$ has exponents 3, each sum to 3).\\[2ex]
\begin{definition}
    A first order ODE given in differential form
    \[M(x,y)dx + N(x,y)dy = 0\]
    is called of \emph{homogeneous coefficients} if both $M(x,y)$ and $N(x,y)$ are homogeneous of the same degree.
\end{definition}
\noindent
\textbf{Example.}
\[(3x^2+2y^2+2xy)dx - 4xydy = 0\]
Both terms are homogeneous of degree 2, therefore this is a differential equation of homogeneous coefficients. 
\begin{theorem}
    A first order ODE of homogeneous coefficients can be made seperable by changing the function using one of the following substitutions:
    \begin{itemize}
        \item Set $u \coloneqq \frac{y}{x}$ or
        \item $u \coloneqq \frac{x}{y}$
    \end{itemize}
\end{theorem}
\noindent
\textbf{Example.} Solve the following IVP 
\[(x^2-y^2)dx + 2xydy = 0 \ \ y(1) = 2\]
\textbf{Solution:} This is a first order ODE with homogeneous coefficients. Let 
\[u \coloneqq \frac{y}{x} \implies y = xu\]
\[\frac{dy}{dx} = 1 \cdot u + x \cdot \frac{du}{dx} \implies dy = udx + xdu\]
So, we have 
\[(x^2-y^2)dx + 2xydy = 0 \implies (x^2 - x^2u^2)dx + 2x(xu)(udx + xdu) = 0\]
Simplyfing, we get 
\begin{align*}
    x^2dx - x^2u^2dx + 2x^2u^2dx + 2x^3udu &= 0\\
    dx - u^2dx + 2u^2dx + 2xudu &= 0\\
    (1 + u^2)dx + 2xudu &= 0\\
    (1+u^2)dx &= -2xudu\\
    -\frac{1}{x}dx &= \frac{2u}{1+u^2}du
\end{align*}
Now that it's seperable, we can integrate both sides,
\begin{align*}
    -\int \frac{1}{x}dx &= \int \frac{2u}{1+u^2}du\\
    -\ln(x) &= \ln(1+u^2) + C\\
\end{align*}
Now using our initial condition, we have $y(1) = 2$. But, our differential equation is a function of $u$ not $y$, so we must calculate $u$ using our initial condition. So, $u(1) = \frac{y(1)}{1} = 2$. So, 
\[-\ln 1 = \ln 5 + C \implies C = -\ln 5\]
Therefore, our solution is
\begin{align*}
    \ln x &= \ln(1 + u^2) - \ln 5\\
    \ln\left(\frac{5}{x}\right) &= \ln(1 + u^2)\\
    \frac{5}{x} &= 1 + u^2\\
    u^2 &= \frac{5}{x} - 1\\
    \frac{y^2}{x^2} &= \frac{5}{x} - 1\\
    y &= \sqrt{5x - x^2}
\end{align*}
We take the positive square root since if we took the negative square root, then $y(1) = -2$ which is not our initial condition.\\[2ex]
\textbf{Example.} Solve the IVP 
\[(2x + y)dx - xdy = 0 \ \ y(1) = -2 \ \ x > 0\]
\textbf{Solution:} This is a first order ODE with homogeneous coefficients. Let
\[u = \frac{y}{x} \implies y = xu\]
\[dy = udx + xdu\]
Substituting into our differential equation, we get
\begin{align*}
    (2x + xu)dx - x(udx + xdu) &= 0\\
    (2+u)dx - (udx + xdu) &= 0\\  
    2dx + udx - udx - xdu &= 0\\
    2dx &= xdu \implies \frac{2}{x}dx = du
\end{align*}
This differential equation in $u$ is sperable, so we can integrate 
\begin{align*}
    \int \frac{2}{x}dx &= \int du\\
    2\ln x &= u + C\\
\end{align*}
Using your initial condition, $y(1) = -2$. so $u(1) = \frac{y(1)}{1} = -2$. Therefore,
\[2\ln 1 = -2 + c \implies C = 2\]
Now solving for $y$, 
\begin{align*}
    u &= 2\ln x - 2\\
    \frac{y}{x} &= 2\ln x - 2\\
    y &= x(2\ln x - 2)
\end{align*}
This is our explicit solution to the initial value problem. 
\section{Exact First Order ODEs}
\begin{definition}
    Given a function $F(x,y)$ of two variables, the differential of $F(x,y)$ denoted by $dF$ is defined by 
    \[dF = \frac{\partial F}{\partial x} dx + \frac{\partial F}{\partial y}dy\]
\end{definition}
\noindent
\textbf{Example.} Let 
\[F(x,y) = 2x^2y^3 + \sin(x+2y)\]
Then 
\[dF = (4xy^3 + \cos(x+2y))dx + (6x^2y^2 + 2\cos(x+2y))dy\]
\textbf{Remark:}
\[dF = 0 \iff \frac{\partial F}{\partial x} dx + \frac{\partial F}{\partial y} dy = 0 \iff \frac{\partial F}{\partial x} = 0 \text{ and } \frac{\partial F}{\partial y} = 0 \]
So, $F(x,y) = C$ is a constant function. Therefore, 
\[dF = 0 \iff F(x,y) = C\]
\begin{definition}
    A first order ODE 
    \[M(x,y)dx + N(x,y)dy = 0\]
    is called \emph{exact} if there exists a continuous function $F(x,y)$ such that 
    \[\frac{\partial F}{\partial x} = M(x,y) \text{ and } \frac{\partial F}{\partial x} = N(x,y)\]
    So if $M(x,y)dx + N(x,y)dy = 0$ is exact, then 
    \[dF = 0 \implies F(x,y) = C\]
\end{definition}
In summary, if $M(x,y)dx + N(x,y)dy = 0$ is exact, then find $F(x,y)$ such that
\[\frac{\partial F}{\partial x} = M(x,y) \text{ and } \frac{\partial F}{\partial x} = N(x,y)\]
Then, the (implicit) solution to the ODE is $F(x,y) = C$. Furthermore, since $M(x,y) = \frac{\partial F}{\partial x}$ and $N(x,y) = \frac{\partial F}{\partial y}$, then 
\[\frac{\partial M}{\partial y} = \frac{\partial}{\partial y} \left(\frac{\partial F}{\partial x}\right) = \frac{\partial^2F}{\partial x \partial y}\]
\[\frac{\partial N}{\partial x} = \frac{\partial}{\partial x} \left(\frac{\partial F}{\partial y}\right) = \frac{\partial^2F}{\partial y \partial x}\]
So by the Clairaut-Schwarz Theorem, the ODE is exact if and only if
\[\frac{\partial M}{\partial y} = \frac{\partial N}{\partial x}\]
\begin{theorem}[Condition for Exactness]
    The first order ODE $M(x,y)dx + N(x,y)dy = 0$ (with $M,N$ continuous) is exact if and only if
    \[\frac{\partial M}{\partial y} = \frac{\partial N}{\partial x}\]
\end{theorem}
\subsection{Steps to Solving Exact ODEs}
\begin{enumerate}
    \item Check exactness: $\frac{\partial M}{\partial y} = \frac{\partial N}{\partial x}$
    \item Look for a function $F(x,y)$ such that
    \[\frac{\partial F}{\partial x} = M \ \ \frac{\partial F}{\partial y} = N\]
    \item The general solution to the ODE is $F(x,y) = C$.
    \item If an intial condition is given, use it to find $C$.
\end{enumerate}
\textbf{Example.} Solve the following IVP
\[(6x - 2y^2 + 2xy^3)dx + (3x^2y^2 - 4xy)dy = 0, \ \ y(1) = -2\]
\textbf{Solution.} We first check exactness.
\[\frac{\partial M}{\partial y} = -4y + 6xy^2 = 6xy^2 - 4y\]
\[\frac{\partial N}{\partial x} = 6xy^2 - 4y\]
Therefore, this ODE is exact. Now we need to find a function $F(x,y)$ satisfying the partial derivatives. We can do this by integrating $N$ with respect to $y$, so we have 
\[\frac{\partial F}{\partial y} = 3x^2y^2 - 4xy\]
\[F(x,y) = \int 3x^2y^2-4xy dy = 3x^2 \int y^2dy - 4x \int ydy = x^2y^3 - 2xy^2 + h(x)\]
We add $h(x)$ since when integrating with respect to $y$, we are treating $x$ as a constant so $h(x)$ is constant with respect to $y$. So we have 
\[F(x,y) = x^2y^3 - 2xy^2 + h(x)\]
Now we can use the first equation to solve for $h(x)$, 
\[\frac{\partial F}{\partial x} = 2xy^3 - 2y^2 + h'(x)\]
This equation is equal to $M$, so we can plug $M$ in and get 
\[M = 6x-2y^2 + 2xy^3 = 2xy^3 - 2y^2 + h'(x) \implies h'(x) = 6x\]
Now we can solve for $h(x)$ by taking the integral, 
\[h(x) = \int 6xdx = 3x^2 + C_1\]
Now, we get 
\[F(x,y) = x^2y^3 - 2xy^2 + 3x^2 + C_1\]
So the general solution to the ODE is
\[x^2y^3 - 2xy^2 + 3x^2 + C_1 = C_2 \implies x^2y^3 - 2xy^2 + 3x^3 = C\]
Now using the intial condition $y(1) = - 2$, then 
\[1^2 (-2)^3 - 2(1)(-2)^2 + 3(1)^2 = C \implies C  = -13\]
Therefore, the solution to the IVP is 
\[x^2y^3 - 2xy^2 + 3x^3 = -13\]
\textbf{Example.} Solve the IVP 
\[(2x\cos(y) - 3x^2y + ye^{xy})dx + (-x^2\sin(y)+xe^{xy} - x^3)dy = 0, \ \ y(0) = 1 \]
\textbf{Solution.} We first check exactness,
\[\frac{\partial M}{\partial y} = -2x\sin(y) - 3x^2 + e^{xy} + ye^{xy}\]
\[\frac{\partial N}{\partial x} = -2x\sin(y) + ye^{xy} + e^{xy} -3x^2\]
Therefore this ODE is exact, so we look for $F(x,y)$ such that 
\[\frac{\partial F}{\partial x} = M \text{ and } \frac{\partial F}{\partial y} = N\]
\begin{align*}
    F(x,y) &= \int 2x\cos(y) - 3x^2y + ye^{xy}dx\\
    &= 2\cos(y)\int xdx - 3y \int x^2dx + y\int e^{xy}dx\\
    &=x^2 - x^3y + y\frac{e^{xy}}{y} + h(y)
\end{align*}
So we have
\[F(x,y) = x^2\cos(y) - x^3y + e^{xy} + h(y)\]
Then, 
\[\frac{\partial F}{\partial y} = -x^2\sin(y) - x^3 + xe^{xy} + h'(y) = N \implies h'(y) = 0\]
So $h(y)$ is a constant, say $h(y) = K$, then our general solution for $F(x,y)$ is 
\[F(x,y) = x^2\cos(y) - x^3y + e^{xy} + k \implies x^2\cos(y) - x^3y + e^{xy} = C\]
Using the condition, $y(0) = 1$, we get
\[(0)^2\cos(1) - (0)^3(1) + e^{0\cdot 1} = C \implies C = 1 \]
Therefore the (implicit) solution to the IVP is
\[x^2\cos(y) - x^3y + e^{xy} = 1\]
\section{First Order ODEs With an Integrating Factor}

\begin{definition}[Integrating Factor]
    We say that the function $\mu(x,y)$ is an \emph{integrating factor} of the first-order ODE 
    \[M(x,y)dx + N(x,y)dy = 0\]
    if the new ODE
    \[\mu(x,y)M(x,y)dx + \mu(x,y)N(x,y)dy = 0\]
    is exact.
\end{definition}
In general, finding an integrating factor is not easy. However, there are some special cases where we can find an integrating factor easily.
\begin{theorem}
    For the ODE
    \[M(x,y)dx + N(x,y)dy = 0\]
    \begin{enumerate}
        \item If 
        \[\frac{\frac{\partial M}{\partial y} - \frac{\partial N}{\partial x}}{M} = g(y)\]
        for some function $g$ of $y$ only, then an integration factor exists given by 
        \[\mu(y) = \exp \left(-\int g(y)dy\right)\]
        \item If
        \[\frac{\frac{\partial M}{\partial y} - \frac{\partial N}{\partial x}}{M} = f(x)\]
        for some function $f$ of $x$ only, then an integration factor exists given by
        \[\mu(x) = \exp\left(\int f(x)dx\right)\]
    \end{enumerate}
\end{theorem}
\textbf{Example.} Solve the IVP 
\[(y^4 + xy)dx + (xy^3 - x^2 + 2y^3e^y)dy = 0, \ \ y(0) = 1\]
\textbf{Solution.} It's clear this ODE is not exact, so we need to find an integrating factor. 
\[\frac{\partial M}{\partial y} - \frac{\partial N}{\partial x} = 4y^3 + x - y^3 + 2x = 3y^3 + 3x\]
If we dvidie by $M$, we get 
\[\frac{\frac{\partial M}{\partial y} - \frac{\partial N}{\partial x}}{M} = \frac{3(y^3 + x)}{y^4 + xy} = \frac{3(y^3 + x)}{y(y^3 + x)} = \frac{3}{y}\]
Therefore, we have our integrating factor 
\[\mu(y) = \exp \left(-\int \frac{3}{y}dy\right) = \exp\left(-3\int\frac{1}{y}dy\right) = \exp\left(\ln(y^{-3})\right) = y^{-3}\]
We multiply the original ODE with $\mu(y) = y^{-3}$
\[y^{-3}(y^4 + xy)dx + y^{-3}(xy^3 - x^2 + 2y^3e^y)dy = (y + xy^2)dx + (x - x^2y^{-3}2e^y)\]
Now we can check the exactness of this ODE, 
\[\frac{\partial M}{\partial y} = 1 - 2xy^{-3} \text{ and } \frac{\partial N}{\partial x} = 1-2xy^{-3}\]
Therefore, this ODE is exact, so we look for $F(x,y)$ such that
\[\frac{\partial F}{\partial x} = M \text{ and } \frac{\partial F}{\partial y} = N\]
The first equation is simpler so we will start with that,
\begin{align*}
    F(x,y) &= \int y + xy^{-2}dx\\
    &= y \int dx + y^{-2}\int xdx\\
    &= xy + \frac{x^2y^{-2}}{2}
\end{align*}
Now we derive with respect to $y$ and use the second equation, 
\[\frac{\partial F}{\partial y} = x - x^2y^{-3} + h'(y) = N = x - x^2y^{-3} + 2e^y \implies h'(y) = 2e^y\]
Then 
\[h(y) = \int h'(y)dy = \int 2e^ydy = 2e^{y} + k\]
So we get the function 
\[F(x,y) = xy + \frac{x^2y^{-2}}{2} + 2e^y + k\]
Then setting $F(x,y)$ equal to a constant to get our (implicit) general solution, 
\[xy+ \frac{x^2y^{-2}}{2} + 2e^y = C\]
Using the initial condition, $y(0) = 1$, we get
\[(0)(1) + \frac{(0)^2(1)^{-2}}{2} + 2e^{1} = C \implies C = 2e\]
Therefore, the (implicit) solution to the IVP is
\[xy+ \frac{x^2y^{-2}}{2} + 2e^y = 2e\]
\textbf{Example.} Solve the following IVP
\[(x^2+4xy + 3y^2)dx + (x^2 + 2xy)dy = 0, \ \ y(1) = 1, \ x > 0\]
\textbf{Solution.} We can see that this function is of homogeneous coefficients, but we will solve it using the integrating factor. First we calculate the partial derivatives,
\[\frac{\partial M}{\partial y} = 4x + 6y\]
\[\frac{\partial N}{\partial x} = 2x + 2y\]
we see that these are not equal so this ODE is not exact, then we calculate the difference,
\[\frac{\partial M}{\partial y} - \frac{\partial N}{\partial x} = 2x + 4y\]
Then to obtain a function of only $x$, we divide by $N$, 
\[\frac{\frac{\partial M}{\partial y} - \frac{\partial N}{\partial x}}{N} = \frac{2x + 4y}{x^2 + 2xy} = \frac{2(x+2y)}{x(x + 2y)} = \frac{2}{x} = f(x)\]
An integrating factor exists and is given by
\[\mu(x) = \exp\left(\int(f(x)dx)\right) = \exp\left(2\ln x\right) = x^2\]
Now we can multiply the original ODE by $\mu(x) = x^2$,
\[(x^2 + 4x^3y + 3x^2y^2)dx + (x^4 + 2x^3y)dy = 0\]
Now we can check the exactness of this ODE, we'll denote the new ODE by $M^*$ and $N^*$,
\[\frac{\partial M^*}{\partial y} = 4x^3 + 6x^2y\]
\[\frac{\partial N^*}{\partial x} = 4x^3 + 6x^2y\]
Therefore, this ODE is exact, so we look for $F(x,y)$ such that
\[\frac{\partial F}{\partial x} = M^* \text{ and } \frac{\partial F}{\partial y} = N^*\]
The second equation is simpler so we'll start with this one, 
\[F(x,y) = \int x^4 + 2x^3ydy = x^4y + x^3y^2 + h(x)\]
Now we derive with respect to $x$ and use the first equation,
\[\frac{\partial F}{\partial x} = 4x^3 + 3x^2y^2 + h'(x)\]
Then, $M^* = x^4 + 4x^3y + 3x^2y^2$, which gives us $h'(x) = x^4$. Now 
\[h(x) = \int x^4dx = \frac{x^5}{5} + k\]
So we get the function
\[F(x,y) = x^4y + x^3y^2 + \frac{x^5}{5} + k\]
The general solution is given by setting $F(x,y)$ equal to a constant,
\[x^4y + x^3y^2 + \frac{x^5}{5} = C\]
Then using our initial value $y(1) = 1$,
\[1 + 1  + \frac{1}{5} = \frac{11}{5}\]
Thus, the solution to the IVP is 
\[x^4y + x^3y^2 + \frac{x^5}{5} = \frac{11}{5}\]
\textbf{Example.} Solve the following IVP with initial condition $y(0) = 1$,
\[(3xy - 2y^2\sin x + 4y)dx + (3x^2 + 8x + 6y \cos x)dy = 0\]
\textbf{Solution.} With $\sin$ and $\cos$ in our function, its certainly not of homogeneous coefficients, we check for exactness, 
\[\frac{\partial M}{\partial y} = 3x - 4y \sin x + 4\]
\[\frac{\partial N}{\partial x} = 6x + 8 - 6y \sin x\]
We see that these are not equal so this ODE is not exact, then we calculate the difference of the partial derivatives 
\[\frac{\partial M}{\partial y} - \frac{\partial N}{\partial x} = 3x + 2y\sin x x  -4\]
Then divide by $M$ to find a function of $y$, 
\[\frac{\frac{\partial M}{\partial y} - \frac{\partial N}{\partial x}}{M} = \frac{-3x + 2y\sin x - 4}{3xy - 2y^2\sin x + 4y} = \frac{-3x + 2y\sin x-4}{-y(-3x+2y\sin x - 4) = -\frac{1}{y}}\]
An integrating factor exists and is given by
\[\mu(y) = \exp\left(-\int -\frac{1}{y}\right) = e^{\ln y} = y\]
Multiplying the original ODE by $\mu(y) = y$,
\[(3xy^2 - 2y^3\sin x + 4y^2)dx + (3x^2y + 8xy + 6y^2\cos x)dy = 0 \]
Now checking exactness of our new ODE, 
\[\frac{\partial M^*}{\partial y} = 6xy - 6y^2 \sin x + 8y\]
\[\frac{\partial N^*}{\partial y} = 6xy - 6y^2 \sin x + 8y\]
This ODE is exact and we can now solve for $F(x,y)$ such that
\[\frac{\partial F}{\partial x} = M^* \text{ and } \frac{\partial F}{\partial y} = N^*\]
Both of the equations are similar in complexity so we'll start with the first one,
\[F(x,y) = \int 3xy^2 - 2y^3 \sin x + 4y dx = \frac{3x^2y^2}{2} = 2y^2\cos x + 4xy^2 + h(y)\]
Then using the second equation to solve for $y$, 
\[\frac{\partial F}{\partial y} = 3x^2y + 6y^2\cos x + 8xy + h'(y)\]
Then, $N^* = 3x^2y + 8xy + 6y^2\cos x$, which gives us $h'(y) = 0$. So we get $h(x) = k$, and the function
\[F(x,y) = \frac{3x^2y^2}{2} + 2y^3 \cos x + 4xy^2 + k\]
The general solution is 
\[\frac{3x^2y^2}{2} + 2y^3 \cos x + 4xy^2 = C\]
Using our inital condition $y(0) = 1$, we get
\[0 + 2 + 0 = C\]
The solution to the IVP is 
\[\frac{3x^2y^2}{2} + 2y^3 \cos x + 4xy^2 = 2\]
\textbf{Example.} Find the general solution of the following ODE,
\[(e^{x+y} ye^y)dx + (xe^y - 1)dy = 0\]
\textbf{Solution.} We have exponential functions so it is certainly not of homogeneous coefficients, we first check for exactness,
\[\frac{\partial M}{\partial y} = e^{x+y} + e^y + ye^y\]
\[\frac{\partial N}{\partial x} = e^y\]
We see that these are not equal so this ODE is not exact, then we calculate the difference of the partial derivatives
\[\frac{\partial M}{\partial y} - \frac{\partial N}{\partial x} = e^{x+y} + ye^y\]
This function is exactly $M$, so we will divide by $M$ to find a function of $y$,
\[\frac{\frac{\partial M}{\partial y} - \frac{\partial N}{\partial x}}{M} = 1 = g(y)\]
An integrating factor exists and is given by
\[\mu(y) = \exp\left(-\int g(y)dy\right) = \exp\left(-\int 1 dy\right) = e^{-y}\]
Multiplying the original ODE by $\mu(y) = e^{-y}$,
\[(e^x + y)dx + (x - e^{-y})dy = 0\]
Now checking exactness of our new ODE,
\[\frac{\partial M^*}{\partial y} = 1\]
\[\frac{\partial N^*}{\partial y} = 1\]
This ODE is exact and we can now solve for $F(x,y)$ such that
\[\frac{\partial F}{\partial x} = M^* \text{ and } \frac{\partial F}{\partial y} = N^*\]
Both of the equations are similar in complexity so we'll start with the first one,
\[F(x,y) = \int e^x + y dx = xe^x + yx + h(y)\]
Then using the second equation to solve for $h(y)$,
\[\frac{\partial F}{\partial y} = x + h'(x)\]
Then, $N^* = x - e^{-y}$, which gives us $h'(y) = -e^{-y}$. So we get
\[h(y) = \int h'(y)dy = -\int e^{-y} = e^y + k\]
Then the general solution is 
\[e^x + xy + e^{-y} = C\]
\section{Linear First-Order ODEs}
\begin{definition}
    A first order ODE that can be written under the form 
    \[y' + f(x)y = r(x)\]
    is called \textbf{linear}.
\end{definition}
\noindent
\textbf{Example.} 
\[xy' + e^xy = \frac{\sin x}{1 + x^2}\]
We can divide by $x$ to get
\[y' + \frac{e^x}{x}y = \frac{\sin x}{x(1+x^2)}\]
This is linear with $f(x) = \frac{e^x}{x}$ and $r(x) = \frac{\sin x}{x(1+x^2)}$
\subsection{Steps to Finding (Explicit) Solutions}
Given a linear first-order ODE in the formm $y' + f(x)y = r(x)$, we can find the solution by following these steps. Start by writing the ODE in differential form by replacing $y'$ with $\frac{dy}{dx}$,
    \begin{align*}
        &\frac{dy}{dx} + f(x)y = r(x)\\
        &\implies dy + f(x)ydx = r(x)dy\\
        &\implies (f(x)y - r(x))dx + 1dy = 0  
    \end{align*}
    This ODE is not exact since 
    \[\frac{\partial M}{\partial y} = f(x)\]
    \[\frac{\partial N}{\partial x} = 1\]
    So we find the integrating factor,
    \[\frac{\frac{\partial M}{\partial y} - \frac{\partial N}{\partial x}}{N} = f(x)\]
    and we get our integrating factor, 
    \[\mu(x) = \exp\left(\int f(x)dx\right)\]
    Note that 
    \[\mu'(x) = \exp\left(f(x)dx\right)\cdot \left(\int f(x)dx\right)' = \mu(x) f(x)\]
    Now we can multply the ODE by $\mu(x)$,
    \[(\mu(x)f(x)y - r(x)\mu(x))dx + \mu(x)dy = 0\]
    Now we can check for exactness, of our new ODE, 
    \[\frac{\partial M^*}{\partial y} =\mu(x)f(x)\]
    \[\frac{\partial n^*}{\partial x} =\mu(x)f(x)\]
    Then we look for our function $F(x,y)$ such that
    \[\frac{\partial F}{\partial x} = M^* \text{ and } \frac{\partial F}{\partial y} = N^*\]
    \[F(x,y) = \int \mu(x) dy = \mu(x)y + h(x)\]
    Then we use the second equation to solve for $h(x)$,
    \[\frac{\partial F}{\partial y} = \mu'(x)y + h'(x) = \mu(x)f(x) + h'(x)\]
    Then, $M^* = \mu(x)f(x)y - \mu(x)r(x)$, which gives us that $h'(x) = - \mu(x)r(x)$. So we get
    \[h(x) = \int \mu(x)r(x)dx  + k \]
    Then we get the function 
    \[F(x,y) = \mu(x)y - \int \mu(x)r(x)dx + k\]
    and our (implicit) general solution is 
    \[\mu(x)y - \int\mu(x)r(x)dx = C\]
    We can find an explicit solution by solving for $y$, 
    \begin{align*}
        &\mu(x)y - \int\mu(x)r(x)dx = C\\
        \implies &\mu(x)y = \int\mu(x)r(x)dx + C\\
        \implies &y = \frac{\int\mu(x)r(x)dx + C}{\mu(x)}\\
        \implies &y = \frac{\int\mu(x)r(x)dx + C}{\exp\left(\int f(x)dx\right)}\\
        \implies &y = \left(\int \exp\left(\int f(x)dx\right)r(x)dx + C\right)\exp\left(\int f(x)\right)^{-1}\\
        \implies &y = \left(\int \exp\left(\int f(x)dx\right)r(x)dx + C\right)\exp\left(-\int f(x)\right)
    \end{align*}
\textbf{Example.} Solve the IVP 
\[y' - 2xy = x \ \ y(0) = \frac{1}{2}\]
\textbf{Solution.} Clearly this is a linear first order ODE witih $f(x) = -2x$, and $r(x) = x$. Then we can apply the formula to get the general solution
\begin{align*}
    y &= \left(\int \exp\left(\int f(x)dx\right)r(x)dx +C \right)\exp\left(-\int f(x)dx\right)  \\
    &=  \left(\int \exp\left(\int -x^2dx\right)xdx + C\right)\exp\left(-\int -2xdx\right)\\
    &= \left(\int x e^{-x^2}xdx + C\right)e^{x^2}\\
\end{align*}
We can solve the integral by substitution, set $u \coloneqq -x^2$, $du = -2xdx$, $dx = \frac{du}{-2x}$, then we get 
\begin{align*}
    \int xe^{-x^2}dx &= \int xe^u \frac{du}{-2x}\\
    & = -\frac{1}{2}\int e^udu\\
    &= -\frac{1}{2} e^{-x^2}
\end{align*}
Then we get the explicit general solution
\[y =  \left(-\frac{1}{2}e^{-x^2} + C\right)  e^{x^2} = -\frac{1}{2} + Ce^{x^2}\]
We can solve for $C$ using the initial condition $y(0) = \frac{1}{2}$,
\[\frac{1}{2} = -\frac{1}{2} + C \implies C = 1\]
So our explicit solution is 
\[y = e^{x^2} -\frac{1}{2}\] 
\textbf{Example.} Solve the IVP
\[y' - 4y = x; \ y(0) = \frac{15}{16}\]
\textbf{Solution.} This is a linear first-order ODE with $f(x) = -4$, and $r(x) = x$. Then we can apply the formula to get the general solution
\begin{align*}
    y &= \left(\int \exp\left(\int f(x)dx\right)r(x)dx + C\right)\exp\left(-\int f(x)\right)\\
    &= \left(\int \exp\left(\int -4dx\right)xdx + C\right)\exp\left(-\int -4dx\right)\\
    &= \left(\int e^{-4x}xdx + C\right)e^{4x}\\
\end{align*}
Now we can solve the integral by parts, set $u = x$, $v' = e^{4x}$, $u' = 1$, $v = \int e^{-4x}dx = -\frac{1}{4}e^{-4x}$, then we get 
\begin{align*}
    \int e^{-4x}xdx &= uv - \int u'vdx\\
    &= -\frac{1}{4}xe^{-4x} - \int -\frac{1}{4}e^{-4x}dx\\
    &= -\frac{1}{4}xe^{-4x} - \frac{1}{16}e^{-4x}
\end{align*}
Then, we get the general solution 
\[y = \left(-\frac{1}{4}xe^{-4x} - \frac{1}{16}e^{-4x} + C\right)e^{4x} = -\frac{1}{4}x - \frac{1}{16} + Ce^{4x}\]
Using our intial condition $y(0) = \frac{15}{16}$, we can solve for $C$,
\[\frac{15}{16} = -\frac{1}{16} + C \implies C = 1 \]
Therefore, our explicit solution is
\[y = -\frac{1}{4}x - \frac{1}{16} + e^{4x}\]
\textbf{Example.} Solve the IVP
\[(1 + \cos x)y' - (\sin x)y = 2x; \ y(0) = \frac{1}{2}\]
\textbf{Solution.} We have to make the coefficient of $y'$ to be 1, so we divide both sides by $1 + \cos x$ to get
\[y' - \frac{\sin x}{1 + \cos x} y = \frac{2x}{1 + \cos x}\]
Now we can use our formula for linear ODE's,
\begin{align*}
    y &= \left(\int \exp\left(\int f(x)dx\right)r(x)dx + C\right)\exp\left(-\int f(x)dx\right)\\
    &= \left(\int \exp\left(\int - \frac{\sin x}{1 + \cos x}dx\right)\frac{2x}{1 + \cos x}dx + C\right)\exp\left(-\int- \frac{\sin x}{1 + \cos x}\right)\\
\end{align*}
Now computing the integral of $f(x)$, set $u = 1 + \cos x$, $du = -\sin x dx$
\begin{align*}
    \int \frac{-\sin x}{1 + \cos x} dx &= \int \frac{-\sin x}{u} \frac{du}{- \sin x}\\
    &= \int \frac{1}{u}du\\
    &= \ln|u| \\
    &= \ln (1 + \cos x)
\end{align*} 
Therefore, we have 
\begin{align*}
    y &= \left(\int \exp\left(\int - \frac{\sin x}{1 + \cos x}dx\right)\frac{2x}{1 + \cos x}dx + C\right)\exp\left(-\int- \frac{\sin x}{1 + \cos x}\right)\\
    &= \left(\int \exp(\ln(1 + \cos x))\frac{2x}{1 + \cos x}dx + C\right)\exp\left(-\ln (1 + \cos x)\right)\\
    &= \left(\int (1 + \cos x)\frac{2x}{1 + \cos x}dx + C\right)(1 + \cos x)^{-1}\\
    &= \frac{\left(\int 2xdx + C\right)}{1 + \cos x}\\
    &= \frac{x^2 + C}{1 + \cos x}\\
\end{align*}
Using our inital condition $y(0) = \frac{1}{2}$, we get 
\[\frac{1}{2} = \frac{C}{2} \implies C = 1\]
Therefore, our explicit solution is
\[y = \frac{x^2 + 1}{1 + \cos x}\]
\section{First-Order Bernoulli ODE's}
\begin{definition}
    A first-order ODE is called of \emph{Bernouilli} type if it can be written in the form
    \[y' + f(x)y = r(x)y^a\]
    for some $a \in \real$.  
\end{definition}
\subsection{Steps to Solving Bernoulli type ODE's}
\begin{enumerate}
    \item Let $u = y^{1-a}$
    \item Compute $u'$:
    \[u' = (1-a)y^{-1}y'\]
    \item Isolate $y'$ from the original ODE and substitute into $u'$
    \item The resulting ODE is linear that we solve for $u$.  
\end{enumerate}
\textbf{Example.} Solve the IVP
\[y' + \frac{4}{x}y = -x^2y^2; \ x > 0, \ y(1) = \frac{1}{3}\]
\textbf{Solution.} This is a first order Bernouilli ODE, with $f(x) = \frac{4}{x}$, $r(x)  = -x^2$, and $a = 2$. Let $u = y^{1 - a} = y^{-1}$, then 
\begin{align*}
    u' &= -y^{-2}y' = -y^{-2}\left(-\frac{4}{x}y - x^3y^2\right)\\
    &= \frac{4}{x}y^{-1} + x^3\\
    &= \frac{4}{x}u + x^3\\
    &= u' - \frac{4}{x}u = x^3
\end{align*}
Now this is a linear first-order ODE in the function $u$ with $f(x) = -\frac{4}{x}$ and $r(x) = x^3$. Then, the general solution is
\begin{align*}
    u &= \left(\int \exp\left(\int f(x)dx\right)r(x)dx + C\right)\exp\left(-\int f(x) \right)\\
    &= \left(\int \exp\left(\int -\frac{4}{x}\right)x^3dx + C\right)\exp\left(-\int -\frac{4}{x}dx\right)\\
    &= \left(\int \exp(-4\ln x)x^3 dx+ C \right)\exp(4\ln x)\\
    &= \left(\int x^{-4}x^3dx + C\right)x^4\\
    &= \left(\int x^{-1}dx + C\right)x^4\\
    &= \left(\ln x + C\right)x^4\\
\end{align*}
Now, we know $u = y^{-1}$, so
\[y = \frac{1}{x^4(\ln x + C)}\]
Using our intial condition $y(1) = \frac{1}{3}$, 
\[\frac{1}{3} = \frac{1}{1 \cdot (\ln 1 + C)} = \frac{1}{C} \implies C = 3\]
Therefore, the explicit solution to our IVP is 
\[y = \frac{1}{x^4(\ln x + 3)}\]
\textbf{Example.} Solve the IVP
\[y' + \frac{2}{x}y = 2 \sqrt{y}; \ x > 0 \ y(1) = 1\]
\textbf{Solution.} This is a first order Bernouilli ODE, with $f(x) = \frac{2}{x}$, $r(x) = 2$, and $a = \frac{1}{2}$. Let $u = y^{1 - a} = y^{\frac{1}{2}} = \sqrt{y}$. Then, 
\begin{align*}
    u' &= \frac{1}{2}y^{-\frac{1}{2}}y'\\
    &= \frac{1}{2}y^{-\frac{1}{2}}\left(-\frac{2}{x}y + 2y^{\frac{1}{2}}\right)\\
    &= -\frac{1}{x}y^{\frac{1}{2}} + 1
\end{align*} 
Then we get the linear first order ODE in $u$
\[u' + \frac{1}{x}u = 1\]
Our general solution for $u$ is 
\begin{align*}
    u &= \left(\int \exp\left(\int f(x)dx\right)r(x)dx + C\right)\exp\left(-\int f(x) dx\right)\\
    &= \left(\int \exp\left(\int\frac{1}{x}dx\right)1dx + C\right)\exp\left(-\int \frac{1}{x}dx\right)\\
    &= \left(\int \exp(\ln x)dx + C\right)\exp(-\ln x)\\
    &= \left(\int xdx + C\right)\exp(-2\ln x)\\
    &= \left(\frac{1}{2}x^2 + C\right)x^{-1}\\
    &= \frac{x}{2} + \frac{C}{x}
\end{align*}
Then using your equation for $u$ in terms of $y$, 
\[u = \sqrt{y} \implies y =\left(\frac{x}{2}+  \frac{C}{x}\right)^2\]
Using our initial condition $y(1) = 1$, 
\[1 = \left(\frac{1}{2} + C\right)^2 \implies C = \frac{1}{2}\]
Therefore, the explicit solution to our IVP is
\[y = \left(\frac{x^2 + 1}{2x}\right)^2\]
\chapter{Second Order Linear Homogeneous ODEs}
\begin{definition}[Linear Independence]
    We say that the functions $y_1, y_2, \ldots, y_n$ are linearly indepedent on an interval $I$ if 
    \[c_1y_1 + c_2y_2 + \cdots + c_ny_n = 0 \implies c_1 = c_2 = \cdots = c_n = 0\]
\end{definition}
\begin{theorem}
    Two functions $y_1,y_2$ are linearly indepedent if and only if $\frac{y_1}{y_2}$ does not equal a constant. 
\end{theorem}
\section{Wronskian}
\begin{definition}[Wronskian]
    Let $y_1, y_2, \ldots, y_n$ be $n$ functions such that the first $n-1$ derivatives of each function exists, and are continuous on $I$. The \emph{Wronskian} of $y_1, y_2, \ldots y_n$ at a point $x \in I$ is the determinant
    \[W[y_1,y_2,\ldots,y_n](x) = \det\begin{bmatrix}
        y_1(x) & y_2(x) & \cdots & y_n(x)\\
        y'_1(x) & y'_2(x) & \cdots & y'_n(x)\\
        \vdots & \vdots & \ddots & \vdots\\
        y_1^{(n-1)}(x) & y_2^{(n-1)}(x) & \cdots & y_n^{(n-1)}(x)
    \end{bmatrix}\]
\end{definition}
\noindent
\textbf{Example.} Take $y_1 = 1$, $y_2 = \sin x$, $y_3 = \cos x$. Then, the Wronskian is 
\begin{align*}
    W[y_1, y_2,y_3](x) &= \det\begin{bmatrix}
        1 & \sin x & \cos x\\
        0 & \cos x & -\sin x\\
        0 & -\sin x & -\cos x
    \end{bmatrix} \\
    &= 1 \det \begin{bmatrix}
        \cos x & -\sin x\\
        -\sin x & -\cos x
    \end{bmatrix} - 0 + 0\\
    &= \cos^2x - \sin^2x\\
    &= -(\cos^2x + \sin^2x) = -1        
\end{align*}

\begin{theorem}
    Let $y_1,y_2,\ldots,y_n$ be continuous functions with continuous first $n-1$ derivatives on an interval $I$. If $W[y_1,y_2,\ldots,y_n](x) \neq 0$ for some $x \in I$, then $y_1,y_2,\ldots,y_n$ are linearly independent on $I$.
\end{theorem}
\textbf{Example.} $y_1 = x$, $y_2 = e^x$, $y_3 = e^{2x}$. Prove that $\{y_1,y_2,y_3\}$ are linearly independent on $\mathbb{R}$.\\[3ex]
\textbf{Solution.} We'll use the Wronskian to show that these functions are linearly independent.
\begin{align*}
    W[y_1,y_2,y_3](x) &=  \det\begin{bmatrix}
        x & e^x & e^{2x}\\
        1 & e^x & 2e^{2x}\\
        0 & e^x & 4e^{2x}
    \end{bmatrix}\\
    &=\det \begin{bmatrix}
        0 & -xe^x + e^x & -2xe^{2x}+e^{2x}\\
        1 & e^{x} & 2e^{2x}\\
        0 & e^x & 4e^{2x}
    \end{bmatrix}\tag{$-xR_2 + R_1 \to R_1$}\\
    &= -\det\begin{bmatrix}
        -xe^x + e^x & -2e^{2x} + e^{2x}\\
        e^x & 4e^{2x}
    \end{bmatrix}\\
    &= -e^{3x}(3-2x)
\end{align*}

\begin{definition}
    An ODE of order $n$ is called \emph{linear} if it is of the form 
    \[a_n(x) y^{(n)} + a_{n-1}(x)y^{(n-1)} + \cdots + a_1(x)y' + a_0(x)y = r(x)\]
    If $r(x) = 0$, then the ODE is called \emph{homogeneous}. If $r(x) \neq 0$, then the ODE is called \emph{non-homogeneous}.
\end{definition}

\textbf{Example.} 
\[2xy''' + e^xy'' - y = \frac{1}{1+x^2}\]
This is a linear non-homogeneous ODE.
\begin{theorem}
    The set of all solutions to a homogeneous linear ODE of order $n$
    \[a_n(x) y^{(n)} + a_{n-1}(x)y^{(n-1)} + \cdots + a_1(x)y' + a_0(x)y = 0\]
    is a vector space of dimension $n$.
\end{theorem}
\subsection{Steps to finding a general solution of a homogeneous linear ODE}
The theorem suggets the following steps to finding the general solution to a homogeneous linear ODE of order $n$:
\begin{enumerate}
    \item Find $n$ linearly indepdent solutions $y_1,y_2,\ldots,y_n$ to the ODE.
    \item $\{y_1, y_2,\ldots,y_n\}$ form a basis of solutions to the ODE.
    \item The general solution to the ODE is a linear combination of the basis functions 
    \[y = c_1 y_1 + c_2y_2 + \cdots + c_ny_n\]
\end{enumerate}
In this chapter, we want to find the general solution to second order linear homogeneous ODEs
\[a_2(x)y'' + a_1(x)y' + a_0(x)y = 0\]
\section{Second-Order Linear Homogeneous ODEs with Constant Coefficients}
For this group of ODE's, we consider the case when $a_2(x), a_1(x), a_0(x)$ are constants. In this case, we can write the ODE as
\[ay'' + by' + cy = 0\]
We'll look at an example to illustrate the process of finding solutions for this type of ODE.\\[3ex]
\textbf{Example.} Consider the ODE 
\[y'' - y = 0\]
\textbf{Solution.} We know that we need 2 linearly independent solutions to the ODE. We have that $y'' - y = 0 \implies y'' = y$. Two functions that satisfy this equation are $y_{1} = e^x$, and $y_2 = e^{-x}$. We can check that these are linearly indepedent by checking their ratio 
\[\frac{e^x}{e^{-x}} = e^{2x}\]
Therefore, these solutions form a basis for the solution space for $y'' - y = 0$. Then the general solution of the ODE is a linear combination of the the basis functions 
\[y = c_1e^x + c_2e^{-x}\]
The above example suggetss that we look for \emph{exponential} solutions in the case of constant coefficients
\[y'' + ay' + by = 0\]
In general, we have
\[y = e^{\lambda x}\]
for some constant $\lambda$. Then we can differentiate $y$ to get 
\[y' = \lambda e^{\lambda x}, \ y'' = \lambda^2 e^{\lambda x}\]
Plugging our equations into our ODE, we get
\[\lambda^2 e^{\lambda x} + a\lambda e^{\lambda x} + b e^{\lambda x} = 0 \implies \lambda^2 + a \lambda + b = 0\]
This is our \emph{characteristic equation}. There are 3 possible cases for the roots of the (quadratic) characteristic equation,
\begin{itemize}
    \item \textbf{Case 1:} We have two distinct real roots $\lambda_1, \lambda_2$, then $y_1 = e^{\lambda x}$, $y_2 = e^{\lambda_2x}$. Then our general solution is 
    \[y = c_1 e^{\lambda_1 x} + c_2 e^{\lambda_2 x}\]
    \item \textbf{Case 2:} We have one real root $\lambda$ with multiplicity 2. In this case, $y_1 = e^{\lambda x}$ is one solution, and our second solution is $y_2 = xe^{\lambda x}$. Then our general solution is
    \[y = c_1e^{\lambda x} + c_2 x e^{\lambda x}\]
    \item \textbf{Case 3:} We have two complex conjugate roots $\lambda_1 = \alpha + i\beta$ and $\lambda_2 = \alpha - i\beta$. In this case, we can show 
    \[y_1 = e^{\alpha x}\cos(\beta x), \ y_2 = e^{\alpha x}\sin(\beta x)\]
    is a bais of solutions, so 
    \[y = c_a e^{\alpha x} \cos (\beta x)  + c_2 e^{\alpha x}\sin(\beta x)\]
\end{itemize}

\subsection{Steps to Solving Second Order Linear Homogeneous ODEs with Constant Coefficients}
To summarize, the steps to solving the ODE 
\[y'' + ay' + by = 0\]
are as follows. 
\begin{enumerate}
    \item Write the characteristic equation $\lambda^2 + a\lambda +b = 0$.
    \item Find the roots of the characteristic equation.
    \item If the characteristic equation has two distintc real roots $\lambda_1, \lambda_2$, then $\{e^{\lambda_1x}, e^{\lambda_2x}\}$ is a basis of solutions and the general solution is 
    \[y = c_1e^{\lambda_1x} + c_2e^{\lambda_2x} \]
    \item If the characteristic equation has a double real root $\lambda_1 = \lambda_2 = \lambda$. Then $\{y_1 = e^{\lambda x}, xe^{\lambda x}\}$ is a basis of solutions and the general solution is
    \[y = c_1e^{\lambda x} + c_2xe^{\lambda x}\]
    \item If the characteristic equation has 2 complex conjugate roots 
    \[\lambda_1 = \alpha + i\beta, \ \lambda_2 = \alpha - i\beta\]
    Then $\{y_1 = e^{\alpha x}\cos(\beta x), y_2 e^{\lambda x}\sin(\beta x)\}$ is a basis of solutions. The general solution in this case is 
    \[y = c_1e^{\lambda x}\cos(\beta x)+ c_2e^{\lambda x}\sin(\beta x)\]
\end{enumerate}
\textbf{Example.} Solve the IVP
\[y'' - 5y' + 6y = 0, \ y(0) = -1, \ y'(0) = 2\]
\textbf{Solution.} This is a second order homogeneous ODE with constant coefficients. The characteristic equation is 
\[\lambda^2 - 5\lambda + 6 = 0\]
Now we can solve for the roots of the characteristic equation,
\[\lambda^2 - 5\lambda + 6 = 0 \implies (\lambda -2)(\lambda - 3) = 0\]
We have 2 distinct real roots $\lambda_1 = 2$, $\lambda_2 = 3$. So our basis of solutions is 
\[\{y_1 = e^{2x}, y_2 = e^{3x}\}\]
The general solution is 
\[y = c_1e^{2x} + c_2e^{3x}\]
Now we can use our inital conditions 
\[y' = 2c_1e^{2x}+3c_2e^{3x}\]
\[y(0) = -1 \implies -1 = c_1 + c_2\]
\[y'(0) = 2 \implies 2 = 2c_1 + 3c_2\]
We can solve this system of equations 
\[2c_1 + 3c_2 - 2(c_1 + c_2) = c_2 \implies c_2 = 2 - (-2) = 4\]
\[c_1 + 4 = -1 \implies c_1 = -5\]
Therefore our solution is
\[y = -5e^{2x} + 4e^{3x}\]
\noindent
\textbf{Example.} Solve the IVP 
\[y'' + 2y' + 2y = 0, \ y(0) = 1, \ y'(0) = 0\]
\textbf{Solution.} This is a second order homogeneous ODE with constant coefficients. The characteristic equation is
\[\lambda^2 + 2\lambda + 2 = 0\]
We can solve for the roots of the characteristic equation,
\begin{align*}
    \lambda &= \frac{-2 \pm \sqrt{2^2 - 4(1)(2)}}{2(1)}\\
    &= \frac{-2 \pm \sqrt{4 - 8}}{2}\\
    &= \frac{-2 \pm \sqrt{-4}}{2}\\
    &= \frac{-2 \pm 2\sqrt{-1}}{2}\\
    &= -1 \pm i
\end{align*}
We have 2 complex conjugate rootts, $\lambda_1 = -1 + i$, $\lambda_2 = -1 - i$. So our basis of solutions is
\[\{y_1 = e^{-x}\cos(x), \ y_2 = e^{-x}\sin(x)\}\]
The general solution is 
\[y = c_1e^{-x}\cos(x) + y_2 + c_2e^{-x}\sin(x)\]
Using our initial conditions,
\[y' = -c_1e^{-x}\cos(x) - c_1e^{-x}\sin(x) - c_2e^{-x}\sin(x) + c_2e^{-x}\cos(x)\]
\[y(0) = 1 \implies 1 = c_1\]
\[y'(0) = 0 \implies 0 = -c_1 + c_2 \implies c_2 = c_1 = 1\]
Therefore our unique solution is 
\[y = e^{-x}\cos(x) + y_2 + e^{-x}\sin(x)\]
\textbf{Example.} Solve the IVP
\[y'' + 4y' + 4y = 0, \ y(0) = y'(0) = 2\]
\textbf{Solution.} This is a second order homogeneous ODE with constant coefficients. The characteristic equation is
\[\lambda^2 + 4\lambda + 4 = 0\]
We can solve for the roots of the characteristic equation,
\[\lambda^2 + 4\lambda + 4 = 0 \implies (\lambda + 2)(\lambda +2) = 0\]
Therefore our roots are $\lambda_1 = \lambda_2 = \lambda = -2$. So our basis of solutions is
\[\{y_1 = e^{-2x}, y_2 = xe^{-2x}\}\]
The general solution is
\[y = c_1e^{-2x} + c_2xe^{-2x}\]
Now we can use our initial condition 
\[y' = -2c_1e^{-2x} + c_2e^{-2x}-2c_2xe^{-2x}\]
\[y(0) = 2 \implies 2 = c_1\]\
\[y'(0) = 2 \implies 2 = -2c_1 + c2 \implies c_2 = 6 \]
Therefore our unique solution is
\[y = 2e^{-2x} + 6xe^{-2x}\]
\section{Second-order Euler-Cauchy Equations}
\begin{definition}
    A second-order ODE is \emph{Euler-Cauchy} if it has the following form
    \[x^2y'' + axy' + by = 0\]
    With $x >0$, $a,b \in \real$.
\end{definition}
Unlike the case of constant coefficients where we looked for exponential functions, we look for solutions of the form
\[y = x^m, \ y' = mx^{m-1}, \ y'' = m(m-1)x^{m-2} \]
Substituting these back into our ODE, we get
\begin{align*}
    0 &= x_2m(m-1)x^{m-2} + axmx^{m-1} + bx^m\\
    0 &= m(m-1)x^m + amx^m + bx^m \\
    0 &= m^2 - m + am + b \\
    0 &= m^2 + (a-1)m + b 
\end{align*}
This is our characteristic equation for Euler-Cauchy equations. We again have 3 cases for the roots of our characteristic equation,
\begin{itemize}
    \item \textbf{Case 1.} The characteristic equation has 2 distinct real roots $m_1, m_2$. In this case our basis of solutions is
    \[y_1 = x^{m_1}, \ y_2 = x^{m_2}\]
    The general solution is
    \[y = c_1x^{m_1} + c_2x^{m_2}\]
    \item \textbf{Case 2.} The characteristic equation has a double real root $m_1 = m_2 = m$. In this case our basis of solutions is
    \[y_1 = x^m, \ y_2 = x^m\ln x\]
    and the general solution is
    \[y = c_1x^m +_2x^m\ln x\]

    \item \textbf{Case 3.} The characteristic equation has complex conjugate roots $m_1 = \alpha + i\beta$, $m_2 = \alpha - i\beta$. In this case our basis of solutions is
    \[y_1 = x^\alpha\cos(\beta\ln x), \ y_2 = x^\alpha \sin(\beta\ln x)\]
\end{itemize}
\noindent
\textbf{Example.} Solve the IVP
\[x^2y'' - 3xy' + 4y = 0, \ y(1) = 2, \ y'(1) = 1\]
\textbf{Solution.} This is a second order Euler-Cauchy ODE. The characteristic equation is
\[m^2 - 4m + 4 = 0 \implies (m-2)^2 = 0\] 
We have a double real root $m_1 = m_2 = 2$. So our basis of solutions is
\[\{y_1 = x^2, y_2 = x^2\ln x\}\]
Our general solution is 
\[y = c_1x^2 + c_2x^2\ln x\]
Now we can use our initial conditions
\[y' = 2c_1x + 2c_2x\ln x + 2c_2x\]
\[y(1) = 2 \implies 2 = c_1\]
\[y'(1) = 1 \implies 1 = 2c_1 + 2c_2 \implies c_2 = 1 - 2c_1 = -3\]
The unique solution to our IVP is
\[y = 2x^2 - 3x^2\ln x\]
\textbf{Example.} Find the general solution to the ODE 
\[x_2y'' - 3xy' + 5y = 0\]
\textbf{Solution.} This is a second order Euler-Cauchy ODE. The characteristic equation is
\[m^2 - 4m + 5 = 0\]
The roots are 
\begin{align*}
    m &= \frac{4 \pm \sqrt{16 - 4(5)(1)}}{2(1)}\\
    &= \frac{4 \pm \sqrt{-4}}{2}\\
    &= \frac{4 \pm 2\sqrt{-1}}{2}\\
    &= 2 \pm i 
\end{align*}
So we have 2 complex conjugate roots $m_1 = 2 + i$, $m_2 = 2 - i$. So our basis of solutions is
\[y_1 = x^2\cos(\ln x), \ y_2 = x^2\sin(\ln x)\]
The general solution is 
\[y = c_1x^2\cos(\ln x) + c_2x^2\sin(\ln x)\]
\textbf{Example.} Solve the IVP
\[x_2y'' + 5xy' + 4y = 0, \ y(1) = 0, \ y'(1) = 2\]
\textbf{Solution.} This is a second order Euler-Cauchy ODE. The characteristic equation is
\[m^2 + 4m + 4 = 0 \implies (x+2)^2\]
We have a double real root $m_1 = m_2 = -2$. So our basis of solutions is
\[\{y_1 = x^{-2}, y_2 = x^{-2}\ln x\}\]
So our general solution is 
\[y = c_1x^{-2} + c_2x^{-2}\ln x\]
We can use our initial conditions
\[y' = -2c_1x^{-3} + -2c_2x^{-3}\ln x + c_2x^{-3}\]
\[y(1) = 0 \implies 0 = c_1\]
\[y'(1) = 2 \implies 2 = -2c_1 + c_2 \implies c_2 = 2\]
The unique solution to the IVP is 
\[y = 2x^{-2}\ln x \]
\textbf{Example.} Find the general solution of the ODE 
\[x^2y'' - 2xy' + 2y = 0\]
\textbf{Solution.} This is a second order Euler-Cauchy ODE. The characteristic equation is
\[m^2 - 3y + 2 = 0 \implies (m- 2)(m-1) = 0\]
The roots are $m_1 = 2$, $m_2 = 1$. So our basis of solutions is
\[\{y_1 = x^2, y_2 = x^1\}\]
The general solution is
\[y = c_1x^2 + c_2x\]
\section{Higher-order Linear ODEs with Constant Coefficients.}
We turn our attention to higher order linear homogeneous ODE's \
\[a_m(x)y^{(m)} + \cdots + a_1(x)y' + a_0(x) = 0\]
\begin{theorem}
    If $a_m(x), \ldots, a_0(x)$ are continuous, then the set of solutions of the ODE is a vector space of dimension $m$.
\end{theorem}
Like in the case of order 2, we consider two families of linear homogeneous ODE's, those with constant coefficients 
\[y^{(n)} + a_{n-1}y^{(n-1)} + \cdots + a_2y'' + a_1 y' + a_0y = 0\]
For constants $a_{n-1}, \ldots, a_0$.  Similarly to ODE's of order 2, we look for exponential solutions of the form $y = e^{\lambda x}$. Then we differentiate $y$ 
\[y' = \lambda e^{\lambda x}, \ y'' = \lambda^2 e^{\lambda x}, y''' =  \lambda^3e^{\lambda x},\ldots, y^{(n)} = \lambda^n e^{\lambda x}\]
Substituting these back into our ODE 
\[\lambda^ne^{\lambda x} + a_{n-1}\lambda^{n-1}e^{\lambda x } + \cdots + a_1\lambda e^{\lambda x} + a_1e^{\lambda x } = 0\]
Dividing by $e^{\lambda x}$, we get
\[\lambda^n + a_{n-1}\lambda^{n-1} + \cdots + a_1\lambda + a_0 = 0\]
This is our characteristic equation for any order linear homogeneous ODE with constant coefficients. The general solution depends on the roots of the characteristic equation, 
\begin{itemize}
    \item If $\lambda$ is a root of multiplicity $k$ of our characteristic equation, then it contributes the following equations to our basis of solutions, 
    \[y_1 = e^{\lambda x}, \ y_2 = xe^{\lambda x}, \ y_3 = x^2e^{\lambda x}, \ldots, y_k = x^{k-1}e^{\lambda x}\]
    Each root will contribute $k$ equations to our basis of solutions in this manner.
    \item If $\alpha + i\beta$ is a pair of complex conjugate roots, then it contributes the following 2 equations to our basis of solutions,
    \[y_1 = e^{\alpha x}\cos(\beta x), \ y_2 = e^{\alpha x}\sin(\beta x)\]
\end{itemize} 
\textbf{Example.} Find the general solution of the following ODE 
\[y^{(5)} - 2y''' + 2y'' - 3y' + 2y = 0\]
\textbf{Solution.} This is a linear homogeneous ODE with constant coefficients of order 5, our characteristic equation is 
\[\lambda^5 - 2 \lambda^3 + 2\lambda^2 - 3\lambda + 2 = 0\]
We start by finding 1 root, we'll guess and check the roots. We'll try $\lambda = 1$,
\[1 - 2 + 2 - 3  + 2 = 0 \]
Therefore $\lambda -1$ is a factor of our characteristic equation. We can use polynomial long division to find the other factors, and we get 
\[(\lambda - 1)^2(\lambda + 2)(\lambda^2 + 1) = 0\]
So our roots are $\lambda_1 = 1$, $\lambda_2 = -2$, $\lambda_3 = i$, and $\lambda_4 = -i$. $\lambda_1$ is a real root with multiplicity 2, so it will contribute the following 2 equations to our basis of solutions 
\[y_1 = e^x, \ y_2 = xe^{x}\]
$lambda_2 = -2$ is a real root with multiplicity 1, so it will contribute the following equation to our basis of solutions
\[y_3 = e^{-2x}\]
$\lambda_3 = i$ and $\lambda_4 = -i$ are to complex conjugate roots, they contribute 2 equations to our basis of solutions with $\alpha = 0$ and $\beta = 1$
\[y_4 = \cos(x), \ y_5 = \sin x\]
Therefore, our basis of solutions is
\[\{e^{x}, xe^{x}, e^{-2x}, \cos x, \sin x\}\]
The general solution is
\[y = c_1e^{x} + c_2xe^{x} + c_3e^{-2x} + c_4\cos x + c_5 \sin x\]
\textbf{Example.} Solve the following IVP, 
\[y''' - 5y'' + 6y' = 0, \ y(0) = -1, \ y'(0) = 2, \ y''(0) = 0\]
\textbf{Solution.} This is a linear homogeneous ODE with constant coefficients of order 3. Our characteristic equation is
\[\lambda^3 - 5\lambda^2 + 6\lambda = 0 \implies \lambda(\lambda^2 - 5\lambda + 6) = 0 \implies \lambda(\lambda - 3)(\lambda -2)\]
We have 3 roots, $\lambda_1 = 0$, $\lambda_2 = 3$, and $\lambda_3 = 2$. So we have the following equations 
\[y_1 = 1, \ y_2 = e^{2x}, \ y_3 = e^{3x}\]
So our basis of solutions is
\[y = c_1 + c_2e^{2x} + c_3e^{3x}\]
Using our initial conditions, 
\[y' = 3c_2e^{3x} + 2c_3e^{2x}\]
\[y'' = 9c_2e^{3x} + 4c_3e^{2x}\]
\[y(0) = -1 \implies -1 = c_1 + c_2+ c_3\]
\[y'(0) = 2 \implies 2 = 2c_2 + 3c_3\]
\[y''(0) = 0 \implies 0 = 4c_2 + 9c_3\]
We can use Guass Jordan elimination to solve this system of equations,
\begin{align*}
    \left[
        \begin{array}{ccc|c}
            1 & 1 & 1 & -1\\
            0 & 2 & 3 & 2\\
            0 & 4 & 9 & 0
        \end{array}  
    \right] &\sim \left[
        \begin{array}{ccc|c}
            1 & 1 & 1 & -1\\
            0 & 1 & \frac{3}{2} & 1\\
            0 & 4 & 9 & 0
        \end{array}
        \right] \sim \left[
            \begin{array}{ccc|c}
                1 & 1 & 1 & -1\\
                0 & 1 & \frac{3}{2} & 1\\
                0 & 0 & 3 & -4\\
            \end{array}
        \right]\\
        &\sim \left[
            \begin{array}{ccc|c}
                1 & 1 & 0 & \frac{1}{3}\\
                0 & 1 & 0 & 3\\
                0 & 0 & 1 & -\frac{4}{3}
            \end{array}
        \right] \sim \left[
            \begin{array}{ccc|c}
                1 & 0 & 0 & -\frac{8}{3}\\
                0 & 1 & 0 & 3\\
                0 & 0 & 1 & -\frac{4}{3}
            \end{array}
        \right]
    \end{align*}
    Therefore, we get $c_1 = -8/3$, $c_2 = 3$, and $c_3 = -4/3$. Thus our unique solution is 
    \[y = -\frac{8}{3} + 3e^{2x} - \frac{4}{3}e^{3x}\]
    \textbf{Example.} Find the general solution for the following ODE 
    \[y^{(4)} - y = 0\]
    \textbf{Solution.} This is a homogeneous linear ODE with constant coefficients of order 4. Our characteristic equation is
    \[\lambda^4 - 1 = 0 \implies (\lambda^2 - 1)(\lambda^2 + 1) = 0 \implies (\lambda + 1)(\lambda - 1)(\lambda^2 + 1) = 0\]
    Solving $\lambda^2 + 1 = 0$ we get $\lambda = \pm i$. So our roots are $\lambda_1 = -1$, $\lambda_2 = 1$, $\lambda_3 = i$, and $\lambda_4 = -i$. So our basis of solutions is
    \[\{y_1 = e^{-x}, y_2 = e^{x}, y_3 = \cos x, y_4 = \sin x\}\]
    Thus our general solution is 
    \[y = c_1e^{-x} + c_2e^{x} +c_3\cos x + c_4\sin x\]
    \textbf{Example.} Find the general solution to the ODE 
    \[y^{(5)} - 6y^{(4)} + 13y''' - 14y'' + 12y' - 8y' = 0\]
    \textbf{Solution.} This is a homogeneous linear ODE with constant coefficients of order 5. Our characteristic equation is
    \[\lambda^5 - 6\lambda^4 + 13\lambda^3 - 14 \lambda^2 + 12\lambda - 8 = 0\]
    We'll guess and check for roots, we get that $\lambda = 2$ works so we can use polynomial long division to find the other factors. we get 
    \[(\lambda - 2)(\lambda^4 - 4\lambda^3 + 5\lambda^2 - 4\lambda +4) = 0\]
    If we take $\lambda = 1$, we get the second term is 0 so our second factor is $(\lambda - 2)$ and we can preform polynomial long division again to get
    \[(\lambda - 2)^2(\lambda^3-2\lambda^2 + \lambda -2)\]
    We notice again that $\lambda = 2$ is a root, so we get 
    \[(\lambda-2)^3(\lambda^2 + 1)\]
    We cannot simplify further, so our basis of solutions is 
    \[\{y_1 = e^{2x}, y_2 = xe^{2x}, y_3 = x^2e^{2x}, y_4 = \cos (x), y_5 = \sin (x)\}\]
    Therefore our general solution is 
    \[y = c_1e^{2x}+ c_2 xe^{2x}+c_3x^2e^{2x}+ c_4\cos (x) + c_5\sin (x)\]
    \section{Higher-order Euler-Cauchy Equations.}
    \begin{definition}
        A linear homogeneous ODE of order $n$ is called \emph{Euler-Cauchy} if it can be written under the form 
        \[x^my^{(m)} + a_{n-1}x^{n-1}y^{(n-1)} + \cdots a_2x^2y'' + a_1xy' + a_0 y = 0 \]
        for $x > 0$
    \end{definition}
    Similar to what we do in the case of second order Euler-Cauchy equations, we look for solutions of the form
    \[y = x^m\]
    We can differentiate $y$ $n$ times and subtitute back into our ODE to get the characteristic equation. The general form characteristic equation is long so this process is easier to be repeated for each order. Again, we have the roots of the characteristic equation fall into 2 cases,
    \begin{itemize}
        \item If $m$ is a root of the characteristic equation of multiplicity $k$, then it contributes the following equations to our basis of solutions
        \[y_1 = x^m, \ y_2 = x^m\ln x, \ y_3 = x^m (\ln x)^2 \ldots, y_k = x^m(\ln x)^{k-1}\]
        \item If $\alpha \pm i\beta$ is a pair of complex conjugate roots of the characteristic equation, then the pair contributes the following 2 equations to our basis of solutions
        \[y_1 = x^{\alpha} \cos(\beta \ln x), \ y_2 = x^\alpha\sin(\beta \ln x)\]
    \end{itemize}
    \noindent
    \textbf{Example.} Assume that the fifth order Euler-Cauchy ODE has the following characteristic equation.
    \[(m-1)^3(m^2+4) = 0\]
    Then, the root $m=1$ has multiplicity 3 and contributes the following 3 functions to our basis of solutions
    \[y_1 = x^1, y_2 = x^1 \ln x, y_3 = x^1(\ln x)^2\]
    Our second factor $m^2 + 4$ has roots $m = \pm 2i$. So it contributes to our basis of solutions 
    \[y_4 = x^0 \cos (2\ln x), y_5 = x^0 \sin(2 \ln x)\]
    Therefore, our general solution is 
    \[y = c_1x + c_2x\ln x + c_3 x (\ln x)^2 + c_4 \cos (2\ln x) + c_5\sin(2\ln x)\]
    \textbf{Example.} Solve the IVP 
    \[x^3y''' - 2x^2y'' + 4xy' - 4y = 0, \ y(1) = 0, \ y'(1) = -3, \ y''(1) = 3\]
    \textbf{Solution.} This is a linear homogeneous ODE of Euler-Cauchy type. Assume $y = x^m$, then we can differentiate $y$ 3 times to get
    \[y' = mx^{m-1}, \ y'' = m(m-1)x^{m-2}, \ y''' = m(m-1)(m-2)x^{m-3}\]
    Substituting these back into our ODE, we get
    \begin{align*}
      0 &= x^3m(m-1)(m-2)x^{m-3} - 2x^2m(m-1)x^{m-2} + 4xmx^{m-1} - 4x^m\\
      &= m(m-1)(m-2)x^m - 2m(m-1)x^m + 4mx^m - 4x^m\\  
      &= m(m-1)(m-2) - 2m(m-1) + 4m - 4\\ 
      &= m(m-1)(m-2) - 2m(m-1) + 4(m-1)\\
      &= (m-1)(m(m-2) - 2m + 4)\\
      &= (m-1)(m^2 - 4m + 4)\\
      &= (m-1)(m-2)^2
    \end{align*}
    So we have our roots $m_1 = 1$, $m_2 = m_3 = 2$. Thus we have the basis of solutions
    \[\{y_1 = x^1, y_2 = x^2, y_3 = x^2\ln x\}\]
    Our general solution is 
    \[y = c_1x + c_2 x^2 + c_3 x^2\ln x\]
    Using our intial conditions,
    \[y' = c_1 + 2c_2x + 2c_3x\ln x + c_3x\]
    \[y'' = 2c_2 + 2c_3\ln x + 2c_3 + c_3 = 2c_2 + 2c_3\ln x + 3c_3\]
    \[y(1) = 0 \implies 0 = c_1 + c_2\]
    \[y'(1) = -3 \implies -3 = c_1 + 2c_2 + c_3\]
    \[y''(1) = 3 \implies 3 = 2c_2 + 3c_3\]
    We can solve this system of equations using Guass Jordan elimination, and we find $c_1 = 12$, $c_2 = -12$, $c_3 = 9$. Therefore our unique solution is
    \[y = 12x -12x^2 + 9 x^2\ln x\]
    \textbf{Example.} Solve the IVP 
    \[x^3y''' + x^2y'' - 2xy' + 2y = 0, \ y(1) = 1, \ y'(1) = -2, \ y''(1) = 0\]
    \textbf{Solution.} This is a linear homogeneous ODE of Euler-Cauchy type. Assume $y = x^m$, then we can differentiate $y$ 3 times to get
    \[y' = mx^{m-1}, \ y'' = m(m-1)x^{m-2}, \ y''' = m(m-1)(m-2)x^{m-3}\]
    Then substituting these back into our ODE, we get
    \begin{align*}
        0 &= x^3m(m-1)(m-2)x^{m-3} + x^2m(m-1)x^{m-2} - 2xmx^{m-1} + 2x^m\\
        &= m(m-1)(m-2)x^m + m(m-1)x^m - 2mx^m + 2x^m\\  
        &= m(m-1)(m-2) + m(m-1) - 2m - 2\\ 
        &= m(m-1)(m-2) + m(m-1) - 2(m-1)\\
        &= (m-1)(m(m-2) + m - 2)\\
        &= (m-1)(m^2 - m - 2)\\
        &= (m-1)(m-2)(m+1)
    \end{align*}
    We have three roots $m_1 = 1$, $m_2 = 2$, and $m_3 = -1$. So we have the basis of solutions
    \[\{x^1, x^2, x^{-1}\}\]
    Our general solution is
    \[y = c_1x + c_2x^2 + c_3x^{-1}\]
    Using our initial conditions,
    \[y' = c_1 + 2c_2x - c_3x^{-2}\]
    \[y'' = 2c_2 + 2c_3x^{-3} \]
    \[y(1) = 1 \implies 1 = c_1 + c_2 + c_3\]
    \[y'(1) = -2 \implies -2 = c_1 + 2c_2 - c_3\]
    \[y''(1) = 0 \implies 0 = 2c_2 + 2c_3 \implies 0 = c_2 + c_1\]
    We can solve this system of equations using Guass Jordan elimination, and we find $c_1 = c_2 = 1$ and $c_3 = -1$. Therefore our unique solution is
    \[y = x +x^2 - 1x^{-1}\] 

\chapter{Non Homogeneous Linear ODEs}

We turn our attention to solving non-homogeneous linear ODEs of the form 
\[a_n(x)y^{(n)} + \cdots a_(x)y' + a_0(x)y = r(x)\]

\begin{theorem}
    If $a_n(x), \ldots, a_0(x), r(x)$ are continuous, then the general solution to the ODE has teh form 
    \[y = y_H + y_P\]
    where $y_H$ is the general solution to the corresponding homogeneous ODE 
    \[a_n(x)y^{(n)} + \cdots a_1(x)y' + a_0(x)y = 0\]
    and $y_P$ is any particular solution to the non-homogeneous ODE.
\end{theorem}
We will learn 2 methods to find $y_P$, the method of undetermined coefficients and the method of variation of parameters.
\section{The Method of Undetermined Coefficients}
The goal of this method is to find on particular solution to 
\[a_n(x)y^{(n)} + \cdots + a_1(x)y' + a_0(x)y = r(x)\]
This method works under the following two conditions, 
\begin{enumerate}
    \item All the coefficients on the left $(a_n(x), \ldots, a_0(x))$ are constant, or
    \item $r(x)$ is either a polynomial, an exponential function, a sinusoidal function, or a combination of these.
\end{enumerate}

\subsection{Decsription of The Method}
The undetermined Coefficients method is based on the following three rules, 
\begin{itemize}
    \item \textbf{Rule 1: Basic Rule.} If $r(x)$ is an exponential in the form $ke^{\lambda x}$, then our choice for $y_p$ is $Ae^{\lambda x}$. If it is a polynomial of degree $n$, then our choice for $y_p$ is a polynomial of degree $n$. If we have a sinusoidal function, $k\cos(wx)$ or $k\sin(wx)$, then we have $y_p = A\cos(wx) + B\sin(wx)$. If we have a combination of exponentional and sinusoidal $ke^{\alpha x}\cos(wx)$ or $ke^{\alpha x}\sin(wx)$, then $y_p = Ae^{\alpha x}\cos(wx) + Be^{\alpha x}\sin(wx)$. Then if we have a combination of a polynomial and exponentional $r(x) = p(x)e^{\lambda x}$, then $y_p = q(x)e^{\lambda x}$ where $q(x)$ is a polynomial of the same degree as $p(x)$.
    \item \textbf{Rule 2: The Modifcation Rule.} If at least one term in our inital choice of $y_p$ form rule 1 is an element of the basis of solutions for the corresponding homogeneous ODE, we modify our choice by multiplying with $x$. We repeate this process until no term in our choice is is common with the solutions of the corresponding homogeneous ODE. 
    \item \textbf{Rule 3: The Sum Rule.} If $r(x) = r_1(x) + \cdots r_t(x)$ then we choose $y_p$ as the sum of the particular solution corresponding to each $r_i(x)$.
\end{itemize}
\textbf{Example.} Solve the following IVP 
\[y''' + 3y'' - 4y' = 15e^x + 34\sin x; y(0) = 5, y'(0) = 1, y''(0) = -1\]
\textbf{Solution.} The corresponding homogeneous ODE is 
\[y''' + 3y'' - 4y' = 0\]
The characteristic equation is 
\[\lambda^3 + 3\lambda^2 - 4 \lambda = 0\]
Factoring this we get 
\[\lambda(\lambda^2 + 3\lambda - 4) = \lambda(\lambda-1)(\lambda+4) = 0\]
We have three real roots $\lambda_1 = 0$, $\lambda_2 = 2$, $\lambda = -4$. So our basis of solutions is 
Then our basis of solutions is 
\[{e^{0x}, e^x, e^{-4x}}\]
The general solution of the homogeneous ODE is 
\[y_H = c_1 + c_2e^x + c_3e^{-4x}\]
For $y_p$, we can apply the undetermined coefficients method. We have $r(x) = 15e^x + 34\sin x$. Using the sine rule, we start with $r_1(x) = 15e^x$. From rule 1, we have a choice $y_p = Ae^x$. Now $Ae^x$ is in the basis of solutions for $y_H$ so we must modify our choice by multiplying by $x$ to get 
\[y_p = Axe^x\]
Our other term $r_2(x) = 34\sin x$, from rule 1 we have a choice $y_p = B\sin x + C\cos x$. This is not in our basis of solutions for the homogeneous ODE so we do not need to modify it. Therefore, our choice for $y_p$ is
\[y_p = Axe^x + B\cos x + C \sin x\]
Now we can use our initial conditions to find $A$, $B$, and $C$.
\[y'_p = Ae^x + Axe^x - B\sin x + C \cos x\]
\[y''_p = 2Ae^x + Axe^x - B \cos x - C \sin x\]
\[y''' = 3Ae^x + Axe^x + B \sin x - C \cos x\]
Now we plug our equations into the non homogeneous ODE, 
\begin{align*}
    15e^x + 34\sin x &= 3Ae^x + Axe^x + B\sin x - C \cos x + 6Ae^x + 3Axe^x\\
    &-3C\cos x - 3C\sin x - 4Ae^x - 4Axe^x + 4B\sin x - 4C\cos x\\
    &= 5Ae^x + (-5C - 3B)\cos x + (5B - 3C)\sin x
\end{align*}
We need the coefficients on the left to match the coefficients on the right, so we get the following equations. $5A = 15$, $-5C - 3B = 0$, $5B - 3C = 34$. It's easy to see that $A = 3$, $B = 5$, and $C = -3$. Thus our particular solution is 
\[y_p = 3xe^x + 5\cos x - 3 \sin x \] 
Therefore our general solution for the non homogeneous ODE is $y = y_H + y_p$,
\[y = c_1 + c_2e^x + c_3e^{-4x} + 3xe^x + 5\cos x - 3 \sin x\]
Using our initial conidtions we can solve for our constants, 
\[y' = c_2e^x - 4c_3e^{-4x} + 3e^x + 3xe^x - 5 \sin x - 3 \cos x\]
\[y'' = c_2e^x + 16c_3e^{-4x} + 3e^x + 3e^x + 3xe^x - 5\cos x + 3 \sin x\]
\[y(0) = 5 \implies 5 = c_1 + c_2 + c_3 + 5 \implies c_1+ c_2 + c_3 = 0\]
\[y'(0) = 1 \implies c_2 - 4c_3 = 1\]
\[y''(0) = -1 \implies c_2 + 16c_3 = -2\]
\textbf{Example.} Find the general solution to the ODE, 
\[y'' + y = x+1 + 2\cos x\]
\textbf{Solution.} The corresponding homogeneous ODE is
\[y'' + y = 0\]
The characteristic equation is
\[\lambda^2 + 1 = 0\]
So we have 2 complex conjugate roots $\lambda = \pm i$, so our basis of solutions is
\[\{e^{0x}\cos x, e^{0x}\sin x\} = \{\cos x, \sin x\}\]
The general solution to our homogeneous ODE is 
\[y_H = c_1\cos x + c_2 \sin x\]
We can apply the undertermined coefficients method 
\[r(x) = x+1 + 2 \cos x\]
For $r_1(x) = x+1$, our choice from rule 1 is $y_p = Ax + B$. This is not in our basis of solutions for $y_H$ so we do not need to modify it. For $r_2(x) = 2 \cos x$, we have $y_p = D \cos x + D\sin x$. This is in our basis of solutions so we have to modify it by multiplying by $x$. We get 
\[y_p = Ax + B + Cx\cos x + Dx \sin x\]
Now we can plug our equations into the non homogeneous ODE,
\[y'_p = A + C\cos x - Cx\sin x + D\sin x + Dx\cos x\]
\begin{align*}
    y'' &= -C\sin x - C\sin x - Cx\cos x + D\cos x\\
    &- Dx\sin x = -2C\sin x - Cx\cos x + 2D \cos x - Dx \sin x  
\end{align*}
\begin{align*}
    x+1 + 2\cos x &= -2c\sin x - Cx\cos x + 2D\cos x - Dx\sin x \\
    &+ Ax + B + Cx \cos x + Dx \sin x\\
    &=-2C\sin x + 2D\cos x + Ax + B
\end{align*}
We need the coefficients on the left to match the coefficients on the right, so we get the following equations, $-2C = 0$, $2D = 2$, $A = 1$, $B = 1$. Therefore our particular solution is
\[y_p = x + 1 + x\sin x \]
Therefore our general solution for the non homogeneous ODE is 
\[y = y_H + y_p = c_1\cos x + c_2\sin x + x + 1 + x\sin x\]
\textbf{Example.} Solve the IVP 
\[y''' + y'' = 2e^{-x} + 18x - 2; y(0) = 1, y'(0) = -2, y''(0) = 0\]
\textbf{Solution.} The corresponding homogeneous ODE is
\[y''' + y'' = 0\]
The characteristic equation is
\[\lambda^3 + \lambda ^2 = 0\]
Factoring this we get
\[\lambda^2(\lambda + 1) = 0\]
Therefore $\lambda_1 = \lambda_2 = 0$, and $\lambda_3 = -1$. Then our basis of solutions is 
\[\{e^{0x}, xe^{0x},e^{-x}\} = \{1, x, e^{-x}\} \]
Then, we have $r(x) = 2e^{-x} + 18x - 2$. For $r_1(x) = 2e^{-x}$, our choice from rule 1 is $y_p = Ae^{-x}$. This is in our basis of solutions so we have to modify it by multiplying by $x$. We get $y_p = Axe^{-x}$. Next, we have $r_2(x) = 18x-2$. Then we choose a polynomial of degree 1 $y_p = Bx + C$. This is in our basis of solutions so we must modify it by multiplying by $x^2$. We get $y_p = Bx^3 + Cx^2$. Therefore our choice for $y_p$ is
\[y_p = Axe^{-x} + Bx^3 + Cx^2\]
Now we must differentiate and plug back into our non homogeneous ODE
\[y'_p = Ae^{-x} - Axe^{-x} + 3Bx^2 + 2Cx\]
\[y''_p = -Ae^{-x} - Ae^{-x} + Axe^{-x} + 6Bx + 2C= -2Ae^{-x} + Axe^{-x} 6Bx + 2C\]
\[y'''_p = 2Ae^{-x} + Ae^{-x} - Axe^{-x} + 6B = 3Ae^{-x} - Axe^{-x + 6B}\]
\begin{align*}
    2e^{-x} + 18x - 2 &= 3Ae^{-x} - Axe^{-x} + 6B -2Ae^{-x} + Axe^{-x} 6Bx + 2C\\
    &= Ae^{-x} + 6Bx + 6B + 2C
\end{align*}
So we have $A = 2$, $6B = 18 \implies B = 3$, $6B + 2C = -2 \implies 18 + 2C = -2 \implies C = -10$. Therefore our particular solution is
\[y_p = 2xe^{-x} + 3x^3 - 10x^2\]
Now, we have our general solution 
\[y = y_H + y_p = c_1 + c_2x + c_3e^{-x} + 2xe^{-x} + 3x^3 - 10x^2\]
Using our initial conditions we can solve for our constants,
\[y' = c_2 - c_3e^{-x} + 2e^{-x} - 2xe^{-x} + 9x^2 - 20x\]
\[y'' = c_3e^{-x} - 2e^{-x} - 2e^{-x} + 2xe^{-x} + 18x - 20 = c_3e^{-x} - 4e^{-x} + 2xe^{-x} + 18x - 20\]
\[y(0) = 1 \implies c_1 + c_3 = 1\]
\[y'(0) = -2 \implies c_2 - c_3 + 2 = -2 \implies c_2 -c_3 = -4\]
\[y''(0) = 0 \implies c_3 - 4 - 20 = 0 \implies c_3 = 24\]
Therefore, $c_1 = -23$, $c_2 = 20$, $c_3 = 24$. Thus our unique solution is
\[y = -23 + 20x + 24e^{-x} + 2xe^{-x} + 3x^3 - 10x^2\]
\textbf{Example.} Find the general solution of the ODE 
\[y''' - 2y'' - 9y' + 18y = 5e^{2x} - 18x^2 - 18x + 40\]
\textbf{Solution.} The corresponding homogeneous ODE is 
\[y''' - 2y'' - 9y' + 18y = 0\]
The characteristic equation is
\[\lambda^3 - 2\lambda^2 - 9\lambda + 18 = 0\]
We can factor this my grouping to get 
\[\lambda^2(\lambda - 2) -9(\lambda-2) = (\lambda-2)(\lambda^2-9) = (\lambda-2)(\lambda -3)(\lambda +3)= 0 \]
So we have 3 unique real roots $\lambda_1 = 2$, $\lambda_2 = 3$, $\lambda_3 = -3$. Then our basis of solutions is
\[\{e^{2x}, e^{3x}, e^{-3x}\}\]
Then, we have $r(x) = 5e^{2x} - 18x^2 - 18x - 40$. For $r_1(x) = 5e^{2x}$, our choice is $y_p = Ae^{2x}$. This is in our basis of solutions so we must modify it to get $y_p = Axe^{2x}$, then for $r_2(x) = -18x^2 - 18x + 40$, our choice is $y_p = Bx^2 + Cx + D$. This is not in our basis of solutions so we do not need to modify it. Therefore our choice for $y_p$ 
\[y_p = Axe^{2x} + Bx^2 + Cx + D\]
So we have 
Now we differential and plug back into our non homogeneous ODE,
\[y'_p = Ae^{2x} + 2Axe^{2x} + 2Bx + C\]
\[y''_p = 2Ae^{2x} + 2Ae^{2x} + 4Axe^{2x} + 2B = 4Ae^{2x} + 4Axe^{2x} + 2B\]
\[y'''_p = 8Ae^{2x} + 4Ae^{2x} + 8Axe^{2x} = 12Ae^{2x} + 8Axe^{2x}\]

\begin{align*}
    5e^{2x} - 18x^2 - 18x + 40 &= 12Ae^{2x} + 8Axe^{2x} - 8Ae^{2x}\\
    &- 8Axe^{2x} - 4B - 9Ae^{2x} - 18Axe^{2x} - 18Bx- 9C\\
    &+ 18Axe^{2x} + 18Bx^2 + 18Cx + 18D\\
    &= -5Ae^{2x} + 18Bx^2 + 18Cx - 18Bx + 18D - 4B - 9C\\
    &= 5Ae^{2x} + 18Bx^2 + (18C - 18B)x + 18D - 4B - 9C
\end{align*}
We get the equations $-5A = 5 \implies A = -1$, $18B = -18 \implies B = -1$, $18C - 18B = -18 \implies C = -2$, $18D - 4B - 9C = 18D + 4 + 18 = 40 \implies D = 1$. Therefore our particular solution is
\[y_p = -xe^{2x} - x^2 - 2x + 1\] 
Therefore our general solution to the non homogeneous ODE is 
\[y = c_1e^{2x} + c_2e^{3x} + c_3e^{-3x} - xe^{2x} - x^2 - 2x + 1\]
\section{The Method of Variation of Parameters}
Given a non homogeneous ODE of the form
\[a_n(x)y^{(n)} + \cdots + a_1(x)y' + a_0y(x) = r(x)\]
Our goal is to find $y_p$ in the equation $y = y_H + y_p$. The method of variation of parameters suggests a particular solution of the form 
\[y_p = u_1y_1 + u_2y_2 + \cdots + u_ny_n\]
Where $\{y_1,y_2,\ldots,y_n\}$ is a basis of solutions for the corresponding homogeneous ODE 
\[a_n(x)y^{(n)} + \cdots + a_1(x)y' + a_0(x)y = 0\]
$u_1, u_2, \ldots, u_n$ are functions that satisfy the following system of equations 
\[\begin{cases}
    u_1'y_1 + u2'y_2 + \cdots + u_n'y_n & =0\\
    u_1'y_1' + u2'y_2' + \cdots + u_n'y_n' &= 0\\
    \vdots & \vdots\\
    u_1'y_1^{(n-1)} + u_2'y_2^{(n-1)} + \cdots + u_n'y_n ^{(n-1)} &= \frac{r(x)}{a_n(x)}
\end{cases}\]
\textbf{Example.} Solve the IVP 
\[x^2y'' - 2xy' + 2y = 3\sqrt{x}; y(1) = 1, y'(1) = 0\]
\textbf{Solution.} The corresponding homogeneous ODE is 
\[x^2y'' - 2xy' + 2y = 0\]
The characteristic equation is
\[m^2 - 3m + 2 = 0 \implies (m-1)(m-2)=0\]
This gives us $m_1 = 1$, $m_2 = 2$. So our basis of solutions is
\[\{x^1,x^2\} \implies y_H = c_1x + c_2x^2\]
Now we must use variation of parameters for $y_p$. We have a solution of the form 
\[y_p = u_1y_1 + u_2y_2 = u_1x + u_2x^2\]
We must find $u_1$ and $u_2$ such that they satisfy the following system of equations 
\[\begin{cases}
    u_1'x + u_2'x^2 &= 0\\
    u_1' + 2u_2'x &= \frac{r(x)}{a_2(x)} = \frac{3\sqrt{x}}{x^2}\\
\end{cases}\]
Now we can combine the equations to get
\[xu_2' = 3x^{-3/2} \implies u_2' = 3x^{-5/2}\]
Similarly, $u_1' = -3x^{-3/2}$. Now we can integrate to find $u_1$ and $u_2$, 
\begin{align*}
    u_1 &= \int u_1'dx\\
    &= \int -3x^{-3/2}dx\\
    &= -3 \int x^{-3/2}dx\\
    &= -3 \frac{x^{-1/2}}{-1/2}\\
    &= \frac{6}{\sqrt{x}}\\
    u_2 &= \int 3x^{-5/2}dx\\
    &= 3 \frac{x^{-3/2}}{-3/2}\\
    &= -2x^{-3/2}
\end{align*} 
Thus our particular solution is
\[y_p = u_1y_1 + u_2 y_2 = \frac{6}{\sqrt{x}}x - 2x^{-3/2}x^{2} = \frac{6x-2}{\sqrt{x}}\]
Therefore our general solution is
\[y = y_H + y_p = c_1x + c_2x^2 + 6\sqrt{x} - 2x^{-1/2}\]
Now we can use our initial conditions to solve for our constants,
\[y' = c_1 + 2c_2x + 3x^{-1/2} + x^{-3/2}\]
\[y(1) = 1 \implies c_1 + c_2 = -3\]
\[y'(1) = 0 \implies c_1 + 2c_2 = -4\]
Then we get $c_2 = -1$, $c_1 = -2$. Therefore our unique solution is
\[y = -2x - 1x^2 + 6\sqrt{x} - 2x^{-1/2}\]
\textbf{Example.} Find the general solution for the ODE 
\[x^3y''' -3x^2y'' + 6xy' - 6y = 3(1 + \ln x) \]
\textbf{Solution.} The corresponding homogeneous ODE is
\[x^3y''' - 3x^2y'' + 6xy' - 6y = 0\]
We take $y = x^m$ as a solution and compute 
\[y' = mx^{m-1}\]
\[y'' = m(m-1)x^{m-2}\]
\[y''' = m(m-1)(m-2)x^{m-3}\]
Then we plug back in 
\[m(m-1)(m-2) - 3m(m-1) + 6m - 6 = 0\]
We can factor this as 
\[(m-1)(m(m-2) - 3m + 6) = (m-1)(m-2)(m-3)\]
So we have 3 unique real roots $m_1 = 1$, $m_2 = 2$, $m_3 = 3$. So our basis of solutions is
\[\{x^1, x^2,x^3\}\]
Now using variation of parameters, we have a solution of the form
\[y_p = u_1y_1 + u_2y_2 + u_3y_3 = u_1x + u_2x^2 + u_3x^3\]
We must find $u_1$, $u_2$, and $u_3$ such that they satisfy the following system of equations
\[\begin{cases}
    u_1'x + u_2'x^2 + u_3'x^3 &= 0\\
    u_1' + 2u_2'x + 3u_3'x^2 &= 0\\
    u_1'(0) + 2u_2' + 6u_3'x &= \frac{3(1 + \ln x)}{x^3}
\end{cases} \rightarrow \begin{cases}
    u_1' + u_2'x + u_3'x^2 &= 0\\
    u_1' + 2u'_2x + 3u_3'x^2 &= 0\\
    0 + 2u_2' + 6xu_3' &= \frac{3(1+\ln x)}{x^3}
\end{cases}\]
Subtracting the first equation from the second one we get 
\[u_2'x + 2u_3x^2 = 0 \implies u_2' + 2u_3'x = 0\]
Now we we subtract 2 times this equation from equation 3 to get
\[2xu_3' = \frac{3(1+\ln x)}{x^3} \implies u_3' = \frac{3(1+\ln x)}{2x^4}\]
Then 
\[u_2' = -\frac{3(1+\ln x)}{x^3}\]
\[u_1' = xu'_2 - x^2u_3' = \frac{3(1+\ln x)}{x^2} - \frac{3(1+\ln x)}{x^2} = \frac{3-\ln x}{2x^2}\]
Now we integrate each function 
\begin{align*}
    u_1 &= \int u_1'dx\\
    &= \int \frac{3-\ln x}{2x^2}dx\\
    &= (3-\ln x)\left(-\frac{1}{2x}\right) - \int \frac{1}{2x^2}dx\\
    &= \frac{\ln x - 3}{2x} - \frac{1}{2}\int x^{-2}dx\\
    &= \frac{\ln x - 3}{2x} + \frac{1}{2x}\\
    &= \frac{\ln x - 2}{2x}\\
    u_2 &= \int u_2'dx\\
    &= \int \frac{-3(1+\ln x)}{x^3}dx\\
    &= -3(1+\ln x)\left(-\frac{1}{2x^2}\right) - \int \frac{3}{x^3}dx\\
    &= \frac{3(1+\ln x)}{2x^2} - \frac{3}{2}\int x^{-3}dx\\
    &= \frac{3(1+ \ln x)}{2x^2} + \frac{3}{4x^2}\\
    &= \frac{6 (1 + \ln x) + 3 }{4x^2}\\
    u_3 &= \int u_3'dx\\
    &= \int \frac{3(1+\ln x)}{2x^4}dx\\
    &= 3(1 + \ln x) \left(-\frac{1}{6x^3}\right) - \int \frac{3}{x}\frac{1}{6x^3}dx\\
    &= -\frac{1 + \ln x}{2x^3} + \frac{1}{2}\int x^{-4}dx\\
    &= \frac{-4-3\ln x}{6x^3}
\end{align*}
Therefore our particular solution is
\begin{align*}
    y_p &= \frac{\ln x - 2}{2x} \cdot x + \frac{6\ln x + 9}{4x^2} \cdot x^2 - \frac{4 + 3 \ln x}{6x^3}\cdot x^3\\
    &= \frac{\ln x - 2}{2} + \frac{6\ln x + 9}{4} - \frac{4 + 3 \ln x}{6}\\
    &= \frac{6\ln x - 12}{12} + \frac{18\ln x + 27}{12} - \frac{8 + 6 \ln x}{12}\\
    &= \frac{6\ln x + 18 \ln x - 6 \ln x - 12 + 27 - 6}{12}\\
    &= \frac{18\ln x + 9}{12}\\
    &= \frac{6\ln x + 3}{4}
\end{align*}
Thus our general solution is 

\[y = c_1x + c_2x^2 + c_3x^3 + \frac{6\ln x + 3}{4}\]

\chapter{Systems of Linear ODEs}
\section{Generalities}
\begin{definition}
    A system of first order linear ODEs is a system of the form 
    \[y_1' = a_{11}(x)y_1 + a_{12}(x)y_2 + \cdots a_{1n}(x)y_n + r_1(x)\]
    \[y_2' = a_{21}(x)y_1 + a_{22}(x)y_2 + \cdots a_{2n}(x)y_n + r_2(x)\]
    \[\vdots\]
    \[y_n' = a_{n1}(x)y_1 + a_{n2}(x)y_2 + \cdots a_{nn}(x)y_n + r_n(x)\]
    Let 
    \[Y = \begin{bmatrix}
        y_1(x)\\
        y_2(x)\\
        \vdots\\
        y_n(x)
    \end{bmatrix}\]
    \[Y' = \begin{bmatrix}
        y_1'(x)\\
        y_2'(x)\\
        \vdots\\
        y_n'(x)
    \end{bmatrix}\]
    \[A = \begin{bmatrix}
        a_{11}(x) & a_{12}(x) \cdots a_{1n}(x)\\
        a_{21}(x) & a_{22}(x) \cdots a_{2n}(x)\\
        \vdots & \vdots \ddots \vdots\\
        a_{n1}(x) & a_{n2}(x) \cdots a_{nn}(x)\\
    \end{bmatrix}\]
    \[\vec{r(x)} = \begin{bmatrix}
        r_1(x)\\
        r_2(x)\\
        \vdots\\
        r_n(x)
    \end{bmatrix}\]
    The above system can be written as 
    \[Y' = AY + \vec{r(x)}\]
    If $\vec{r(x)} = 0$, then we say that this system is homogeneous. Otherwise, it is non-homogeneous.
\end{definition}
In this course we only deal with systems of the form $Y' = AY + \vec{r(x)}$ where $A$ is a $2\times 2$ matrix with constant coefficients. 
\section{Homogeneous Systems of First order ODEs with Constant Coefficients}
Similar to in the case of First-Order constant coefficient ODEs, we have 
\[y' = ay \implies y' - ay = 0\]
We look for exponential solutions,
\[Y = Ve^{\lambda x} \implies Y' = \lambda Ve^{\lambda x}\]
Then we have
\[\lambda V e^{\lambda x} = AV e^{\lambda x} \implies AV = \lambda V\]
This means that $\lambda$ is an eigen value for $A$ and $V$ is a corresponding eigen vector. 
\subsection{Steps to Solving Homogeneous Systems}
To solve the homogeneous system $Y' = AY$,
\begin{enumerate}
    \item Find the eigen values of $A$ by solving the characteristic equation $|\lambda I - A| = 0$
    \item If $A$ has 2 distinct real eigenvalues $\lambda_1,\lambda_2$, then the general solution is
    \item \[Y = c_1V_1e^{\lambda_1x} + c_2V_2e^{\lambda_2x}\]
    \item If $\lambda$ is an eigenvalue with multiplicity 2, then the general solution is
    \[Y = c_1Ve^{\lambda x} + c_2(xV + \rho)e^[\lambda x]\]
    where $V$ is a basis for the eigenspace of $\lambda$ and $\rho$ is a generalized eigenvector.
    \item If $\lambda_1 = \alpha + i\beta$ and $\lambda_2 = \alpha - i\beta$ are complex eigenvalues, then find a basis for the complex eigenspace for $\lambda_1 = \alpha + i\beta$, and form the complex vector $Ve^{(\alpha + i\beta)x}$. Then write 
    \[Ve^{(\alpha + i\beta)x} = V_1 + iV_2\]
    where $V_1$, $V_2$ have real entries. Then the general solution is
    \[y = c_1V_1 + c_2V_2\]
\end{enumerate}

\textbf{Example.} Solve the following IVP 
\[\begin{cases}
    y_1' = 2y_1 + 9y_2 & y_1(0) = -1\\
    y_2' = y_1 + 2y_2 & y_2(0) = 2
\end{cases}\]
\textbf{Solution.} We start by rewriting this in matrix form, 
\[Y' = \begin{bmatrix}
    2 & 9\\
    1 & 2
\end{bmatrix}Y\]
Then we find the eigenvalues of $A$ by solving the characteristic equation
\[\det\begin{bmatrix}
    \lambda - 2 & -9\\
    -1 & \lambda - 2
\end{bmatrix} = (\lambda - 2)^2 - 9 = \lambda^2 - 4\lambda - 5\]
Our roots are $\lambda_1 = -1$, $\lambda_2 = 5$. We have two distinct real eigen values so our general solution is
\[Y = c_1V_1e^{-x} + c_2V_2e^{5x}\]
We want to find $V_1$ and $V_2$, we solve the 
\[[A + I| 0] = \begin{bmatrix}
    3 & 9 & 0\\
    1 & 3 & 0
\end{bmatrix}\]
We have $x_2$ is a free variable $x_2 = t$, and $x_1 = -3t$. Thus our eigenvector is 
\[V_1 = \begin{bmatrix}
    -3t\\
    t
\end{bmatrix} = t\begin{bmatrix}
    -3\\
    1
\end{bmatrix}\]
For $\lambda_2 = 5$, we solve the system
\[[A - 5I|0] = \begin{bmatrix}
    -3 & 9 & 0\\
    1 & -3 & 0
\end{bmatrix}\]
This gives us $x_2 = t$, $x_1 = 3t$. Thus our eigenvector is
\[V_2 = \begin{bmatrix}
    3t\\
    t
\end{bmatrix} = t\begin{bmatrix}
    3\\
    1
\end{bmatrix}\]
Therefore our general solution is
\[Y = c_1\begin{bmatrix}
    -3\\
    1
\end{bmatrix}e^{-x} + c_2\begin{bmatrix}
    3\\
    1
\end{bmatrix}e^{5x} = \begin{bmatrix}
    -3c_1e^{-x} + 3c_e^{5x}\\
    c_1e^{-x} + c_2e^{5x}
\end{bmatrix}\]
Now we can use our initial conditions,
\[y_1(0) = -1 \implies -3c_1 + 3c_3 = -1\]
\[y_2(0) = 2 \implies c_1 + c_2 = 2\]
Solving this gives us $c_1 = \frac{7}{6}$, $c_2 = \frac{5}{6}$. Therefore our unique solution is
\[Y = \begin{bmatrix}
    -\frac{7}{2}e^{-x} + \frac{5}{2}e^{5x}\\
    \frac{7}{6}e^{-x} + \frac{5}{6}e^{5x}
\end{bmatrix}\]

\textbf{Example.} Solve the IVP 
\[Y' = \begin{bmatrix}
    1 & - 1\\
    1 & 3
\end{bmatrix}Y; Y(0) = \begin{bmatrix}
    1\\
    3
\end{bmatrix}\]
\textbf{Solution.} Solve for the eigenvalues of $A$ 
\[\det\begin{bmatrix}
    \lambda - 1 & 1\\
    -1 & \lambda - 3
    \end{bmatrix} = (\lambda - 1)(\lambda - 3) + 1 = \lambda^2 - 4\lambda + 4 = (\lambda - 2)^2\]
We have $\lambda = 2$ with multiplicity 2, so we find the eigenvector by solving the system
\[[A - 2I|0] = \begin{bmatrix}
    -1 & -1 & 0\\
    1 & 1 & 0
\end{bmatrix}\]
Solving this gives us $x_2 = t$, $x_1 = -t$. Thus our eigenvector is
\[V = \begin{bmatrix}
    -1\\
    1
\end{bmatrix}\]
The generalized eigenvector $\rho$ is the solution to the system
\[[A - 2I | V] = \begin{bmatrix}
    -1 & -1 & -1\\
    1 & 1 & 1
\end{bmatrix}\]
This gives us $x_2 = t$, $x_1 = 1 -t$, thus 
\[\rho = \begin{bmatrix}
    1 - t\\
    t
\end{bmatrix}\]
We take a specific value of $t$, for simplicity take $t = 0$ to get 
\[\rho = \begin{bmatrix}
    1\\
    0
\end{bmatrix}\]
Therefore our general solution is
\begin{align*}
    Y &= c_1Ve^{2x} + c_2(xV + \rho)e^{2x}\\
    &= c_1\begin{bmatrix}
        -1\\
        1
    \end{bmatrix}  e^{2x} + c_2\left(x\begin{bmatrix}
        -1\\
        1
    \end{bmatrix} + \begin{bmatrix}
        1\\
        0
    \end{bmatrix}\right) e^{2x}\\
    &= \begin{bmatrix}
        -c_1e^{2x} + c_2(-x + 1)e^{2x}\\
        c_1e^{2x} + c_2xe^{2x}
    \end{bmatrix}
\end{align*}
Now using our initial conditions we get 
\[Y(0) = \begin{bmatrix}
    1\\
    3
\end{bmatrix}\implies \begin{bmatrix}
    -c_1 + c_2\\
    c_1
\end{bmatrix} = \begin{bmatrix}
    1\\
    3
\end{bmatrix}\]
Thus $c_1 = 3$, $c_2 = 4$. The solution to our IVP is 
\[Y = \begin{bmatrix}
    -3e^{2x} + 4(-x+1)e^{2x}\\
    3e^{2x} + 4xe^{2x}
\end{bmatrix}\]
\textbf{Example.}
Solve the IVP 
\[Y' = \begin{bmatrix}
    2 & -9\\
    1 & 2\\
\end{bmatrix}Y;\ Y(0) = \begin{bmatrix}
    6\\
    -4
\end{bmatrix}\]
\textbf{Solution.} Find the eigenvalues of $A$, 
\begin{align*}
    \det\begin{bmatrix}
        \lambda - 2 & 9\\
        -1 & \lambda - 2
    \end{bmatrix} &= (\lambda - 2)^2 + 9\\
    &= \lambda^2 - 4\lambda + 13
    \implies \lambda &= \frac{4 \pm \sqrt{16 - 4(13)}}{2}\\
    &= \frac{4 \pm 2\sqrt{4 - 13}}{2}\\
    &= 2 \pm \sqrt{-9}\\
    &= 2 \pm 3i
\end{align*}
We have 2 complex conjugate roots $\lambda_1 = 2 + 3i$, $\lambda_2 = 2-3i$. Now we find the eigenvector for $\lambda_1 = 2 + 3i$. Recall that the general solutions for the system is 
\[Y = Ve^{\lambda_1 x} = c_1V_1 + c_2V_2\]
Where $V$ is the eigenvector and $V_1$, $V_2$ are the real components of $V$. We solve the system
\[[A - (2+3i)I|0] = \begin{bmatrix}
    2 - (2 + 3i) & -9 & 0\\
    -1 & 2 - (2 + 3i) & 0
\end{bmatrix} = \begin{bmatrix}
    -3i & -9 & 0\\
    -1 & -3i & 0
\end{bmatrix}\]
Solving this we get 
\[[A - (2+3i)I|0] = \begin{bmatrix}
    1 & -3i & 0\\
    0 & 0 & 0
\end{bmatrix}\]
$x_2 = t$ is a free variable, $x_1 = 3it$. Thus our eigenvector is
\[V = \begin{bmatrix}
    3i\\
    1
\end{bmatrix}t\]
Now we multiply by $e^{\lambda x}$ and split the vector up 
\begin{align*}
    Ve^{(2+3i)x} &= \begin{bmatrix}
        3i\\
        1
    \end{bmatrix}e^{2x}e^{3ix}\\
    &= \begin{bmatrix}
        3i\\
        1
    \end{bmatrix}e^{2x}(\cos(3x) + i\sin(3x))\\
    &= \begin{bmatrix}
        3i\\
        1
    \end{bmatrix}(e^{2x}\cos(3x) + ie^{2x}\sin(3x))\\
    &= \begin{bmatrix}
        3ie^{2x}\cos(3x) - 3e^{2x}\sin(3x)\\
        e^{2x}\cos(3x) + ie^{2x}\sin(3x)
    \end{bmatrix}\\
    &= \begin{bmatrix}
        -3e^{2x}\sin(3x)\\
        e^{2x}\cos(3x)
    \end{bmatrix} + i\begin{bmatrix}
        3e^{2x}\cos(3x)\\
        e^{2x}\sin(3x)
    \end{bmatrix}
\end{align*}
Therefore our general solution to the system is 
\[Y = c_1\begin{bmatrix}
    -3e^{2x}\sin(3x)\\
    e^{2x}\cos(3x)
\end{bmatrix} + c_2\begin{bmatrix}
    3e^{2x}\cos(3x)\\
    e^{2x}\sin(3x)
\end{bmatrix} = \begin{bmatrix}
    -3c_1e^{2x}\sin(3x) + 3c_2e^{2x}\cos(3x)\\
    c_1e^{2x}\cos(3x) + c_2e^{2x}\sin(3x)
\end{bmatrix}\]
Now using our inital conditions 
\[Y(0) = \begin{bmatrix}
    6\\
    -4
\end{bmatrix} \implies \begin{bmatrix}
    3c_2\\
    c_1
\end{bmatrix} = \begin{bmatrix}
    6\\
    -4
\end{bmatrix}\]
Therefore $c_1 = -4$, $c_2 = 2$ and our unique solution is 
\[Y = \begin{bmatrix}
    12e^{2x}\sin(3x)+6e^{2x}\cos(3x)\\
    -4e^{2x}\cos(3x) + 2e^{2x}\sin(3x)
\end{bmatrix}\]

\section{Non-homogeneous Systems of First Order ODEs}
In this section, we find the general solution of the following system
\[y_1' = a_{11}y_1 + a_{12}y_2 + r_1(x)\]
\[y_2' = a_{21}y_1 + a_{22}y_2 + r_2(x)\]
We can write this in matrix form as
\[Y' = AY + \vec{r(x)}\]
Similar to the case of non-homogeneous ODEs, the general solutions has the form 
\[Y = Y_H + Y_p\]
Where $Y_H$ is the general solution to the corresponding homogeneous system $Y' = AY$, and $Y_p$ is the particular solution to the non-homogeneous system $Y' = AY + \vec{r(x)}$. To find $y_p$, we use the method of undetermined coefficients. The constants in the undetermined coefficients method are replaced by constant vectors. For example, if 
\[\vec{r(x)} = \begin{bmatrix}
    2\cos x + x^2 - x \\
    3e^x - x^2 + 2x + 1
\end{bmatrix}\]
Then we decompose it as 
\[\begin{bmatrix}
    2\\
    0
\end{bmatrix}\cos x + \begin{bmatrix}
    0\\
    3
\end{bmatrix} + \begin{bmatrix}
    1\\
    -1
\end{bmatrix} + \begin{bmatrix}
    -1\\
    2
\end{bmatrix}x + \begin{bmatrix}
    0\\
    1
\end{bmatrix}\]
Then we find the particular solution for each term and add them together, in this case 
\[r_1(x) = \begin{bmatrix}
    2\\
    0
\end{bmatrix}\cos x \implies \vec{y_p} = \vec{U}\cos x + \vec{V}\sin x\]
\[\begin{bmatrix}
    0\\
    3
\end{bmatrix}e^x \implies \vec{y_p} = \vec{W}e^x\]
\[\begin{bmatrix}
    1\\
    -1
\end{bmatrix}x^2 + \begin{bmatrix}
    -1\\
    2
\end{bmatrix}x + \begin{bmatrix}
    0\\
    1
\end{bmatrix} \implies \vec{y_p} = \vec{C}x^2 + \vec{D}x + \vec{U}\]
\textbf{Example.} Solve the IVP 
\[\vec{y'} = \begin{bmatrix}
    9 & 18\\
    -2 & -3
\end{bmatrix}\vec{y} + \begin{bmatrix}
    9x - 51\\
    7 + e^{2x}
\end{bmatrix}; \ \vec{y}(0) = \begin{bmatrix}
    1\\
    0
\end{bmatrix}\]
\textbf{Solution.} Start by solving the corresponding ODE
\[\vec{y}' = \begin{bmatrix}
    9 & 18\\
    -2 & -3
\end{bmatrix}\vec{y}\]
We find the eigenvalues of $A$ by solving the characteristic equation
\[\det\begin{bmatrix}
    9 - \lambda & 18\\
    -2 & -3 - \lambda\\
\end{bmatrix} = (9 - \lambda)(-3-\lambda) + 36 = \lambda^2 -6\lambda + 9\]
This gives us the equation $(\lambda -3)^2$ so we have $\lambda = 3$ with multiplicity 2. Now we find the eigenvector 
\[[A - 3I| 0] = \begin{bmatrix}
    6 & 18 & 0\\
    -2 & -6 & 0
\end{bmatrix} \sim \begin{bmatrix}
    1 & 3 & 0\\
    0 & 0 & 0
\end{bmatrix}\]
This gives us $x_2 = t$, $x_1 = -3t$. So our eigenvector is 
\[V = t\begin{bmatrix}
    -3\\
    1
\end{bmatrix}\]
Now we need a generalized eigenvector, so we solve the system
\[[A - 3I| V] = \begin{bmatrix}
    6 & 18 & -3\\
    -2 & -6 & 1
\end{bmatrix} \sim \begin{bmatrix}
    1 & 3 & -1/2\\
    0 & 0 & 0
\end{bmatrix}\]
This gives us $x_2 = t$, $x_1 = -3t - 1/2$, the general solutions is 
\[\vec{\rho} = \begin{bmatrix}
    -3t - 1/2\\
    t
\end{bmatrix}\]
We'll take $t= 0$, 
\[\vec{\rho} = \begin{bmatrix}
    -1/2\\
    0
\end{bmatrix}\]
The general solution to the homogeneous ODE is 
\begin{align*}
    \vec{y}_H &= c_1\begin{bmatrix}-3\\1\end{bmatrix}xe^{3x} + c_2\left(x\begin{bmatrix}
        -3\\
        1
    \end{bmatrix} + \begin{bmatrix}
        -1/2\\
        0
    \end{bmatrix}\right)e^{3x}  \\
    &= \begin{bmatrix}
        -3c_1e^{3x} - 3c_2xe^{3x} - \frac{1}{2}c_2e^{3x}\\
        c_1e^{3x} + c_2xe^{3x}
    \end{bmatrix}
\end{align*}
Now for $y_p$, we have 
\[r(x) = \begin{bmatrix}
    9x - 51\\
    7 + e^{2x}
\end{bmatrix} = \begin{bmatrix}
    9\\
    0
\end{bmatrix}x + \begin{bmatrix}
    -51\\
    7
\end{bmatrix} + \begin{bmatrix}
    0\\
    1
\end{bmatrix}e^{2x}\]
For the first term, we have a polynomial of degree 1, so 
\[y_p = Ux + V\]
For the exponential term, we take 
\[y_p = We^{2x}\]
Therefore we have the equation 
\[y_p = Ux + V + We^{2x}\]
We can rewrite the non-homogeneous system as 
\[y' = \begin{bmatrix}
    9 & 18\\
    -2 & -3
\end{bmatrix}y + \begin{bmatrix}
    9\\0
\end{bmatrix}x + \begin{bmatrix}
    -51\\7
\end{bmatrix} + \begin{bmatrix}
    0\\1
\end{bmatrix}e^{2x}\]
We must compute the derivative of $y_p$, 
\[y_p' = U + 2We^{2x}\]
This gives us 
\begin{align*}
   U + 2We^{2x} &= A(Ux + V + We^{2x}) + \begin{bmatrix}
    9\\0
\end{bmatrix}x + \begin{bmatrix}
    -51\\7
\end{bmatrix} + \begin{bmatrix}
    0\\1
\end{bmatrix}e^{2x}\\
&= \left(AU + \begin{bmatrix}
    9\\0
\end{bmatrix}\right) + \left(AW + \begin{bmatrix}
    0\\1
\end{bmatrix}\right)e^{2x} +AV + \begin{bmatrix}
    -51\\7
\end{bmatrix}
\end{align*}
This gives us the three equations 
\[AU + \begin{bmatrix}
    9\\0
\end{bmatrix} = 0\]
\[AV + \begin{bmatrix}
    -51\\7
\end{bmatrix} = U\]
\[AW + \begin{bmatrix}
    0\\1
\end{bmatrix} = 2W\]
The first equation gives us 
\[AU = \begin{bmatrix}
    -9\\0
\end{bmatrix} \implies U = A^{-1}\begin{bmatrix}
    -9\\0
\end{bmatrix} = \frac{1}{9}\begin{bmatrix}
    -3 & -18\\
    2 & 9
\end{bmatrix}\begin{bmatrix}
    -9\\0
\end{bmatrix} = \begin{bmatrix}
    3\\-2
\end{bmatrix}\]
The second equation gives us 
\[AV = U - \begin{bmatrix}
    -51\\7
\end{bmatrix} = \begin{bmatrix}
    54\\-9
\end{bmatrix}\implies V = A^{-1}\begin{bmatrix}
    54\\-9
\end{bmatrix} = \begin{bmatrix}
    0\\3
\end{bmatrix}\]
The third equation gives us 
\begin{align*}
    AW - 2W = \begin{bmatrix}
        0\\-1
    \end{bmatrix} &\implies (A - 2I)W = \begin{bmatrix}
        0\\-1
    \end{bmatrix}\\
    &\implies W = \frac{1}{1}\begin{bmatrix}
        -5 & -18\\
        2 & 7
    \end{bmatrix}\begin{bmatrix}
        0\\-1
    \end{bmatrix}\\
    &\implies W = \begin{bmatrix}
        18\\-7
    \end{bmatrix}        
\end{align*}
This gives us our particular solution
\begin{align*}
    y_p &= Ux + V + We^{2x}\\
    &= \begin{bmatrix}
        3\\-2
    \end{bmatrix}x + \begin{bmatrix}
        0\\3
    \end{bmatrix} + \begin{bmatrix}
        18\\-7
    \end{bmatrix}e^{2x}\\
    &= \begin{bmatrix}
        3x + 18e^{2x}\\
        -2x+3 - 7e^{2x}
    \end{bmatrix}  
\end{align*}


The general solution to the non-homogeneous ODE is 
\begin{align*}
    y &= y_H + y_p\\
    &= \begin{bmatrix}
        -3c_1e^{3x} - 3c_2xe^{3x} - \frac{1}{2}e^{3x}\\
        c_1e^{3x} + c_2xe^{3x}
    \end{bmatrix}
    + \begin{bmatrix}
        3x + 18e^{2x}\\
        -2x+3 - 7e^{2x}
    \end{bmatrix}\\
    &= \begin{bmatrix}
        -3c_1e^{3x} - 3c_2xe^{3x} - \frac{1}{2}e^{3x} + 3x + 18e^{2x}\\
        c_1e^{3x} + c_2xe^{3x} -2x+3 - 7e^{2x}
    \end{bmatrix}
\end{align*}
Using the intial condition, 
\[y(0) = \begin{bmatrix}
    1\\0
\end{bmatrix}\implies \begin{bmatrix}
    -3c_1 - \frac{1}{2}c_2 + 18\\
    c_1 -4
\end{bmatrix} = \begin{bmatrix}
    1\\0
\end{bmatrix}\]
This gives us the equation 
\[-3c_1 - \frac{1}{2} + 18 = 1\]
\[c_1 = 5 \implies c_2 = 10\]
Therefore the unique solution is 
\[y = \begin{bmatrix}
    -15e^{3x} - 30xe^{3x} - \frac{1}{2}e^{3x} + 3x + 18e^{2x}\\
    5e^{3x} + 10xe^{3x} -2x+3 - 7e^{2x}
\end{bmatrix}\]

\chapter{Laplace Transforms}

\begin{definition}
    Let $a \in \real$, and let $f$ be a function on $[a, \infty)$ then we define the type 1 improper integral 
    \[\int_a^\infty f(x)dx = \lim_{L\rightarrow\infty} \int_a^L f(x)dx\]
    If this limit exists, then the integral converges. 
\end{definition}

\noindent
\textbf{Example.} 
\[\int_0^\infty \frac{1}{1+x}dx = \lim_{L\rightarrow\infty}\int_0^L \frac{1}{1+x}dx = \lim_{L\rightarrow \infty} \ln(1 + L)\]
We can see that this integral diverges since $\ln(1 + L) \rightarrow \infty$.\\[2ex]
\noindent
\textbf{Example.} 
\[\int_0^\infty \frac{1}{1+x^2}dx = \lim_{L\rightarrow\infty} \arctan(L) = \frac{\pi}{2}\]
This integral converges. 
\begin{definition}
    Let $\alpha \in \real$. A function $f: [\alpha, \infty) \rightarrow \infty$ is called of \emph{exponential order} $\alpha$ if there exists $t_0$ such that  
    \[|f(t)| \leq Me^{\alpha t}\]
    for some $M > 0$, and any $t \geq t_0$. 
\end{definition}
\noindent
\textbf{Example.} Consider the constant function $f(t) = 1$. This is of exponential order $\alpha = 0$ since for any $M \geq 1$, 
\[|f(t)| = 1 \leq Me^{0 t} = M\]
\textbf{Example.} The function $f(t) = t^n$ is of exponential order $\alpha$ for any $\alpha > 0$. (Proof omitted.)\\[2ex]
\textbf{Example.} The exponential function $f(t) = e^{at}$ is of exponential order $\alpha = a$. \\[2ex]
\textbf{Example.} Sinusoidal functions $f(t) = \sin t$ and $g(t) = \cos t$ are of exponential order $\alpha = 0$ since 
\[|\cos t| \leq 1 = 1 e^{0 t}\]
\begin{definition}[Piecewise Continuity]
    We say that a function $f(t)$ is \emph{piecewise continuous} on $[a,b]$ if there exists subintervals 
    \[t_0 = a, t_1, \ldots, t_n = b\]
    such that 
    \begin{enumerate}[label=(\roman*)]
        \item The limits 
        \[\lim_{t \rightarrow t_i^-}f(t), \lim_{t\rightarrow t_i^+} f(t), \lim_{t\rightarrow a^+}f(t), \lim_{t\rightarrow b^-}f(t)\]
        all exist, and
        \item $f(t)$ is continuous on every open subinterval $(x_i, x_{i+1})$.
    \end{enumerate}
\end{definition}

\begin{definition}[Laplace Transform]
    Let $f(t)$ be a function, then we define the Laplace transform of $f(t)$ as 
    \[F(s) = \mathcal{L}\{f(t)\} = \int_{0}^\infty e^{-st}f(t)dt\]
\end{definition}
\noindent
\textbf{Note.} If $F(s) = \mathcal{L}\{f(t)\}$, then $f(t)$ is the \emph{inverse} Laplace transform of $F(s)$ written as 
\[f(t) = \mathcal{L}^{-1}\{F(s)\}\]
\begin{theorem}
    Let $f(t)$ be a function such that 
    \begin{enumerate}[label=(\roman*)]
        \item $f$ is of exponential order $\alpha$, and 
        \item $f$ is piecewise continuous on $[0, \infty)$. 
    \end{enumerate}
    Then the Laplace transform $\mathcal{L}\{f(t)\}$ exists for any $s \geq \alpha$. 
\end{theorem}
\begin{proof}
    We can divide the improper integral over a subinterval $0 \leq t \leq T$, and $T \leq t < \infty$, for any $T$, 
    \[\int_0^\infty e^{-st}f(t)dt = \int_0^T e^{-st}f(t)dt + \int_T^\infty e^{-st}f(t)dt\]
    Since $f(t)$ is piecewise continuous on $0 \leq t \leq T$, this interval is compact and therefore $f$ is Riemann integrable over this region. (See proof in real analysis notes). Then, since $f(t)$ is $O(e^{\alpha t})$, there exists constants $M$ and $T$ such that $|f(t)| \leq Me^{\alpha t}$ for $t \geq T$, hence for the second integral we have  
    \begin{align*}
        \left|\int_T^\infty e^{-st}f(t)dt\right| &\leq \int_T^\infty e^{-st}|f(t)|dt\\
        &\leq \int_T^\infty Me^{-st}e^{\alpha t}dt\\
        &= \int_T^\infty Me^{(\alpha - s)t}dt\\
    \end{align*}
    Now this integral converges when $s \geq \alpha$, if $\alpha > s$, then $e^{(\alpha - s)} \rightarrow \infty$. Therefore, the integral 
    \[\int_0^\infty e^{-st}f(t)dt\]
    converges for $s \geq \alpha$.
\end{proof}

\begin{theorem}
    The Laplace transform and its inverse are linear operators; that is, for arbitrary functions $f$ and $g$ with transformations $F$ and $G$, and an arbitrary constant $c$, we have 
    \[\laplace{f+g} = \laplace{f} + \laplace{g}, \ \laplace{cf} = c\laplace{f}\]
    \[\ilaplace{F+G} = \ilaplace{F} + \ilaplace{G}, \ \ilaplace{cF} = c\ilaplace{F}\]
\end{theorem}

\subsection{Laplace Transforms of Basic Functions}
\begin{enumerate}
    \item $f(t) = 1$, 
    \[F(s) = \laplace{1} = \int_0^\infty e^{-st}dt = \lim_{L\rightarrow \infty} \left(-\frac{1}{s}e^{-sL} + \frac{1}{s}\right) = \frac{1}{s}\]
    \item $f(t) = t$,
    \[\laplace{t} = \int_0^\infty e^{-st}tdt = \lim_{L\rightarrow\infty} \left(-\frac{L}{s}e^{-sL} - \frac{1}{s^2}e^{-sL} + \frac{1}{s^2}\right) = \frac{1}{s^2}\]
    \item $f(t) = t^n$ for some $n \in \nat$, then $\laplace{t^n} = \frac{n!}{s^{n+1}}$. Prove this by induction on $n$, 
    \begin{proof}
        For the base case take $n = 0$, then $t^0 = 1$, and $\laplace{1} = 1/s = 0!/s^{0+1}$. Suppose for $k \geq 1$, 
        \[\laplace{t^k} = \frac{k!}{s^{k+1}}\]
        We want to show 
        \[\laplace{t^{k+1}} = \frac{(k+1)!}{s^{k+1 + 1}}\]
        Notice that
        \begin{align*}
            \frac{d}{ds}\laplace{1} &= \frac{d}{ds}\int_0^\infty e^{-st}dt\\
            &= \int_0^\infty \frac{d}{ds}e^{-st}dt\\
            &= \int_0^\infty -te^{-st}dt\\
            &= \laplace{-t} = -\laplace{t}\tag{Linearty $c = -1$.}
        \end{align*}
        Then, 
        \begin{align*}
            \laplace{t} &= - \frac{d}{ds}\laplace{1}\\
            &= -\frac{d}{ds}\frac{1}{s^2}\\
            &= \frac{1}{s^2}
        \end{align*}
        This pattern continues so forth, 
        \begin{align*}
            \laplace{t^2} &= -\frac{d}{ds}\laplace{t}\\
            &= - \frac{1}{s^2}\\
            &= \frac{2}{s^3} = \frac{2!}{s^{2 + 1}}
        \end{align*}
        By our induction hypothesis we assume this holds for $n = k$, then for $n = k+1$, 
        \begin{align*}
            \laplace{t^{k+1}} &= -\frac{d}{ds} \laplace{t^{k}}\\
            &= -\frac{d}{ds} \frac{k!}{s^{k+1}}\\
            &= (k+1)\frac{k!}{s^{k+1+1}}\\
            &= \frac{(k+1)!}{s^{k+1+1}}
        \end{align*}
        Therefore $\laplace{t^n} = \frac{n!}{s^{n+1}}$ for all $n \in \nat$. 
    \end{proof}
    \item $f(t) = e^{at}$ for any $a \in \complex$. 
    \[\laplace{e^{at}} = \int_0^\infty e^{at}e^{-st} = \int_0^\infty e^{(a-s)t}dt = \lim_{L\rightarrow \infty} \frac{1}{a-s}e^{(a-s)L} - \frac{1}{a-s}\]
    As $L \rightarrow \infty$, $s \geq a$ so $e^{(a-s)L} \rightarrow 0$, 
    \[\laplace{e^{at}} = \frac{1}{s-a}, \ s > a\]
\end{enumerate}
\noindent
\textbf{Example.} Find 
\[\ilaplace{\frac{2s+3}{s^2-2s+2}}\]
\textbf{Solution.} We can decompose the fraction to get 
\[\frac{2s+3}{s^2-2s + 2} = \frac{2s+3}{(s-1)(s+2)} = \frac{7}{s-2} - \frac{5}{s-1}\]
Then, using linearity we get 
\[\ilaplace{\frac{7}{s-2} - \frac{5}{s-1}} = 7\ilaplace{\frac{1}{s-2}} -5\ilaplace{\frac{1}{s-1}} = 7e^{2t} - 5e^t\]

\begin{theorem}
    Let $a \in \real$, then 
    \[\laplace{\cos(at)} = \frac{s}{s^2 + a^2} \iff \ilaplace{\frac{s}{s^2+a^2}} = \cos(at)\]
    and 
    \[\laplace{\sin(at)} = \frac{a}{s^2+a^2} \iff \ilaplace{\frac{a}{s^2+a^2}} = \sin(at)\]
\end{theorem}
\begin{proof}
    This is easily verifiable from the definition of laplace transforms with $f(t) = \sin(at)$. 
\end{proof}
\noindent
\textbf{Example.} The Laplace transform of $f(t) = -3e^{-2t} + \frac{1}{2}t^4 + 6\sin(4t)$ is 
\[\laplace{-3e^{-2t} + \frac{1}{2}t^4 + 6\sin(4t)} = -3\laplace{e^{-2t}} + \frac{1}{2}\laplace{t^4} + 6\laplace{\sin(4t)}\]
Then these are all known Laplace transforms and we find 
\[\frac{-3}{s+2} + \frac{1}{2}\frac{4!}{s^5} + 6 \cdot \frac{4}{s^2 + 4^2}\]
\textbf{Example.} Find 
\[\ilaplace{\frac{2s+3}{s^2+2}}\]
\textbf{Solution.} 
\begin{align*}
    \ilaplace{\frac{2s+3}{s^2+2}} &= 2\ilaplace{\frac{s}{s^2+2}} + 3\ilaplace{\frac{1}{s^2+2}}\\
    &=  2\ilaplace{\frac{s}{s^2+2}} + \frac{3}{\sqrt{2}}\ilaplace{\frac{\sqrt{2}}{s^2+2}} \\
    &= 2\cos(\sqrt{2}t) + \frac{3}{\sqrt{2}}\sin(\sqrt{2}t)
\end{align*}

\section{The Unit Step Function}

\begin{definition}
    Let $a > 0$. The unit step function at $t = a$, also known as the \emph{Heaviside} function, is denoted by $u(t-a)$ defined by 
    \[u(t-a) = \begin{cases}
        0 &  0 \leq t < a\\
        1 & t \geq a 
    \end{cases}\]
\end{definition}

We can compute the Laplace transform of this function 
\begin{align*}
    \laplace{u(t-a)} &= \int_0^\infty e^{-st} u(t-a) dt\\
    &= \int_0^\infty e^{-st}u(t-a)dt + \int_a^\infty e^{-st}u(t-a)dt\\
    &=\int_a^\infty e^{-st}dt\\
    &= \lim_{L\rightarrow\infty} \left(-\frac{1}{s}e^{-sL} + \frac{1}{s}e^{-sa}\right) \\
    &= \frac{e^{-as}}{s}
\end{align*}
So the Laplace transform for the Heaviside function is
\[\laplace{u(t-a)} = \frac{e^{-as}}{s}, \ s > 0\]
or equivalently 
\[\ilaplace{\frac{e^{-as}}{s}} = u(t-a)\]
\textbf{Example.} Compute 
\[\laplace{\frac{2}{3}u(t-4) + \frac{1}{e^{3t}}}+\cos(t)\]
\textbf{Solution.} 
\begin{align*}
    \laplace{\frac{2}{3}u(t-4) + \frac{1}{e^{3t}}}+\cos(t) &= \frac{2}{3}\laplace{u(t-4)} + \laplace{e^{-3t}} + \laplace{\cos t}\\
    &= \frac{2}{3}\cdot \frac{e^{-4s}}{s} + \frac{1}{s + 3} + \frac{s}{s^2+1}
\end{align*}
\section{Dirac Delta Function}
\begin{definition}
    Let $a,k > 0$, define the function $f_{k,a}(t)$ as 
    \[f_{k,a}(t) = \frac{1}{k}\left[u(t-a) - u(t-(a+k))\right] = \begin{cases}
        0 & 0 \leq t < a\\
        \frac{1}{k} & a \leq t < a + k\\
        0 & t \geq a + k
    \end{cases}\]
    Then we define the Dirac delta function at $t = a$ as  
    \[\delta(t-a) = \lim_{k\rightarrow 0} f_{k,a}(t) = \begin{cases}
        0 & 0 \leq t < a \\
        \infty  & a \leq t < a + k\\
        0 & t \geq a + k 
    \end{cases} \rightarrow \begin{cases}
        0 & t \neq a\\
        \infty & t = a
    \end{cases}\]
\end{definition}
Then we can compute the Laplace transform for the Dirac delta function 
\begin{align*}
    \laplace{\delta(t-a)} &= \laplace{\lim_{k\rightarrow 0} \frac{1}{k}\left[u(t-a) - u(t-(a+k))\right] }\\
    &= \lim_{k\rightarrow 0} \frac{1}{k} \laplace{u(t-a) + u(t-(a+k))}\\
    &= \lim_{k\rightarrow 0} \frac{1}{k} \left[\frac{e^{-as}}{s} - \frac{e^{-(a+k)s}}{s}\right]\\
    &= \lim_{k\rightarrow 0}\frac{e^{-as}}{s}\left[\frac{1-e^{-ks}}{k}\right]\\
    &= \frac{e^{-as}}{s}\lim_{k\rightarrow 0}\left(\frac{1-e^{-ks}}{k}\right)\\
    &= \frac{e^{-as}}{s}\lim_{k\rightarrow 0}\left(\frac{se^{-ks}}{1}\right)\tag{L'Hopital's Rule}\\
    &= e^{-as}
\end{align*}
Therefore, the Laplace transform of the Dirac delta function is
\[\laplace{\delta(t-a)} = e^{-as} \implies \ilaplace{e^{-as}} = \delta(t-a)\]
\section{The Two Shifting Theorems}
\begin{theorem}[First Shifting Theorem]
    If $F(s) = \laplace{f(t)}$ and $a \in \real$, then 
    \[F(s-a) = \laplace{e^{at}f(t)} \iff \ilaplace{F(s-a)} = e^{at}f(t)\]
\end{theorem}
\begin{proof}
    The proof follows from the definition of the Laplace transform for $F(s-a)$, 
    \begin{align*}
        F(s-a) &= \int_0^\infty e^{-(s-a)t}f(t)dt\\
        &= \int_0^\infty e^{-st}e^{at}f(t)dt\\
        &= \laplace{e^{at}f(t)}
    \end{align*}
\end{proof}
\noindent
\textbf{Example.} Find 
\[\laplace{e^{-2t}u(t-3)}\]
\textbf{Solution.} Set $F(s) = \laplace{u(t-3)} = \frac{e^{-3s}}{s}$, then 
\[\laplace{e^{-2t}u(t-3)} = F(s+2) = \frac{e^{-3(s+2)}}{s+2}\]
\noindent
\textbf{Example.} Find 
\[\laplace{e^{-2t}\cos(3t)}\]
\textbf{Solution.} Set $F(s) = \laplace{\cos(3t)} = \frac{s}{s^2 + 9}$. Then by the first shifting theorem, 
\[\laplace{e^{-2}t\cos(3t)} = F(s + 2) = \frac{s+2}{s^2+4s+13}\] 

\noindent
\textbf{Example.} Find 
\[\ilaplace{\frac{1}{(s+3)^6}}\]
\textbf{Solution.} We know that 
\[\ilaplace{\frac{1}{s^6}} = \frac{1}{5!} \ilaplace{\frac{1}{s^6}} = \frac{t^5}{5!}\]
So, from the first shifting theorem with $a = -3$, we have 
\[\ilaplace{\frac{1}{(s+3)^6}} = \frac{1}{5!}t^5e^{-3t}\]
\noindent
\textbf{Example.} Find 
\[\ilaplace{\frac{2s+3}{s^3 + 5s^2 + 8s + 4}}\]
\textbf{Solution.} Start by decomposing the fraction to get 
\[\frac{2s+3}{s^3+5s^2+8s + 4} = \frac{2s+3}{(s+1)(s+2)^2} = \frac{1}{s+1} - \frac{1}{s+2}  + \frac{1}{(s+2)^2}\]
Then, 
\begin{align*}
    \ilaplace{\frac{2s+3}{s^3+5s^2 + 8s+ 4}} &= \ilaplace{\frac{1}{s+1}} - \ilaplace{\frac{1}{s+2}} + \ilaplace{\frac{1}{(s+2)^2}}\\
    &= e^{-t} - e^{-2t} + te^{-2t}
\end{align*}
Note that $i\laplace{\frac{1}{s^2}} = t$, so we apply first shifting theorem to get 
\[\ilaplace{\frac{1}{(s+2)^2}} = te^{-2t}\]

\begin{theorem}[Second Shifting Theorem]
    Let $F(s) = \laplace{f(t)}$, and $a > 0$, then 
    \[\laplace{u(t-a)f(t-a)} = e^{-as}F(s) \iff \ilaplace{e^{-as}F(s)} = u(t-a)f(t-a)\]
\end{theorem}
\noindent
\textbf{Example.} Find
\[\laplace{u(t-1)(-3t^2 + 2t-1)}\]
\textbf{Solution.} From second shifting theorem, we have 
\[f(t-1) = (-3t^2 + 2t - 1)\]
Then we can compute $F(s) = \laplace{f(t)}$, so we must find $f(t) = f(t-1+1)$.
\[f(t) = -3(t+1)^2 + 2(t+1) - 1 = -3t^2 -4t -2\]
Now we can apply the second shifting theorem 
\begin{align*}
    \laplace{u(t-1)(-3t^2 + 2t - 1)} &= e^{-s}\laplace{-3t^2-4t-2}\\
    &= e^{-s}\left(-3\frac{2}{s^3} - 4\frac{1}{s^2} - \frac{2}{s}\right)\\
    &= e^{-s}\left(\frac{6}{s^3} - \frac{4}{s^2} - \frac{2}{s}\right)
\end{align*}
\noindent
\textbf{Example.} Find 
\[\laplace{u(t-\pi)\sin(t)}\]
\textbf{Solution.} We can use the second shifting theorem wiht $a = \pi$, and 
\[f(t-\pi) = \sin t \implies f(t) = \sin(t + \pi) = -\sin(t)\]
Then 
\[\laplace{(t-\pi) = e^{-\pi s} \laplace{-\sin t}} = -e^{-\pi s} \frac{1}{s^2+1}\]
\noindent
\textbf{Example.} Find 
\[\ilaplace{\frac{e^{-s}}{s^2-5s+6}}\]
\textbf{Solution.} Recall from the second shifting theorem for inverse laplace transforms, 
\[\ilaplace{e^{-as}F(s)} = u(t-a)f(t-a)\]
Here, $F(s) = \frac{1}{s^2 - 5s + 6}$, then 
\[f(t) = \ilaplace{\frac{1}{(s-2)(s-3)}} = \ilaplace{\frac{1}{s-3}} - \ilaplace{\frac{1}{s-2}} = e^{3t} - e^{-t}\]
Then, 
\[\ilaplace{\frac{e^{-s}}{s^2-5s+16}} = u(t-1)\left(e^{3(t-1)} - e^{2(t-1)} \right)\]


\section{Convolutions}
\begin{definition}
    Let $f(t)$,$g(t)$ be 2 functions on the positive real numbers, we define the \emph{convolution} of $f(t)$ and $g(t)$ as 
    \[(f*g)(t) = \int_0^t f(x)g(t-x)dx \]
\end{definition}
Note that convolutions are commutative $f*g = g*f$, and associate $f*(g*h) = (f*g) * h$.

\begin{theorem}[Convolution Thoerem]
    Let $F(s) = \laplace{f(t)}$ and $G(s) = \laplace{g(t)}$. Then, 
    \[\laplace{f(t)*g(t)} = F(s)\cdot G(s)\]
    or equivalently
    \[\ilaplace{F(s) \cdot G(s)} = f(t)*g(t)\]
\end{theorem}

\textbf{Example.} Find 
\[\ilaplace{\frac{1}{s^2 + 4s - 5}}\]
\textbf{Solution.} We can rewrite the fraction and use the convolution, 
\begin{align*}
    \ilaplace{\frac{1}{s^2 + 4s - 5}} &= \ilaplace{\frac{1}{s-1} \frac{1}{s+5}} \\
    &= \ilaplace{\frac{1}{s-1}} * \ilaplace{\frac{1}{s+5}}\\
    &= e^{t} * e^{-5t}\\
    &= \int_0^t e^{t-x}e^{-5x}dx\\
    &= \int_0^t e^{t}e^{-x}e^{-5x}dx\\
    &= e^t \int_0^t e^{-6x}dx\\
    &= e^t\left(-\frac{1}{6}e^{-6t} + 1\right)\\
    &= \frac{1}{6}e^t - \frac{1}{6}e^{-5t}
\end{align*}
\noindent
\textbf{Example.} Find 
\[\laplace{(u(t-2)(-t^2+1))*\cos(3t)}\]
\textbf{Solution.} From the convolution theorem, 
\begin{align*}
    \laplace{(u(t-2)(-t^2+1))*\cos(3t)} &= \laplace{(u(t-2)(-t^2+1))}\laplace{\cos(3t)}\\
    &= e^{-2s}\laplace{-t^2-4t-3}\frac{s}{s^2+9}\\
    &= e^{-2s}\left(-\frac{2}{s^3} - \frac{4}{s^2} - \frac{3}{s}\right)\frac{s}{s^2+9}
\end{align*}
\noindent
\textbf{Example.} Find 
\[\ilaplace{\frac{1}{(s^2+1)^2}}\]
\textbf{Solution.} We can use the convolution theorem 
\begin{align*}
    \ilaplace{\frac{1}{(s^2+1)^2}} &= \ilaplace{\frac{1}{s^2+1}\frac{1}{s^2+1}}\\
    &= \ilaplace{\frac{1}{s^2 + 1}} * \ilaplace{\frac{1}{s^2+1}}\\
    &= \sin t * \sin t\\
    &= \int_0^t \sin x \sin (t - x)dx\\
    &= \int_0^t \sin x (\sin t \cos x - \cos t \sin x)dx\\
    &= \int_0^t \sin x \sin t \cos x - \sin x\cos t \sin x dx\\
    &= \sin t\int_0^t \sin x \cos xdx - \cos t \int_0^t \sin^2 x dx \\
    &= \frac{\sin t}{2}\int_0^t \sin(2x)dx - \frac{\cos t}{2}\int_0^t 1 - \cos (2x)\\
    &= \frac{\sin t}{2}\left[-\frac{\cos(2x)}{2}\right]_0^t - \frac{\cos t}{2}\left[x - \frac{\sin(2x)}{2}\right]_0^t\\
    &= \frac{\sin t}{2}\left[-\frac{\cos(2t)}{2} + \frac{1}{2}\right] - \frac{\cos t}{2}\left[t - \frac{\sin(2t)}{2}\right]\\
    &= -\frac{\sin t}{4} \cos(2t) + \frac{1}{4}\sin t - \frac{1}{2}t \cos t + \frac{1}{4} \cos t\sin(2t)\\
    &= \frac{1}{2}\sin t - \frac{1}{2}t\cos t
\end{align*}

\noindent


\textbf{Example.} Find 
\[\laplace{\left(u(t-2)\left(-\frac{1}{2}t^2 + t\right)\right)*\sin(3t)}\]
\textbf{Solution.} By the convolution theorem, 
\[\laplace{\left(u(t-2)\left(-\frac{1}{2}t^2 + t\right)\right)*\sin(3t)} = \laplace{u(t-2)\left(-\frac{1}{2}t^2 + t\right)}\laplace{\sin(3t)}\]
From the second shifting theorem, 
\[f(t-2) = -\frac{1}{2}t^2 + t \implies f(t) = -\frac{1}{2}(t+2)^2 + t + 2 = -\frac{1}{2}t^2 - t\]
So, 
\[\laplace{u(t-2)f(t-2)} = e^{-2s}\laplace{f(t)} = -e^{2s}\left(\frac{1}{s^3} + \frac{1}{s^2}\right)\]
Thus, 
\[\laplace{\left(u(t-2)\left(-\frac{1}{2}t^2 + t\right)\right)*\sin(3t)} = -e^{-2s}\left(\frac{1}{s^3} - \frac{1}{s^2}\right)\frac{3}{s^2 + 9} \]
\begin{theorem}
    Let $F(s) = \laplace{f(t)}$, and $n \geq 1$ be an integer, then 
    \[\laplace{t^nf(t)} = (-1)^nF^{(n)}(s)\]
\end{theorem}
\begin{proof}
    TBD.
\end{proof}
\noindent
\textbf{Example.} Find 
\[\laplace{t^2\cos(2t)}\]
\textbf{Solution.}
Using the above theorem, set 
\[F(s) = \laplace{\cos(2t)} = \frac{s}{s^2 + 4}\]
So, 
\[\laplace{t^2\cos(2t)} = (-1)^2F''(s) = -\frac{2s(s^2+4)-4s(-s^2+4)}{(s^2+4)^3}\]
\noindent
\textbf{Example.} Find 
\[\laplace{te^{-2t}\cos(3t)}\]
\textbf{Solution.} To be able to use the theorem, first we have to find 
\[F(s) = \laplace{e^{-2t}\cos(3t)}\]
Using first shifting theorem, 
\[F(s) = \frac{s+2}{(s+2)^2 + 9} =  \frac{s + 2}{s^2 + 4s + 13}\]
Now using the above theorem, 
\[\laplace{te^{-2t}\cos(3t)} = (-1)^1F'(s) = -\frac{(s^2+4s+13)-(s+2)(2s+4)}{(s^2+4s+13)^2}\]

\begin{theorem}
    Let $f(t)$ be a function such that 
    \[\lim_{t\rightarrow 0} \frac{f(t)}{t} = L < \infty\]
    If $F(s) = \laplace{f(t)}$, then
    \[\laplace{\frac{f(t)}{t}} = \int_s^\infty F(x)dx\]
\end{theorem}
\noindent
\textbf{Example.} Find 
\[\laplace{\frac{\sin t}{t}}\]
\textbf{Solution.} We can check that the limit exists 
\[\lim_{t\rightarrow 0}\frac{\sin t}{t} = \frac{0}{0} \implies \lim_{t\rightarrow 0} \frac{\sin t}{t} = \lim_{t\rightarrow 0}\frac{\cos t}{1} = 1 < \infty\]
Then applying the theorem, 
\begin{align*}
    \laplace{\frac{\sin t}{t}} &= \int_s^\infty \frac{1}{x^2 + 1}dx\\
    &= \lim_{L \rightarrow \infty} \int_s^L \frac{1}{x^2+1}dx\\
    &= \lim_{L\rightarrow \infty} \left[\arctan x\right]_s^L \\
    &= \lim_{L \rightarrow \infty} (\arctan L - \arctan s )\\
    &= \frac{\pi}{2} - \arctan s
\end{align*}
\noindent

\textbf{Example.} Find 
\[\laplace{\frac{e^t - 1}{t}}\]
\textbf{Solution.} Same as the previous example, 
\[\lim_{t\rightarrow 0} \frac{e^t - 1}{t} = \frac{0}{0} \implies \lim_{t\rightarrow 0}\frac{e^{t}-1}{t} = \lim_{t\rightarrow 0}\frac{e^t}{1} = 1\]
Then applying the theorem, 
\begin{align*}
    \laplace{\frac{e^t - 1}{t}} &= \int_s^\infty \left(\frac{1}{x-1} -  \frac{1}{x}\right)dx\\
    &= \lim_{L\rightarrow \infty} \int_s^L \left(\frac{1}{x-1} - \frac{1}{x}\right)dx\\
    &= \lim_{L\rightarrow\infty}\left[\ln(x-1) - \ln x\right]_s^L\\
    &= \lim_{L \rightarrow \infty} \left(\ln\left(\frac{L-1}{L}\right) - \ln\left(\frac{s-1}{s}\right)\right)\\
    &= - \ln\left(\frac{s-1}{s}\right)\\
    &= \ln\left(\frac{s}{s-1}\right)
\end{align*}

\subsection{Laplace Transforms of Integerals}
Given a function defined as 
\[f(t) = \int_0^t g(x)dx\]
our goal is to find the laplace transform of $f(t)$. 

\begin{theorem}
    Let $g$ be an integral function on $[0,t]$, for some $t \in \real$, with $F(s) = \laplace{g}$. Then 
    \[\laplace{\int_0^t g(x)dx} = \frac{F(s)}{s} \] 
\end{theorem}

\textbf{Example.} Find 
\[f(t) = \int_0^t \sin(2x)dx\]
\textbf{Solution.} Simply from our theorem, 
\[\laplace{f(t)} = \frac{\frac{2}{s^2+4}}{s} = \frac{2}{s(s^2+4)}\]

\section{Using Laplace Transforms to Solve IVP's}

\begin{theorem}
    Let $y(t)$ be a differentiable function, then 
    \[\laplace{y'(t)} = s\laplace{y(t)} - y(0)\]
\end{theorem}
\begin{proof}
    From the definition, 
    \[
        \laplace{y'(t)} = \int_0^\infty e^{-st}y'(t)dt = \lim_{L\rightarrow \infty}\int_0^L e^{-st}y'(t)dt
    \]
    Solving this integtral, we get 
    \[\int e^{-st}y'(t)dt = e^{-st}y(t) + s\int e^{-st}y(t)dt \]
    Computing the limits gives us 
    \[\int_0^L e^{-st}y'(t)dt = e^{-sL}y(L) - y(0) + s\int_0^L e^{-st}y(t)dt\]
    Now taking the limit as $L \rightarrow \infty$, 
    \[\laplace{y'(t)} = \lim_{L\rightarrow\infty} e^{-sL}y(L) - y(0) + s\int_0^L e^{-st}y(t)dt = s\int_0^\infty e^{-st}y(t)dt - y(0)\]
    So 
    \[\laplace{y'(t)} = s\laplace{y(t)} - y(0)\]
\end{proof}
Note that this can be extended to each order derivative, 
\[\laplace{y''(t)} = s\laplace{y'(t)} - y'(0) = s\left(s\laplace{y(t)} - y(0)\right) - y(0)\]
\[\laplace{y'''(t)} = s^3\laplace{y(t)} - s^2y(0) - sy'(0)-y''(0)\]
In general, 
\[\laplace{y^{(n)}(t)} = s^n\laplace{y(t)} - s^{n-1}y(0) - s^{n-2}y'(0) - \cdots - y^{(n-1)}(0)\]
\subsection{Steps to Solving IVP}
\begin{enumerate}
    \item Set $Y = \laplace{y(t)}$. 
    \item Apply the laplace transform to both sides of the ODE. 
    \item Isolate $Y$ from the after taking the laplace transform. 
    \item Then
    \[y(t) = \ilaplace{Y}\]
\end{enumerate}

\noindent
\textbf{Example.} Solve the IVP 
\[y'' - y = \delta(t-2), \ y(0) = 0, \ y'(0) = 1\]
\noindent
\textbf{Solution.} Set $Y(s) = \laplace{y(t)}$, then take the laplace transform of the ODE, 
\begin{align*}
    &\laplace{y''} - \laplace{y} = \laplace{\delta(t-2)}\\
    \implies& s^2Y - sY(0) - y'(0) - Y = e^{-2s}\\
    \implies& s^2Y - 1 - Y = e^{-2s}\\
    \implies& (s^2 - 1)Y = e^{-2s} + 1\\
    \implies& Y = \frac{e^{-2s} + 1}{s^2-1}
\end{align*}
Now we compute the inverse laplace transform, 
\begin{align*}
    \ilaplace{Y} &= \ilaplace{\frac{e^{-2s} + 1}{s^2 - 1}}\\
    &= \ilaplace{\frac{e^{-2s}}{s^2-1}} + \ilaplace{\frac{1}{s^2-1}}\\
    &= \ilaplace{\frac{1}{(s-1)(s+1)}} + \ilaplace{\frac{e^{-2s}}{s^2-1}}\\
    &= \frac{1}{2}\ilaplace{\frac{1}{s+1}} - \frac{1}{2}\ilaplace{\frac{1}{s-1}} + \ilaplace{\frac{e^{-2s}}{s^2-1}}\\
    &= \frac{1}{2}e^t - \frac{1}{2}e^{-t} + \ilaplace{\frac{e^{-2s}}{s^2-1}}\\
    &= \frac{1}{2}e^{t} - \frac{1}{2}e^{-t} + u(t-2)\left[\frac{1}{2}e^{t-2} - \frac{1}{2}e^{-(t-2)}\right]\\
\end{align*}
Therefore, the solution to the IVP is 
\[y(t) = \frac{1}{2}e^{t} - \frac{1}{2}e^{-t} + u(t-2)\left[\frac{1}{2}e^{t-2} - \frac{1}{2}e^{-(t-2)}\right]\]

\end{document}





